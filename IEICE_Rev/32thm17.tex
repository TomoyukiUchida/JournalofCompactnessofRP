From Lemma \ref{Add-Lemma01} 
and Lemma \ref{Sato1:Lemma15}, 
we have the following theorem.


\begin{thm}\label{Theorem17}
    Let $k \geq 3$, $\sharp\Sigma \geq 2k-1$, $P \in \RPatplus$ and
    $Q \in \RPatkei$.
    Then, the following (i),(ii) and (iii) are equivalent:
    \[
        \begin{tabular}{ll}
            $(\mathrm{i})$ $S_{2}(P) \subseteq L(Q)$,
            $(\mathrm{ii})$ $P \sqsubseteq Q$,
            $(\mathrm{iii})$ $L(P) \subseteq L(Q)$.
        \end{tabular}
    \]
\end{thm}


\begin{proof}
    It is clear that (ii) implies (iii) and (iii) implies (i).
    From Theorem~\ref{Sato1:Theorem10}, 
    if $\sharp\Sigma \geq 2k+1$, then
    (i) implies (ii).
    Let $\sharp Q = k$, $p \in P$, $\sharp\Sigma = 2k-1$ or $2k$.
    Then, we show that (i) implies (ii).
    It suffices to show that $S_{2}(p) \subseteq L(Q)$ implies {\color{red}$\{p\} \sqsubseteq Q$}
    for any regular pattern {\color{red}$p \in P$}.
    The proof is done by mathematical induction on $n$, where $n$ is the number of variable symbols appears in $p$.

    In case $n=0$, $S_{2}(p) = \{p\}$ holds.
    By (i), we have {\color{red}$\{p\} \subset L(Q)$}. Thus, $p \preceq q$ for some $q \in Q$.

    For $n \geq 0$,
    we assume that it is valid for any regular pattern $p$
    with $n$ variable symbols.
    Let $p$ be a regular pattern such that $n+1$ variable symbols appear in $p$
    and $S_{2}(p) \subseteq L(Q)$.
    {\color{red} Let $Q = \{q_{1},\ldots,q_{k}\}$.}
    We assume that $p \not\sqsubseteq Q$, that is, {\color{red}$\{p\} \not\preceq q_{i}$}
    for any $i \in \{1,\ldots,k\}$.
    {\color{red} Let $p_{1}$, $p_{2}$ be regular patterns, $x$ a variable symbol with $p = p_{1}xp_{2}$.}
    For $a, b \in \Sigma$,
    let $p_{a}=p\{x:=a\}$ and $p_{ab}=p\{x:=ab\}$.
    Both $p_{a}$ and $p_{ab}$ have $n$ variable symbols, respectively.
    Thus, $S_{2}(p_{a}) \subseteq L(Q)$ and $S_{2}(p_{ab}) \subseteq L(P)$ hold.
    By the induction hypothesis,
    there exist $i, i' \in \{1,\ldots,k\}$ such that
    $p_{a} \preceq q_{i}$ and $p_{ab} \preceq q_{i'}$.
    Let $D_{i} = \{a \in \Sigma \mid p\{x:=a\} \preceq q_{i}\}$ \ ($i=1,\ldots,k$).
    We assume that $\sharp D_{i} \geq 3$ for some $i \in \{1,\ldots, k\}$.
    By Lemma~\ref{Sato1:Lemma10}, 
    we have $p \preceq q_{i}$.
    This contradicts the assumption.
    Thus, we have $\sharp D_{i} \leq 2$ for any $i \in \{1,\ldots,k\}$.
    If $\sharp\Sigma = 2k-1$, then
    $\sharp D_{i}=2$ or $\sharp D_{i}=1$ for any $i \in \{1,\ldots,k\}$.
    Moreover,
    If $\sharp\Sigma = 2k$, then
    $\sharp D_{i}=2$ for any $i \in \{1,\ldots,k\}$.
    Since $k \geq 3$, {\color{red} $2k-1 \geq k+2$} holds.
    By Lemma~\ref{Add-Lemma01},
    there exists $i \in \{1,\ldots,k\}$ such that $p\{x:=xy\} \preceq q_{i}$.
    Therefore, by Lemma~\ref{Sato1:Lemma15}, 
    we have $p \preceq q_{i}$.
    This contradicts the assumption.
    Thus, (i) implies (ii).
\end{proof}

From Theorem~\ref{Theorem17}, 
the following Corollary~\ref{Col18} holds.


\begin{col}\label{Col18}
    Let $k \geq 3$, $\sharp\Sigma \geq 2k-1$ and $P \in \RPatplus$.
    Then, $S_{2}(P)$ is a characteristic set for $L(P)$ within $\RPatLkei$.
\end{col}


\begin{lem}[Sato et al.\cite{Sato1}]\label{Sato1:Lemma19}%\label{補題19}
    Let $k \geq 3$ and $\sharp\Sigma \leq 2k-2$.
    Then, $\RPatkei$ does not have compactness with respect to containment.
\end{lem}

\begin{proof}
    Let $\Sigma =\{a_{1},\ldots,a_{k-1},b_{1},\ldots,b_{k-1}\}$ and
    $p, q_{i}$ regular patterns, $w_{i} \in \Sigma^{\ast}$\ $(i=1,\ldots,k-1)$
    defined in a similar way to Example~\ref{counter-example4Theorem10}. %\ref{例題1}.
    Let $q_{k}=x_{1}a_{1}w_{1}xyw_{1}b_{1}x_{2}$.
    Since
    $p\{x:=a_{i}\} = x_{1}a_{1}w_{1}a_{i}w_{1}b_{1}x_{2} \preceq q_{i}$ and
    $p\{x:=b_{i}\} = x_{1}a_{1}w_{1}b_{i}w_{1}b_{1}x_{2} \preceq q_{i}$
    for any $i \in \{1,\ldots,k-1\}$,
    we have $S_{1}(p) \subseteq \bigcup_{i=1}^{k-1}L(q_{i})$.
    For any $w \in \{s \in \Sigma^{+} \mid |s| \geq 2\}$,
    $p\{x:=w\}=x_{1}a_{1}w_{1}ww_{1}b_{1}x_{2} \preceq q_{k}$.
    Thus, we have $L(p) \subseteq L(Q)$.
    By Theorem~\ref{regularPatternEquivalence},
    since $p \not \preceq q_{i}$, $L(p) \not \subseteq L(q_{i})$
    for any $i \in \{1,\ldots, k\}$.
    Therefore, $\RPatkei$ does not have compactness with respect to containment.
\end{proof}

From Theorem~\ref{Theorem17} and Lemma~\ref{Sato1:Lemma19}, 
we have the following Theorem~\ref{Theorem5}.

\begin{thm}\label{Theorem5}
    Let $k \geq 3$ and $\sharp\Sigma \geq 2k-1$.
    Then, $\RPatkei$ has compactness with respect to containment.
\end{thm}

In case $k=2$, we have the following theorem.

\begin{thm}\label{Theorem21}
    Let $\sharp\Sigma \geq 4$, $P \in \RPatplus$ and $Q \in \RPat^{2}$.
    The following (i), (ii) and (iii) are equivalent:
    \[
        \mathrm{(i)}\ S_{2}(P) \subseteq L(Q),\ \
        \mathrm{(ii)}\ P \sqsubseteq Q, \ \
        \mathrm{(iii)}\ L(P) \subseteq L(Q).
    \]
\end{thm}

\begin{proof}
    It is clear that (ii) implies (iii), and (iii) implies (i).
    Thus, we show that (i) implies (ii).
    It suffices to show that $S_{2}(p) \subseteq L(Q)$ implies {\color{red}$\{p\} \sqsubseteq Q$}
    for any regular pattern {\color{red}$p \in P$}.
    Let $Q = \{q_{1}, q_{2}\}$.
    The proof is done by mathematical induction on $n$,
    where $n$ is the number of variable symbols appearing in $p$.
    In case $n=0$, $p \in \Sigma^{+}$.
    Since $S_{2}(p) = \{p\} \subseteq L(Q)$, we have $p \preceq q$
    for some $q \in Q$.
    For $n \geq 0$, we assume that it is valid for any regular pattern $p$ with
    $n$ variable symbols.
    Let $p$ be a regular pattern such that $n+1$ variable symbols appear in $p$,
    and $S_{2}(p) \subseteq L(Q)$ {\color{red}holds}.
    We assume that $p \not\preceq q_{i}$ ($i=1,2$).
    Let $p_{1}$, $p_{2}$ be regular patterns and $x$ a variable symbol
    with $p = p_{1}xp_{2}$.
    For $a, b \in \Sigma$,
    let $p_{a} = p\{x:=a\}$ and $p_{ab} = p\{x:=ab\}$.
    Note that $p_{a}$ and $p_{ab}$ have $n$ variable symbols.
    Thus, by the assumption,
    $S_{2}(p_{a}) \subseteq L(Q)$ and $S_{2}(p_{ab}) \subseteq L(Q)$
    imply $p_{a} \preceq q_{i}$ and $p_{ab} \preceq q_{i'}$ for some
    $i, i' \in \{1,2\}$.
    Let $D_{i} = \{a \in \Sigma \mid p\{x:=a\} \preceq q_{i}\}$ $(i=1,2)$.
    By Lemma~\ref{Sato1:Lemma10},
    if $\sharp D_{i} \geq 3$ for some $i \in \{1,2\}$,
    then $p \preceq q_{i}$.
    This contradicts that $p \not\preceq q_{i}$ $(i=1,2)$.
    Thus, we have $\sharp D_{i} \leq 2$ for any $i \in \{1,2\}$.
    Since $\sharp \Sigma \geq 4$,
    we consider that $\sharp D_{1}=2$ and $\sharp D_{2} = 2$.
    From Lemma~\ref{Add-Lemma01},
    $p\{x:=xy\} \preceq q_{i}$ for some $i \in \{1,2\}$.
    From Lemma~\ref{Sato1:Lemma15}, 
    we have $p \preceq q_{i}$ for some $i \in \{1,2\}$.
    This contradicts that $p \not\preceq q_{i}$ ($i=1,2$).
    %Therefore, 
    Hence, (i) implies (ii).
    \end{proof}

%The next example gives a set of regular patterns $P \in \RPatplus$ and a set of regular patterns $Q \in \RPat^{2}$ that, 
%in case $\sharp\Sigma = 3$, 
%the three conditions (i),(ii) and (iii) in Theorem~\ref{Theorem21} are not equivalent.
{\color{red} The following example provides a set of regular patterns $P \in \RPat^{+}$ and a set of regular patterns $Q \in \RPat^{2}$ demonstrating that, when $\sharp\Sigma = 3$, the three conditions (i), (ii), and (iii) stated in Theorem~\ref{Theorem21} are not equivalent.}


\begin{ex}
    Let $\Sigma = \{a,b,c\}$, $p$, $q_{1}$, $q_{2}$ regular patterns and
    $x,x',x''$ variable symbols such that
    $p = x'axbx''$, $q_{1} = x'abx''$ and $q_{2} = x'cx''$.
    Let $w \in \Sigma^{+}$.
    If $w$ contains $c$, then $p \{x:=w\} \preceq q_{2}$.
    On the other hand, if $w$ does not contain $c$,
    then $p\{x:=w\} \preceq q_{1}$.
    Thus, $L(p) \subseteq L(q_{1}) \cup L(q_{2})$.
    However, $p \not\preceq q_{1}$ and $p \not\preceq q_{2}$.
\end{ex}

From Theorem~\ref{Theorem21}, 
the following two corollaries holds.

\begin{col}
    Let $\sharp\Sigma \geq 4$ and $P \in \RPatplus$.
    Then, $S_{2}(P)$ is a characteristic set for $L(P)$ within $\RPatL^{2}$.
\end{col}

\begin{col}
    Let $\sharp\Sigma \geq 4$.
    Then, $\RPat^{2}$ has compactness with respect to containment.
\end{col}
