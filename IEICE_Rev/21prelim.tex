\section{Preliminaries}

\subsection{Basic definitions and notations}\label{subsec:basicdef}

%$A\subsetneq B$  $A\subseteq B$
% Preliminaries
Let $\Sigma$ be a non-empty finite set of constant symbols.
Let $X$ be an infinite set of variable symbols such that $\Sigma \cap X = \emptyset$ holds.
Then, a \textit{string} {\color{red}over} $\Sigma \cup X$ is a sequence of symbols in $\Sigma \cup X$.
Particularly, the string having no symbol is called the \textit{empty string} and is denoted by $\varepsilon$.
We denote by $(\Sigma \cup X)^{\ast}$ the set of all strings {\color{red}over} $\Sigma \cup X$ 
and by $(\Sigma \cup X)^{+}$ the set of all strings {\color{red}over} $\Sigma\cup X$ except $\varepsilon$, i.e., $(\Sigma \cup X)^{+}=(\Sigma \cup X)^{\ast}\setminus \{\varepsilon\}$.

A \textit{pattern} {\color{red}over} $\Sigma \cup X$ is a string in $(\Sigma \cup X)^{\ast}$.
Note that the empty string $\varepsilon$ is a pattern {\color{red}over} $\Sigma \cup X$.
A pattern $p$ is said to be \textit{regular} if each variable symbol appears
at most once in $p$.
The length of $p$, denote by $|p|$, is the number of symbols in $p$.
Note that $|\varepsilon|=0$ holds.
{\color{red} The sets of all patterns and regular patterns {\color{red}over} $\Sigma \cup X$ are denoted by $\Pat_{\Sigma\cup X}$ and $\RPat_{\Sigma\cup X}$, respectively.
When $\Sigma$ and $X$ are clear from the context, 
we omit them in the notation and simply write $\Pat$ and $\RPat$, respectively.}
For a set $S$, we denote by $\sharp S$ the number of elements in $S$.
Let $p,q$ be strings.
If $p$ and $q$ are equal as strings, we denote it by $p=q$.
We denote by $p\cdot q$ the string obtained from $p$ and $q$ by concatenating $q$ after $p$.
Note that for a string $p$ and the empty string $\varepsilon$, $p\cdot \varepsilon = \varepsilon \cdot p = p$.

A substitution $\theta$ is a mapping from $(\Sigma \cup X)^{\ast}$ to $(\Sigma \cup X)^{\ast}$ such that
(1) $\theta$ is a homomorphism with respect to string concatenation, i.e., $\theta(p \cdot q) = \theta(p) \cdot \theta(q)$ holds for patterns $p$ and $q$,
(2) $\theta(\varepsilon)=\varepsilon$ holds,
(3) for each constant symbol $a \in \Sigma$, $\theta(a) = a$ holds,
and (4) for each variable symbol $x \in X$, $|\theta(x)| \geq 1$ holds.
Let $x_{1},\ldots,x_{n}$ are variable symbols and $p_{1},\ldots,p_{n}$ non-empty patterns.
The notation $\{x_{1}:=p_{1},\ldots,x_{n}:=p_{n}\}$ denotes a substitution that replaces each variable symbol $x_{i}$
with a non-empty pattern $p_{i}$ for each $i \in \{1,\ldots,n\}$.
For a pattern $p$ and a substitution $\theta=\{x_{1}:=p_{1},\ldots,x_{n}:=p_{n}\}$, we denote by $p\theta$ a new pattern obtained from $p$ by replacing variable symbols $x_1,\ldots,x_n$ in $p$ with patterns $p_1,\ldots,p_n$ according to $\theta$, respectively.

For a pattern $p$ and $q$,
the pattern $q$ is a \textit{generalization} of $p$, or $p$ is an \textit{instance} of $q$, denoted by $p \preceq q$,
if there exists a substitution $\theta$ such that $p = q\theta$ holds.
If $p \preceq q$ and $p \succeq q$ hold, we denote it by $p \equiv q$.
The notation $p \equiv q$ means that $p$ and $q$ are equal as strings except for variable symbols. 
For a pattern $p$, the \textit{pattern language} of $p$, denoted by $L(p)$, is the set $\{w \in \Sigma^{\ast} \mid w \preceq p\}$.
For patterns $p$ and $q$, it is clear that $L(p) = L(q)$ if $p \equiv q$, and $L(p) \subseteq L(q)$ if $p \preceq q$.
Note that $L(\varepsilon) = \{\varepsilon\}$.
In particular, if $p$ is a regular pattern, we say that $L(p)$ is a \textit{regular pattern language}.
The sets of all pattern languages and regular patterns languages are denoted by $\PatL$ and $\RPatL$, respectively.

\begin{lem}[Mukouchi(Theorem 6.1, \cite{Mukouchi1991})]\label{regularPatternEquivalence}
  Suppose $\sharp \Sigma \geq 3$. Let $p$ and $q$ be regular patterns.
  Then $p \preceq q$ if and only if $L(p) \subseteq L(q)$.
\end{lem}

Next, we consider unions of pattern languages. % or regular pattern languages.
The class of all non-empty finite subsets of $\Pat$ is denoted by $\Patplus$, i.e.,
$\Patplus = \{P \subseteq \Pat \mid 0 < \sharp P < \infty\}$.
For a positive integer $k$ i.e., $k>0$, the class of non-empty sets consisting of at most $k$ patterns, i.e.,
$\Patkei = \{P \subseteq \Pat \mid 0< \sharp P \leq k\}$.
For a set $P$ of patterns, the pattern language of $P$, denoted by $L(P)$, is the set $\bigcup_{p \in P}L(p)$.
We denote by $\PatLkei$ the class of unions of at most $k$ pattern languages,
i.e., $\PatLkei = \{L(P) \mid P \in \Patkei\}$.
In a similar way, we also define $\RPatplus$, $\RPatkei$ and $\RPatLkei$.
For $P$, $Q$ in $\Patplus$,
the notation $P \sqsubseteq Q$ means that for any $p \in P$ there is a pattern $q \in Q$ such that $p \preceq q$ holds.
It is clear that $P \sqsubseteq Q$ implies $L(P) \subseteq L(Q)$.
However, the converse is not valid in general.