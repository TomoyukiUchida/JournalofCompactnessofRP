%\section{非隣接変数正規パターン}
\section{Regular Pattern without Adjacent Variable Symbols}

%隣接した変数記号を持たない正規パターンを\textbf{非隣接変数正規パターン}という.
A regular pattern $p$ is said to be a {\it non-adjacent variable regular pattern} ($\NAV$ regular pattern)  
if $p$ does not contain consecutive variable symbols.
For example, the regular pattern $p=axybc$ is not a $\NAV$ regular pattern because $xy$ is appeared in $p$.
%例えば,パターン$axybc$は正規パターンであるが,非隣接変数正規パターンではない.パターン$axbcy$は非隣接変数正規パターンである.
%$\NAVRP$を非隣接変数正規パターン全体の集合とする.
Let $\NAVRP$ be the set of all $\NAV$ regular patterns.
%$\NAVRP$の空でない有限部分集合の集合を$\NAVRPplus$で,
%高々$k~(k\geq 1)$個のパターンから成る$\NAVRP$の部分集合$\{P\in \NAVRPplus \mid \sharp P \leq k\}$を$\NAVRPkei$で表す.
Let $\NAVRPplus$ be the set of all finite subsets $S$ of $\NAVRP$ such that $S$ is not the empty set, i.e., $\NAVRPplus=\{S \subseteq \NAVRP \mid \sharp S \leq 1\}$,
and $\NAVRPkei$ the set of all subsets $P$ of $\NAVRPplus$ such that $P$ consists of at most $k~(k\geq 1)$ $\NAV$ regular patterns, i.e., $\NAVRPkei=\{P\in \NAVRPplus \mid \sharp P \leq k\}$.
%このとき,次の定理が成り立つ.
Then, we have the following Theorem \ref{KeyTheoforNAVRP}.

\begin{thm}\label{KeyTheoforNAVRP}%\label{非隣接kが4以上}
  $\sharp \Sigma \ge k+2,~P\in \NAVRPplus,~Q \in \NAVRPkei$とする.
Let $\sharp \Sigma \ge k+2,~P\in \NAVRPplus,~Q \in \NAVRPkei$.  
%このとき,以下の{\rm (i), (ii), (iii)}は同値である.
Then, the following (i), (ii) and (iii) are equivalent:
\[
\begin{tabular}{ll}
(i) $S_{2}(P) \subseteq L(Q)$,
(ii) $P \sqsubseteq Q$,
(iii) $L(P) \subseteq L(Q)$.
\end{tabular}
\]
\end{thm}

\begin{proof}
%定義より,
%(ii) $\Rightarrow$ (iii)と(iii) $\Rightarrow$ (i)は自明に成り立つ.
From the definitions of $\NAVRPplus$ and $\NAVRPkei$, it is clear that (ii) implies (iii)  and  (iii) implies (i).
Hence, we will show that (i) implies (ii) 
by mathematical induction on the number $n$ of variable symbols that appear in a $\NAV$ regular pattern $p\in P$ as follows:
%よって,(i) $\Rightarrow$ (ii)が成り立つことを,$p$に現れる変数記号の数$n$に関する数学的帰納法で証明する.
%
%$n=0$のとき,$S_{2}(p)= \{ p \}$であり,$p \in L(Q)$となる.よって,ある$q \in Q$に対して,$p \preceq q$となる.
If $n=0$, then we have $S_{2}(\{p\})= \{ p \}$.
Hence, $p \in L(Q)$.
Therefore, there exists $q \in Q$ such that $p \preceq q$ holds.

If $n \ge 0$, we assume that the proposition holds for any regular $\NAV$ regular pattern containing $n \ge 0$ variable symbols.
%$n \ge 0$個の変数記号を含む任意の正規パターンに対して,題意が成り立つと仮定する.
%$p$を$S_{2}(p) \subseteq L(Q)$を満たす$n+1$個の変数記号を含む非隣接変数正規パターンとする.
Let $p$ be a $\NAV$ regular pattern containing $n+1$ variable symbols such that $S_{2}(\{p\}) \subseteq L(Q)$ holds and $p$ contains a variable symbol $x$.
%$p \not \preceq q_{i}$ ($i=1, 2$)と仮定する.
%Assume that $p \not \preceq q_{1}$ and $p \not \preceq q_{2}$.
%非隣接変数正規パターン$p$を$p=p_{1}xp_{2}$, $Q=\{ q_{1}, \ldots , q_{k} \}$とおく.
%ここで,$p_{1}$は末尾が定数記号である非隣接変数正規パターンであり,$p_{2}$は先頭が定数記号である非隣接変数正規パターン,$x$は変数記号,任意の$i$ ($i=1, \ldots, k$)に対して,$q_{i}$は非隣接変数正規パターンである.
There exist two $\NAV$ regular patterns $p_{1},p_{2}$ such that $p=p_{1}xp_{2}$ holds.
%$a, b \in \Sigma$に対して,$p_{a}=p \{ x := a \}$,$p_{ab}=p \{ x := ab \}$とおく.
By the induction hypothesis, for any constant string $w\in \Sigma^{\ast}$ with $|w|=2$, $\{p\{x:=w\}\}\preceq Q$ holds because $p\{x:=w\}$ contains $n$ variable symbols.
%このとき,$p_{a}, p_{ab}$は$n$個の変数記号が含まれ,$S_{2}(p_{a}) \subseteq L(Q)$かつ$S_{2}(p_{ab}) \subseteq L(Q)$が成り立つことに注意する.
Hence, there exists a $\NAV$ regular pattern $q_{w} \in Q$ such that $p \{ x:=w \} \preceq q_{w}$ holds.
From Lemma \ref{追加補題1}, there exists a regular pattern $q \in Q$ such that $p \{ x:=xy \} \preceq q$ holds, where $y$ is a variable symbol that does not appear in $q$.
This contradicts the condition $Q \in \NAVRPkei$.
%帰納法の仮定より,任意の$a, b \in \Sigma$に対して,$p_{a} \preceq q_{i}$かつ$p_{ab} \preceq q_{i^{\prime}}$を満たすような$i, i^{\prime} \le k$が存在する.
%
%補題\ref{追加補題1}より,ある$i$に対して$p \{ x:=xy \} \preceq q_{i}$が成り立つ.
%このとき,$p \{ x:=xy \} =p_{1}xyp_{2}$の部分パターン$xy$は$q_{i}$の変数記号を置き換えることで生成できない.
%このことは,$q_{i}$に$xy$が含まれることを示している.
%これは,$q_{i}$が非隣接変数正規パターンであることに矛盾する.
Thus, we have that (i) implies (ii).
%以上より,(i) $\Rightarrow$ (ii)が成り立つ.
\end{proof}

\begin{col}
%$\sharp\Sigma \ge k+2$,$P \in \NAVRPplus$とする.
%このとき,$S_{2}(P)$は$\mathcal{RPL^{\mbox{$k$}}_{NAV}}$における$L(P)$の特徴集合である.
Let $\sharp\Sigma \ge k+2$, $P \in \NAVRPplus$.
Then, $S_{2}(P)$ is a characteristic set for $\mathcal{RPL^{\mbox{$k$}}_{NAV}}$.
\end{col}

\begin{lem}\label{Case_k+2}\label{k+2のとき}
%$\sharp\Sigma \le k+1$とする.このとき,$\NAVRPkei$は包含に関してコンパクト性を持たない.
Let $\sharp\Sigma \le k+1$.
Then, $\NAVRPkei$ does not have compactness with respect to containment.
\end{lem}
\begin{proof}
%$\Sigma = \{ a_{1}, \ldots , a_{k+1} \}$を$k+1$個の定数記号から成る集合,$p, q_{i}$を正規パターンとする.
Let $\Sigma$ be the set of $k+1$ constant symbols $a_{1}, \ldots , a_{k+1}$, i.e., $\Sigma = \{ a_{1}, \ldots , a_{k+1} \}$.
%$p \{ x := a_{i}y \} \preceq q_{i}$かつ$p \{ x := ya_{i+1} \} \preceq q_{i}~(i=1,2, \ldots ,k)$とする.
We assume that for $i=1,2,\ldots,k$, $p \{ x := a_{i}y \} \preceq q_{i}$ and $p \{ x := ya_{i+1} \} \preceq q_{i}~(i=1,2, \ldots ,k)$ hold.
%$p \{ x:= a_{k+1}a_{1} \} \preceq q_{1}$であるとき,$S_{2}(p) \backslash S_{1}(p) \subseteq \bigcup^{k}_{i=1} L(q_{i})$となる. 
If $p \{ x:= a_{k+1}a_{1} \} \preceq q_{1}$ holds, $S_{2}(p) \backslash S_{1}(p) \subseteq \bigcup^{k}_{i=1} L(q_{i})$ holds.
%すなわち,$L(p) \subseteq L(Q)$である.
This show that $L(p) \subseteq L(Q)$ holds.
%しかし,$p \not \preceq q_{i}$であるため,$L(p) \not \subseteq L(q_{i})~(i=1,2, \ldots k)$である.
However, for $i=1,2,\ldots,k$, since $p \not \preceq q_{i}$ holds, we have that $L(p) \not \subseteq L(q_{i})$ holds.
%したがって,$\NAVRPkei$は包含に関するコンパクト性を持たない.
Hence, $\NAVRPkei$ does not have compactness with respect to containment.
\end{proof}

%コンパクト性をもたない例を例\ref{反例k+1}に示す.
Next, we give an example for Lemma \label{Case_k+2} in Example \ref{Case_k+1}.
%An example of a lack of compactness is shown in Example \ref{counterexample k+1}.

\begin{ex}\label{Case_k+1}\label{反例k+1}
%$\Sigma= \{a_{1}, a_{2}, a_{3}, a_{4} \}$を$4$つの定数記号から成る集合,$p,q_{1},q_{2},q_{3}$を正規パターン,$x,x^{\prime},x^{\prime\prime}$を変数記号とする.
Let $\Sigma$ be the set of four constant symbols $a,~b,~c,~d$, i.e., $\Sigma= \{a, b, c, d \}$ and $x,x^{\prime},x^{\prime\prime}$ three distinct variable symbols.
Let $p,q_{1},q_{2},q_{3}$ be the $\NAV$ regular patterns given in Fig. \ref{Fig:CounterExampleforNAVR}. 
%$p,q_{1},q_{2},q_{3}$を以下のように定義する.
\begin{figure*}[tb]
\begin{align*}
p & = x^{\prime}cadadaadacbadadaadaxadadaadacbadadaadabx^{\prime\prime},\\
q_{1} & = x^{\prime}cadadaadacbadadaadabx^{\prime\prime},\\
q_{2} & = x^{\prime}badadaadacx^{\prime\prime},\\
q_{3} & = x^{\prime}aadadx^{\prime\prime}.
\end{align*}
\caption{$\NAV$ regular patterns $p$, $q_{1}$, $q_{2}$, and $q_{3}$}\label{Fig:CounterExampleforNAVR}
\end{figure*}
\noindent
Then, we have $L(p) \subseteq L(q_{1}) \cup L(q_{2}) \cup L(q_{3})$.
Since $p \not \preceq q_{1},~p \not \preceq q_{2}$ and $p \not \preceq q_{3}$ hold,

%これは,$L(p) \subseteq L(q_{1}) \cup L(q_{2}) \cup L(q_{3})$となる.
%しかし,$p \not \preceq q_{1},~p \not \preceq q_{2}$かつ$p \not \preceq q_{3}$である.
%However, since $p \not \preceq q_{1},~p \not \preceq q_{2}$ and $p \not \preceq q_{3}$ hold,

\end{ex}

%定理\ref{非隣接kが4以上}と補題\ref{k+2のとき}より,次の定理が成り立つ.
From Theorem \ref{KeyTheoforNAVRP} and Lemma \ref{Case_k+2}, we have the following theorem.

\begin{thm}\label{MainTheforNAVRP}
%$\sharp\Sigma \ge k+2$とする.
%このとき,$\RPat^{k}$は包含に関してコンパクト性を持つ.
Let $\sharp\Sigma \ge k+2$.
Then, the set $\RPat^{k}$ has compactness with respect to containment.
\end{thm}