\subsection{$D = \{ a_{1}b_{1}, a_{2}b_{2}, a_{3}y\}$ and $D = \{ a_{1}b_{1}, a_{2}b_{2}, yb_{3}\}$}\label{subsec:d3b}

{\color{black} In this subsection, for 
$D = \{ a_{1}b_{1}, a_{2}b_{2}, a_{3}y\}\subsetneq \RPat_{\Sigma\cup X}$ or $D = \{ a_{1}b_{1}, a_{2}b_{2}, yb_{3}\} \subsetneq \RPat_{\Sigma\cup X}$,
we consider a regular pattern $q$ in $\RPat_{\Sigma\cup X}$ 
such that $q$ minimally supports $D$ for a regular pattern $p$. 
Obviously, we remark that $D = \{ a_{1}b_{1}, a_{2}b_{2}, a_{3}y\}$ and $D = \{ a_{1}b_{1}, a_{2}b_{2}, yb_{3}\}$ {\color{black} satisfy} that $a_{i}\neq a_{3}$ and $b_{i}\neq b_{3}$ for $i=1,2$, respectively.
}

\begin{lem}\label{lem:3consts_i}
 % Let $\Sigma$ be an alphabet with $\sharp\Sigma \ge 3$, 
 {\color{black}
 Let $\Sigma$ be an alphabet with $\sharp\Sigma \ge 3$.
  Let $p$ and $q$ be regular patterns in $\RPat_{\Sigma\cup X}$.
  Define $D=\{ a_{1}b_{1}, a_{2}b_{2}, a_{3}y\}\subsetneq \RPat_{\Sigma\cup X}$
  where $a_{i}\neq a_{j}$ for $i,j~(1\leq i,j\leq 3, i\neq j)$, $b_{1}\neq b_{2}$
   and $y$ is a variable symbol in $X$ that occurs in neither $p$ nor $q$.
  Suppose that $p\preceq q$ or that $q$ minimally supports $D$ for $p$.
Then $p \{ x := xy \} \preceq q$.}
\end{lem}

\begin{proof}
{\color{black} It suffices to consider the case $p\not\preceq q$, since the case $p\preceq q$ is trivial.}
We assume that $p \{ x := xy \} \not\preceq q$. 
{\color{black} Since $q$ minimally supports $D$ for $p$, i.e., $p\{x:=r\}\preceq q$ for all $r\in D$, }
from the same argument as in the proof of Lemma~\ref{lem:oneside_i}, it is sufficient to consider the following five cases (\ref{lem:3consts_i}-1)--(\ref{lem:3consts_i}-5) of $q$: For $y_{1} \in X$,

\begin{tabular}{ll}
(\ref{lem:3consts_i}-1) & $q=q_{1}a_{1}b_{1}wa_{2}b_{2}w^{\prime}a_{3}y_{1}q_{2}$,\\
(\ref{lem:3consts_i}-2) & $q=q_{1}a_{1}b_{1}b_{2}y_{1}q_{2}$ ($a_{2}=b_{1}$ and $a_{3}=b_{2}$),\\
(\ref{lem:3consts_i}-3) & $q=q_{1}a_{1}b_{1}b_{2}wa_{3}y_{1}q_{2}$ ($b_{1}=a_{2}$),\\
(\ref{lem:3consts_i}-4) & $q=q_{1}a_{3}y_{1}wa_{1}b_{1}b_{2}q_{2}$ ($b_{1}=a_{2}$),\\
(\ref{lem:3consts_i}-5) & $q=q_{1}a_{1}b_{1}y_{1}wa_{2}b_{2}q_{2}$ ($b_{1}=a_{3}$),
\end{tabular}

\noindent
where no variable symbol occurs in both $w$ and $w'$.

\smallskip
\noindent
(\ref{lem:3consts_i}-1) Case of $q=q_{1}a_{1}b_{1}wa_{2}b_{2}w^{\prime}a_{3}y_{1}q_{2}$:
The following conditions must be satisfied: For $y_{1}^{\prime}\in X$,
\begin{align*}
\textrm{(1)}~& p_{1} \preceq q_{1}, & \textrm{(1')}~& p_{2} \preceq wa_{2}b_{2}w^{\prime}a_{3}y_{1}q_{2}, \\
\textrm{(2)}~& p_{1} \preceq q_{1}a_{1}b_{1}w, & \textrm{(2')}~& p_{2} \preceq w^{\prime}a_{3}y_{1}q_{2}, \\
\textrm{(3)}~& p_{1} \preceq q_{1}a_{1}b_{1}wa_{2}b_{2}w^{\prime}, & \textrm{(3')}~& p_{2} \preceq q_{2} \mbox{~or~} p_{2} \preceq y_{1}^{\prime}q_{2}.
\end{align*}
\noindent
{\color{black} This leads to a contradiction, as demonstrated by the following inductive argument:}
\begin{itemize}
  {\color{black} \item $|w|=|w^{\prime}|$: From (1') and (2'), we have $a_2=a_3$.
  This contradicts that $a_2\neq a_3$.}
\item $|w|+1=|w^{\prime}|$: From (2) and (3), {\color{black}both }$a_{1}b_{1}wa_{2}b_{2}w^{\prime}$ and $a_{1}b_{1}w$ are suffixes of $p_{1}$.
Since there exists a constant symbol $w_{1}$ such that $w^{\prime}=w_{1}w$ and $b_{2}w_{1}w=a_{1}b_{1}w$,
then $b_{2}=a_{1}$.
Moreover, {\color{black}both }$wa_{2}b_{2}w^{\prime}a_{3}$ and $w^{\prime}a_{3}$ are prefixes of $p_{2}$ from (1') and (2').
Since there exists a constant symbol $w_{2}$ such that $w^{\prime}=ww_{2}$ and $wa_{2}b_{2}=ww_{2}a_{3}$,
then $b_{2}=a_{3}$.
Thus, $a_{1} = a_{3}$.
This contradicts the assumption of $a_{1} \ne a_{3}$.
%
\item $|w|+1 < |w^{\prime}|$: From (2) and (3), {\color{black}both }$a_{1}b_{1}wa_{2}b_{2}w^{\prime}$ and $a_{1}b_{1}w$ are suffixes of $p_{1}$.
Hence, {\color{black}$a_{1}b_{1}w$ is the suffix of $w^{\prime}$}.
Moreover, {\color{black}both }$wa_{2}b_{2}w^{\prime}a_{3}$ and $w^{\prime}a_{3}$ are prefixes of $p_{2}$ from (1') and (2').
Hence, there exist constant {\color{black}strings} $w_{1}$ and $w_{2}$ such that $w^{\prime}=w_{1}w$, $w^{\prime}=ww_{2}$ and $|a_{2}b_{2}w_{1}|=|w_{2}a_{3}|+1$.
Thus, since the second-to-last symbol of $w_{1}$ is $a_{3}$, the equation $a_{1}=a_{3}$ holds.
This contradicts the assumption of $a_{1} \ne a_{3}$.
%
\item $|w|=|w^{\prime}|+1$: From (1') and (2'), {\color{black}both }$wa_{2}b_{2}w^{\prime}a_{3}$ and $w^{\prime}a_{3}$ are prefixes of $p_{2}$.
Since there exists a constant symbol $w_{1}$ such that $w=w^{\prime}w_{1}$ and $w^{\prime}w_{1}=w^{\prime}a_{3}$, then $w_{1}=a_{3}$.
Moreover, since {\color{black}both }$a_{1}b_{1}wa_{2}b_{2}w^{\prime}$ and $a_{1}b_{1}w$ are suffixes of $p_{1}$ from (2) and (3), 
there exists a constant symbol $w_{2}$ such that $w=w_{2}w^{\prime}$ and {\color{black}$w_{1}a_{2}b_{2}w^{\prime}=a_{1}b_{1}w_{2}w^{\prime}$}.
Hence, $w_{1}=a_{1}$.
Thus, $a_{1}=a_{3}$.
This contradicts the assumption of $a_{1}\ne a_{3}$.
%
\item $|w| > |w^{\prime}|+1$: Since {\color{black}both }$wa_{2}b_{2}w^{\prime}a_{3}$ and $w^{\prime}a_{3}$ are prefixes of $p_{2}$ from (1') and (2'),
there exists a constant string $w_{1}$ such that $w=w^{\prime}w_{1}$ and the first symbol of $w_{1}$ is $a_{3}$.
Moreover, since there exists a constant string $w_{2}$ such that $w=w_{2}w^{\prime}$ and {\color{black}$w_{1}a_{2}b_{2}=a_{1}b_{1}w_{2}$ from (2) and (3)},
$a_{1}b_{1}$ is a prefix of $w_{1}$.
Thus, $a_{3}=a_{1}$.
This contradicts the assumption of $a_{1} \ne a_{3}$.
\end{itemize}

\smallskip

\noindent
(\ref{lem:3consts_i}-2) Case of $q=q_{1}a_{1}b_{1}b_{2}y_{1}q_{2}$ ($a_{2}=b_{1}$ and $a_{3}=b_{2}$):
The following conditions must be satisfied: For $y_{1}^{\prime} \in X$,
\begin{align*}
\textrm{(1)}~& p_{1} \preceq q_{1}, & \textrm{(1')}~& p_{2} \preceq b_{2}y_{1}q_{2}, \\
\textrm{(2)}~& p_{1} \preceq q_{1}a_{1}, & \textrm{(2')}~& p_{2} \preceq y_{1}q_{2}, \\
\textrm{(3)}~& p_{1} \preceq q_{1}a_{1}b_{1}, & \textrm{(3')}~& p_{2} \preceq q_{2} \mbox{~or~} p_{2} \preceq y_{1}^{\prime}q_{2}.
\end{align*}

From (2) and (3), {\color{black}both }$a_{1}b_{1}$ and $a_{1}$ are suffixes of $p_{1}$.
Hence, $b_{1}=a_{1}$.
Thus, from the assumption of $b_{1}=a_{2}$, the equation $a_{1}=a_{2}$ holds.
This contradicts the assumption of $a_{1} \ne a_{2}$.
\smallskip

\noindent
(\ref{lem:3consts_i}-3) Case of $q=q_{1}a_{1}b_{1}b_{2}wa_{3}y_{1}q_{2}$ ($b_{1}=a_{2}$):
The following conditions must be satisfied: For $y_{1}^{\prime} \in X$,
\begin{align*}
\textrm{(1)}~& p_{1} \preceq q_{1}, & \textrm{(1')}~& p_{2} \preceq b_{2}wa_{3}y_{1}q_{2}, \\
\textrm{(2)}~& p_{1} \preceq q_{1}a_{1}, & \textrm{(2')}~& p_{2} \preceq wa_{3}y_{1}q_{2}, \\
\textrm{(3)}~& p_{1} \preceq q_{1}a_{1}b_{1}b_{2}w, & \textrm{(3')}~& p_{2} \preceq q_{2} \mbox{~or~} p_{2} \preceq y_{1}^{\prime}q_{2}.
\end{align*}
\noindent
{\color{black} This leads to a contradiction, as demonstrated by the following inductive argument:}
\begin{itemize}
\item $|w|=0$: From (2) and (3), {\color{black}both }$a_{1}$ and $a_{1}b_{1}b_{2}$ are suffixes of $p_{1}$.
Hence, $a_{1}=b_{2}$.
Moreover, since {\color{black}both }$b_{2}a_{3}$ and $a_{3}$ is prefixes of $p_{2}$, $b_{2}=a_{3}$.
Thus, $a_{1}=a_{3}$.
This contradicts the assumption of $a_{1} \ne a_{3}$.
%
\item $|w| \ge 1$: Since {\color{black}both }$a_{1}$ and $a_{1}b_{1}b_{2}w$ are suffixes of $p_{1}$ from (2) and (3),
the last symbol of $w$ is $a_{1}$.
Moreover, since {\color{black}both }$b_{2}wa_{3}$ and $wa_{3}$ are prefixes of $p_{2}$ from (1') and (2'),
the last symbol of $w$ is $a_{3}$.
Thus, $a_{1}=a_{3}$.
This contradicts the assumption of $a_{1} \ne a_{3}$.
\end{itemize}

\smallskip

\noindent
(\ref{lem:3consts_i}-4) Case of $q=q_{1}a_{3}y_{1}wa_{1}b_{1}b_{2}q_{2}$ ($b_{1}=a_{2}$):
The following conditions must be satisfied: For $y_{1}^{\prime} \in X$,
\begin{align*}
\textrm{(1)}~& p_{1} \preceq q_{1}, & \textrm{(1')}~& p_{2} \preceq wa_{1}b_{1}b_{2}q_{2} \mbox{ or } \\
& & & p_{2} \preceq y_{1}^{\prime}wa_{1}b_{1}b_{2}q_{2},\\
\textrm{(2)}~& p_{1} \preceq q_{1}a_{3}y_{1}w, & \textrm{(2')}~& p_{2} \preceq b_{2}q_{2}, \\
\textrm{(3)}~& p_{1} \preceq q_{1}a_{3}y_{1}wa_{1}, & \textrm{(3')}~& p_{2} \preceq q_{2}.
\end{align*}

From (3), there exist regular patterns $p_{1}^{\prime}$ and $p_{1}^{\prime\prime}$ such that $p_{1}=p_{1}^{\prime}p_{1}^{\prime\prime}$, $p_{1}^{\prime} \preceq q_{1}a_{3}$, and $p_{1}^{\prime\prime} \preceq y_{1}wa_{1}$.
Hence, if $p_{2} \preceq wa_{1}b_{1}b_{2}q_{2}$ of (1') holds, since $p=p_{1}xp_{2}=p_{1}^{\prime}p_{1}^{\prime\prime}xp_{2}\preceq q_{1}a_{3}p_{1}^{\prime\prime}xwa_{1}b_{1}b_{2}q_{2}=q \{ y_{1} := p_{1}^{\prime\prime}x \}$, then $p \preceq q$.
{\color{black}
Thus, this contradicts the assumption that $p$ and $q$ satisfy 
  $p\not\preceq q$.}
Similarly, $p_{2} \preceq y_{1}^{\prime}wa_{1}b_{1}b_{2}q_{2}$ of (1') leads to a contradiction.

\smallskip

\noindent
(\ref{lem:3consts_i}-5) Case of $q=q_{1}a_{1}b_{1}y_{1}wa_{2}b_{2}q_{2}$ ($b_{1}=a_{3}$):
The following conditions must be satisfied: For $y_{1}^{\prime} \in X$,
\begin{align*}
\textrm{(1)}~& p_{1} \preceq q_{1}, & \textrm{(1')}~& p_{2} \preceq y_{1}wa_{2}b_{2}q_{2}, \\
\textrm{(2)}~& p_{1} \preceq q_{1}a_{1}, & \textrm{(2')}~& p_{2} \preceq wa_{2}b_{2}q_{2} \mbox{ or } \\
& & & p_{2} \preceq y_{1}^{\prime}wa_{2}b_{2}q_{2},\\
\textrm{(3)}~& p_{1} \preceq q_{1}a_{1}b_{1}y_{1}w, & \textrm{(3')}~& p_{2} \preceq q_{2}.
\end{align*}
\noindent
Let $q_{1}^{\prime}=q_{1}a_{1}b_{1}$, $q_{2}^{\prime}=y_{1}w$, $q_{3}^{\prime}=a_{2}b_{2}q_{2}$.
From (3), $p_{1} \preceq q_{1}^{\prime}q_{2}^{\prime}$, and from (1'), $p_{2} \preceq q_{2}^{\prime}q_{3}^{\prime}$.
{\color{black}From Theorem \ref{Sato1:Lemma9}, we have $p\preceq q$, since $q_{2}^{\prime}$ contains a variable symbol $y_{1}$.}
{\color{black}
This contradicts the assumption that $p$ and $q$ satisfy 
  $p\not\preceq q$.}
\end{proof}

\begin{lem}\label{lem:3consts_ii}
  %Let $\Sigma$ be an alphabet $\sharp\Sigma \ge 3$, 
  {\color{black}Let $\Sigma$ be an alphabet with $\sharp\Sigma \ge 3$}.
  {\color{black}Let $p$ and $q$} be regular patterns {\color{black} in $\RPat_{\Sigma\cup X}$.
  Define $D=\{ a_{1}b_{1}, a_{2}b_{2}, yb_{3}\} \subsetneq \RPat_{\Sigma\cup X}$
  where $a_{1}\neq a_{2}$, $b_{i}\neq b_{j}$ for {\color{black}$1\leq i<j\leq 3$} with $i\neq j$ and $y$ is a variable symbol in $X$ that {\color{black}occurs in neither $p$ nor $q$}.
  Suppose that $p\preceq q$ or that $q$ minimally supports $D$ for $p$.
Then $p \{ x := xy \} \preceq q$.}
\end{lem}

\begin{proof}
The proof follows by reversing $p$ and $q$ and subsequently applying Lemma~\ref{lem:3consts_i}.
\end{proof}

\subsection{$D = \{ a_{1}b_{1}, a_{2}b_{2}, a_{3}b_{3}\}$}\label{subsec:d3c}

{\color{black}
In this subsection, for $D = \{ a_{1}b_{1}, a_{2}b_{2}, a_{3}b_{3}\} \subsetneq \RPat_{\Sigma\cup X}$, we consider a regular pattern $q$ in $\RPat_{\Sigma\cup X}$ 
such that $q$ minimally supports $D$ for a regular pattern $p$ in $\RPat_{\Sigma\cup X}$
under some conditions with the symbols $a_i,~b_i \in \Sigma$ for $i~(1\leq i \leq k)$.
}

\begin{lem}\label{lem:3consts_iii}
 % Let $\Sigma$ be an alphabet with $\sharp\Sigma \ge 3$, 
 {\color{black}Let $\Sigma$ be an alphabet with $\sharp\Sigma \ge 3$}.
 {\color{black}Let $p$ and $q$ be} regular {\color{black}patterns in $\RPat_{\Sigma\cup X}$}
 such that a variable symbol $y\in X$ does not occur in $p$.
  {\color{black} Define $D=\{ a_{1}b_{1}, a_{2}b_{2}, a_{3}b_{3}\}\subsetneq \RPat_{\Sigma\cup X}$ where $a_{i} \ne a_{j} \mbox{ and } b_{i} \ne b_{j} \mbox{ with } i\ne j ~{\color{black}(1\le i<j\le 3)}$.
  Suppose that $p\preceq q$ or that $q$ minimally supports $D$ for $p$.
  Then $p \{ x := xy \} \preceq q$.}
\end{lem}

\begin{proof}
{\color{black} It suffices to consider the case $p\not\preceq q$, since the case $p\preceq q$ is trivial.}
We assume that $p \{ x := xy \} \not\preceq q$.
Since $q$ minimally supports $D$ for $p$,
it is sufficient to consider the following four cases (\ref{lem:3consts_iii}-1)-(\ref{lem:3consts_iii}-4) of $q$ for some regular patterns $q_{1},q_{2}$ and some constant strings $w,w^{\prime}$ ($|w|\geq 0$ and $|w^{\prime}|\geq 0$):

\smallskip

\noindent
\begin{tabular}{ll}
(\ref{lem:3consts_iii}-1) & $q=q_{1}a_{1}b_{1}wa_{2}b_{2}w^{\prime}a_{3}b_{3}q_{2}$,\\
(\ref{lem:3consts_iii}-2) & $q=q_{1}a_{1}b_{1}a_{3}b_{3}q_{2}$ ($b_{1}=a_{2}$ and $a_{3}=b_{2}$),\\
(\ref{lem:3consts_iii}-3) & $q=q_{1}a_{1}b_{1}b_{2}wa_{3}b_{3}q_{2}$ ($b_{1}=a_{2}$),\\
(\ref{lem:3consts_iii}-4) & $q=q_{1}a_{1}b_{1}wa_{2}b_{2}b_{3}q_{2}$ ($b_{2}=a_{3}$).
\end{tabular}

\smallskip

\noindent
(\ref{lem:3consts_iii}-1) Case of $q=q_{1}a_{1}b_{1}wa_{2}b_{2}w^{\prime}a_{3}b_{3}q_{2}$:
The following conditions must be satisfied:
\begin{align*}
\textrm{(1)}~& p_{1} \preceq q_{1}, & \textrm{(1')}~& p_{2} \preceq wa_{2}b_{2}w^{\prime}a_{3}b_{3}q_{2}, \\
\textrm{(2)}~& p_{1} \preceq q_{1}a_{1}b_{1}w, & \textrm{(2')}~& p_{2} \preceq w^{\prime}a_{3}b_{3}q_{2}, \\
\textrm{(3)}~& p_{1} \preceq q_{1}a_{1}b_{1}wa_{2}b_{2}w^{\prime}, & \textrm{(3')}~& p_{2} \preceq q_{2}.
\end{align*}
\noindent
{\color{black} This leads to a contradiction, as demonstrated by the following inductive argument:}
\begin{itemize}
\item $|w|=|w^{\prime}|$: From (2) and (3), {\color{black}both }$a_{1}b_{1}wa_{2}b_{2}w^{\prime}$ and $a_{1}b_{1}w$ are suffixes of $p_{1}$.
Then, $a_{1}b_{1}w=a_{2}b_{2}w^{\prime}$.
Hence, $a_{1}b_{1}=a_{2}b_{2}$.
This contracts the assumption of $a_{1} \ne a_{2}$ and $b_{1} \ne b_{2}$.
%
\item $|w|+1=|w^{\prime}|$: {\color{black} From (1') and (2')}, $wa_{2}b_{2}w^{\prime}a_{3}b_{3}$ and $w^{\prime}a_{3}b_{3}$ are prefixes of $p_{2}$.
If there exists a constant symbol $w_{1}$ such that $w^{\prime}a_{3}b_{3}=ww_{1}a_{3}b_{3}$,
then {\color{black}$b_{2}=a_{3}$} from $wa_{2}b_{2}=ww_{1}a_{3}$.
{\color{black}From} (2) and (3), {\color{black}both } $a_{1}b_{1}wa_{2}b_{2}w^{\prime}$ and $a_{1}b_{1}w$ are suffixes of $p_{1}$.
Then, there exists a constant symbol $w_{2}$ such that $w^{\prime}=w_{2}w$,
then we have $b_{2}=a_{1}$ from $b_{2}w_{2}w=a_{1}b_{1}w$.
Hence, from $b_{2}=a_{3}$, {\color{black} we have $a_{3}=a_{1}$}.
This contradicts the assumption of $a_{3} \ne a_{1}$.
%
\item $|w|+1 < |w^{\prime}|$: From (2) and (3), 
{\color{black}both }$a_{1}b_{1}wa_{2}b_{2}w^{\prime}$ and $a_{1}b_{1}w$ are suffixes of $p_{1}$.
If there exists a constant string $w_{1}$ ($|w_{1}|\geq 2$) such that $w^{\prime}=w_{1}w$, then $a_{1}b_{1}$ is a suffix of $w_{1}$.
From  conditions (1') and (2'), 
{\color{black}both }$wa_{2}b_{2}w^{\prime}a_{3}b_{3}$ and $w^{\prime}a_{3}b_{3}$ are prefixes of $p_{2}$.
If there exist constant strings $w_{1}$ and $w_{2}$ such that $w^{\prime} = w_{1}w=ww_{2}$, then {\color{black} $a_{3}b_{3}$ is a suffix of $w_{1}$ from $|w_1|=|w_2|$ and $ww_{2}a_{3}b_{3}=wa_{2}b_{2}w_{1}$}.
Hence, $a_{1}b_{1}=a_{3}b_{3}$.
This contradicts the assumption of $a_{1} \ne a_{3}$ and $b_{1} \ne b_{3}$.
%
\item $|w|>|w^{\prime}|$: We can prove the contradiction in a similar way as $|w|\le|w^{\prime}|$.
\end{itemize}

\smallskip

\noindent
(\ref{lem:3consts_iii}-2) Case of $q=q_{1}a_{1}b_{1}a_{3}b_{3}q_{2}$ ($b_{1}=a_{2}$ and $a_{3}=b_{2}$):
The following conditions must be satisfied:
\begin{align*}
\textrm{(1)}~& p_{1} \preceq q_{1}, & \textrm{(1')}~& p_{2} \preceq a_{3}b_{3}q_{2}, \\
\textrm{(2)}~& p_{1} \preceq q_{1}a_{1}, & \textrm{(2')}~& p_{2} \preceq b_{3}q_{2}, \\
\textrm{(3)}~& p_{1} \preceq q_{1}a_{1}b_{1}, & \textrm{(3')}~& p_{2} \preceq q_{2}.
\end{align*}

From (2) and (3), since {\color{black}both }$a_{1}b_{1}$ and $a_{1}$ are suffixes of $p_{1}$, 
the equation $b_{1} = a_{1}$ holds.
From the assumption of $b_{1}=a_{2}$, the equation $a_{1}=a_{2}$ holds.
This contradicts the assumption of $a_{1}\not= a_{2}$.
\smallskip

\noindent
(\ref{lem:3consts_iii}-3) Case of $q=q_{1}a_{1}b_{1}b_{2}wa_{3}b_{3}q_{2}$ ($b_{1}=a_{2}$):
The following conditions must be satisfied:
\begin{align*}
\textrm{(1)}~& p_{1} \preceq q_{1}, & \textrm{(1')}~& p_{2} \preceq b_{2}wa_{3}b_{3}q_{2}, \\
\textrm{(2)}~& p_{1} \preceq q_{1}a_{1}, & \textrm{(2')}~& p_{2} \preceq wa_{3}b_{3}q_{2}, \\
\textrm{(3)}~& p_{1} \preceq q_{1}a_{1}b_{1}b_{2}w, & \textrm{(3')}~& p_{2} \preceq q_{2}.
\end{align*}
\noindent
{\color{black} This leads to a contradiction, as demonstrated by the following inductive argument:}
\begin{itemize}
\item $|w|=0$: From (2) and (3), {\color{black}both }$a_{1}$ and $a_{1}b_{1}b_{2}$ are suffixes of $p_{1}$. Moreover, from (1') and (2'), {\color{black}both }$b_{2}a_{3}b_{3}$ and $a_{3}b_{3}$ are prefixes of $p_{2}$.
Since $b_{2}=a_{1}$ and {\color{black}$b_{2}=a_{3}$, we have $a_{1}=a_{3}$}.
This contradicts the assumption of $a_{1}\not= a_{3}$.
%
\item $|w| \ge 1$: From (2) and (3), {\color{black}both }$a_{1}$ and $a_{1}b_{1}b_{2}w$ are suffixes of $p_{1}$.
Hence, the last symbol of $w$ is $a_{1}$.
Moreover, {\color{black}both }$b_{2}wa_{3}b_{3}$ and $wa_{3}b_{3}$ are prefixes of $p_{2}$ from (1') and (2').
Hence, the last symbol of $w$ is $a_{3}$.
Therefore, $a_{1}=a_{3}$.
This contradicts the assumption of $a_{1} \ne a_{3}$.
\end{itemize}

\smallskip

\noindent
(\ref{lem:3consts_iii}-4) Case of $q=q_{1}a_{1}b_{1}wa_{2}b_{2}b_{3}q_{2}$ ($b_{2}=a_{3}$):
The following conditions must be satisfied:
\begin{align*}
\textrm{(1)}~& p_{1} \preceq q_{1}, & \textrm{(1')}~& p_{2} \preceq wa_{2}b_{2}b_{3}q_{2}, \\
\textrm{(2)}~& p_{1} \preceq q_{1}a_{1}b_{1}w, & \textrm{(2')}~& p_{2} \preceq b_{3}q_{2}, \\
\textrm{(3)}~& p_{1} \preceq q_{1}a_{1}b_{1}wa_{2}, & \textrm{(3')}~& p_{2} \preceq q_{2}.
\end{align*}
\noindent
{\color{black} This leads to a contradiction, as demonstrated by the following inductive argument:}
\begin{itemize}
\item $|w|=0$: From (2) and (3), {\color{black}both }$a_{1}b_{1}$ and $a_{1}b_{1}a_{2}$ are suffixes of $p_{1}$. And from (1') and (2'), {\color{black}both }$a_{2}b_{2}b_{3}$ and $b_{3}$ are prefixes of $p_{2}$.
Since $b_{1}=a_{2}$ and $a_{2}=b_{3}$, then $b_{1}=b_{3}$.
This contradicts the assumption of $b_{1}\not= b_{3}$.
%
\item $|w| \ge 1$: Since {\color{black}both }$a_{1}b_{1}w$ and $a_{1}b_{1}wa_{2}$ are suffixes of $p_{1}$ from (2) and (3), the first symbol of $w$ is $b_{1}$.
Moreover, since {\color{black}both }$wa_{2}b_{2}b_{3}$ and $b_{3}$ are prefixes of $p_{2}$ from (1') and (2'),
the first symbol of $w$ is $b_{3}$.
Therefore, $b_{1}=b_{3}$.
This contradicts the assumption of $b_{1} \ne b_{3}$.
\end{itemize}
\vspace*{-1em}
\end{proof}

{\color{black}
The conditions in Lemmas~\ref{lem:3consts_i}, \ref{lem:3consts_ii}, and \ref{lem:3consts_iii} are illustrated in the cases (9), (10), and (11) in Fig.~\ref{fig:lem7bigraph}.
}

\begin{figure*}[t]
  \begin{center}
    \includegraphics[scale=0.8]{figs/lem7bigraph.png}
    \caption{Let $\Sigma=\{a,b,c,d,e,f,g\}$ and $p,q \in \RPat$. We assume that the symbols in $\Sigma$ are mutually distinct.
    {\color{black} The figures (9), (10) and (11) express cases of $D$s in Lemmas~\ref{lem:3consts_i}, \ref{lem:3consts_ii}, and \ref{lem:3consts_iii}, respectively.}
    {\color{black} In these cases, if $q$ minimally supports $D$ for $p$, }
    then $p \{ x := xy \} \preceq q$.}\label{fig:lem7bigraph}
  \end{center}
\end{figure*}

