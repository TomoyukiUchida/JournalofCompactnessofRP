


{\color{black}In this paper, this study revisits and corrects the compactness theorem for regular pattern languages originally proposed by Sato et al.~\cite{Sato1} by identifying an error in their proof and providing a revised argument under additional conditions. Furthermore, we establish that for the subclass of non-adjacent regular patterns, finite unions can be efficiently learned under weaker constraints on constant symbols than those required in the general case. 
For an integer $k~(k\ge 2)$, we have shown the conditions on the number of constant symbols in $\Sigma$, summarized in Table \ref{table:results}, required for the classes $\RPatkei$ of all the set of $k$ regular pattern languages and $\NAVRPkei$ of all the set of $k$ non-adjacent variable regular patterns in $\NAVRP$ to have compactness with respect to {\color{black} language} containment.
These results refine the theoretical understanding of compactness and learnability in pattern languages and offer a solid foundation for future research in Computational Learning Theory.

The results in this paper} leads to design an efficient learning algorithm for finite unions of languages of non-adjacent variable regular patterns in $\NAVRP$, based on the learning algorithm for $\RPatkei$ proposed by Arimura et al.~\cite{Arimura1994}.
\begin{table}[t]
\caption{The conditions on the number $\sharp \Sigma$ of constant symbols in $\Sigma$ required for compactness with respect to {\color{black} language} containment.}\label{table:results}
\begin{center}
\begin{tabular}{c|c|c}
  Class & $k=2$ & $k\ge 3$\\
  \hline
  \raisebox{-5pt}{$\RPatkei$} & \raisebox{-5pt}{$\sharp \Sigma \ge 4$} & \raisebox{-5pt}{$\sharp \Sigma \ge 2k-1$} \\[10pt]
  \hline
  \raisebox{-5pt}{$\NAVRPkei$} & \multicolumn{2}{c}{\raisebox{-5pt}{$\sharp \Sigma \ge k+2$}}\\[10pt]
\end{tabular}
\end{center}
\end{table}

Extending the notion of strong compactness, as introduced by Arimura et al.~\cite{Arimura1996}, to finite unions of regular pattern languages with non-adjacent variables remains as a topic for future research.
Furthermore, based on the characteristic set for $\NAVRPkei$, we plan to propose a polynomial-time inductive inference algorithm that identifies finite unions of regular pattern languages with non-adjacent variables in the limit from positive examples.
Ishinada et al. \cite{Ishinada2023} investigated a query learning model that employs high-precision Graph Convolution Networks (GCNs) as oracles for tree patterns.
Applying the findings of the present study to tree pattern languages, with the aim of enabling the extension of their work to finite unions of tree pattern languages, remains an important direction for future research.