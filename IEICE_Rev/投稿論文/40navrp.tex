

A regular pattern $p$ is said to be a {\it non-adjacent variable regular pattern} ($\NAV$ regular pattern)  
if $p$ does not contain consecutive variable symbols.
For example, the regular pattern $p=axybc$ is not an $\NAV$ regular pattern because $xy$ is appeared in $p$.
Let {\color{red}$\NAVRP_{\Sigma\cup X}$} be the set of all $\NAV$ regular patterns {\color{red} over $\Sigma\cup X$}.
Let {\color{red}$\NAVRPplus_{\Sigma\cup X}$} be the set of all finite subsets $S$ of {\color{red}$\NAVRP_{\Sigma\cup X}$} such that $S$ is not the empty set, i.e., {\color{red}$\NAVRPplus_{\Sigma\cup X}=\{S \subseteq \NAVRP_{\Sigma\cup X} \mid \sharp S {\color{red} \geq} 1\}$}.
{\color{red}For an integer $k$ with $k\geq 1$, let $\NAVRPkei_{\Sigma\cup X}$ be} the set of all subsets $P$ of {\color{red}$\NAVRPplus_{\Sigma\cup X}$} such that $P$ consists of at most $k$ $\NAV$ regular patterns {\color{red}over $\Sigma\cup X$}, i.e., {\color{red}$\NAVRPkei_{\Sigma\cup X}=\{P\in \NAVRPplus_{\Sigma\cup X} \mid \sharp P \leq k\}$.}
We define the compactness with respect to {\color{red} language} containment for {\color{red}$\NAVRPkei_{\Sigma\cup X}$ in the same manner as for} {\color{red}$\RPatkei_{\Sigma\cup X}$}.%Def.\ref{def:compactness}.
 For an integer $k$ with $k\geq 1$, any $\NAV$ regular pattern $p \in \NAVRP_{\Sigma\cup X}$ and any set $Q \in \NAVRPkei_{\Sigma\cup X}$,
  the set $\NAVRPkei_{\Sigma\cup X}$ {\color{red} is said} to have {\it compactness with respect to {\color{red}language} containment}
  if there exists an $\NAV$ regular pattern $q \in Q$ over $\Sigma\cup X$ such that $L(p) \subseteq L(q)$ if $L(p) \subseteq L(Q)$.
Then, the following {\color{red}theorem} holds.

\begin{thm}\label{KeyTheoforNAVRP}
%For an integer $k~(k\ge 2)$, let $\sharp \Sigma \ge k+2,~P\in \NAVRPplus,~Q \in \NAVRPkei$.  
{\color{red} Let $k$ be an integer with $k\ge 2$ and $\Sigma$ an alphabet with $\sharp \Sigma \ge k+2$.}
Then, {\color{red} for two sets \NAV regular patterns $P\in \NAVRPplus_{\Sigma\cup X}$ and $Q \in \NAVRPkei_{\Sigma\cup X}$,} the following (i), (ii) and (iii) are equivalent:
\[
\begin{tabular}{ll}
(i) $S_{2}(P) \subseteq L(Q)$,
(ii) $P \sqsubseteq Q$,
(iii) $L(P) \subseteq L(Q)$.
\end{tabular}
\]
\end{thm}

\begin{proof}
From the definitions of {\color{red}$\NAVRPplus_{\Sigma\cup X}$} and {\color{red}$\NAVRPkei_{\Sigma\cup X}$}, it is clear that (ii) implies (iii)  and  (iii) implies (i).
Hence, we will show that (i) implies (ii) 
by mathematical induction on the number $n$ of variable symbols that appear in an $\NAV$ regular pattern $p\in P$ as follows:
If $n=0$, then we have $S_{2}(\{p\})= \{ p \}$.
Hence, $p \in L(Q)$.
Therefore, there exists $q \in Q$ such that $p \preceq q$.

If $n \ge 0$, we assume that the proposition holds for any regular $\NAV$ regular pattern containing $n \ge 0$ variable symbols.
Let $p$ be an $\NAV$ regular pattern containing $n+1$ variable symbols such that $S_{2}(\{p\}) \subseteq L(Q)$ and $p$ contains a variable symbol $x$.
There exist two $\NAV$ regular patterns $p_{1},p_{2}$ such that $p=p_{1}xp_{2}$.
By the induction hypothesis, for any constant string $w\in \Sigma^{\ast}$ with $|w|=2$, {\color{red}$\{p\{x:=w\}\}\sqsubseteq Q$} because $p\{x:=w\}$ contains $n$ variable symbols.
Hence, there exists an $\NAV$ regular pattern $q_{w} \in Q$ such that $p \{ x:=w \} \preceq q_{w}$.
From Lemma \ref{Add-Lemma01}, 
there exists a regular pattern $q \in Q$ such that $p \{ x:=xy \} \preceq q$, where $y$ is a variable symbol that does not appear in $q$.
This contradicts the condition $Q \in \NAVRPkei$.
Thus, we have that (i) implies (ii).
\end{proof}

\begin{col}
%Let $k\ge 2$, $\sharp\Sigma \ge k+2$ and {\color{red}$P \in \NAVRPplus$}.
{\color{red}Let $k$ be an integer with $k\ge 2$ and $\Sigma$ an alphabet with $\sharp\Sigma \ge k+2$.}
Then, for {\color{red}$P \in \NAVRPplus_{\Sigma\cup X}$}, $S_{2}(P)$ is a characteristic set of $\NAVRPLkei$.
\end{col}

\begin{lem}\label{Case_k+2}
%Let $k\ge 2$ and $\sharp\Sigma \le k+1$.
{\color{red}Let $k$ be an integer with $k\ge 2$ and $\Sigma$ an alphabet with $\sharp\Sigma \le k+1$.}
Then, $\NAVRPkei_{\Sigma\cup X}$ does not have compactness with respect to {\color{red} language }containment.
\end{lem}
\begin{proof}
Let $\Sigma = \{ a_{1}, \ldots , a_{k+1} \}$.
We assume that for $i=1,2,\ldots,k$, $p \{ x := a_{i}y \} \preceq q_{i}$ and $p \{ x := ya_{i+1} \} \preceq q_{i}$. 
If $p \{ x:= a_{k+1}a_{1} \} \preceq q_{1}$, $S_{2}(p) \backslash S_{1}(p) \subseteq \bigcup^{k}_{i=1} L(q_{i})$.
This show that $L(p) \subseteq L(Q)$.
However, for $i=1,2,\ldots,k$, since $p \not \preceq q_{i}$, we have $L(p) \not \subseteq L(q_{i})$.
Hence, $\NAVRPkei$ does not have compactness with respect to {\color{red} language }containment.
\end{proof}

Next, in Example \ref{Case_k+1}, we give an example for Lemma \ref{Case_k+2}.
\begin{ex}\label{Case_k+1}
{\color{red}Let $\Sigma=\{a,b,c,d\}$.
Let $p,q_{1},q_{2},q_{3}$ be the following $\NAV$ regular patterns over $\Sigma\cup X$:% given in Fig. \ref{Fig:CounterExampleforNAVR}. 
%\begin{figure}[tb]

\noindent
%  \begin{tabular}{l}
$p  = x^{\prime}cadadaadacbadadaadaxadadaadacbadadaadabx^{\prime\prime}$,\\
$q_{1} = x^{\prime}cadadaadacbadadaadacx^{\prime\prime}$,\\
$q_{2} = x^{\prime}badadaadacx^{\prime\prime}$,\\
$q_{3} = x^{\prime}aadadx^{\prime\prime}$,\\
%  \end{tabular}
%\caption{$\NAV$ regular patterns $p$, $q_{1}$, $q_{2}$, and $q_{3}$}\label{Fig:CounterExampleforNAVR}
%\end{figure}
\noindent
where $x,x^{\prime},x^{\prime\prime}$ are three distinct variable symbols in $X$}.
Then, we have  $L(p) \subseteq L(q_{1}) \cup L(q_{2}) \cup L(q_{3})$.
This show that for $P=\{p\},~Q=\{q_{1},q_{2},q_{3}\}$, (iii) of Theorem \ref{KeyTheoforNAVRP} holds.
However, since $p \not \preceq q_{1},~p \not \preceq q_{2}$ and $p \not \preceq q_{3}$,
we have $P \not \sqsubseteq Q$, that is, (ii) of Theorem \ref{KeyTheoforNAVRP} does not hold.
\end{ex}

From Theorem \ref{KeyTheoforNAVRP} and Lemma \ref{Case_k+2}, 
we have the following theorem.

\begin{thm}\label{MainTheforNAVRP}
%Let $k\ge 2$ and $\sharp\Sigma \ge k+2$.
{\color{red}Let $k$ be an integer with $k\ge 2$ and $\Sigma$ an alphabet with $\sharp\Sigma \ge k+2$.}
Then, the set {\color{red}$\NAVRPLkei_{\Sigma\cup X}$} has compactness with respect to {\color{red} language }containment.
\end{thm}