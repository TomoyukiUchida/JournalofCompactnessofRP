\begin{summary}
%  {\color{red}
%A regular pattern is defined as a string composed of constant symbols and distinct variable symbols.
%The language $L(p)$ of a regular pattern $p$ is the set of all constant strings obtained by replacing each variable symbol in $p$ with a constant string.
%Let $\RPatkei$ denote the class of all sets containing at most $k~(k\geq 2)$ regular patterns.
%Sato et al. (Proc. ALT'98, 1998) demonstrated that the finite set $S_2(P)$,
%derived from a set $P\in \RPatkei$ by replacing variables with constant strings of length at most two, serves as a characteristic set for the language $L(P)=\bigcup_{p\in P}L(p)$.
%They also established that $\RPatkei$  exhibits compactness with respect to {\color{red} language} containment when the number of constant symbols is greater than or equal to $2k-1$.
%In this paper, we revisit their results and correct an error in the proof of their theorem by introducing additional conditions.
%Specifically, we show that any generalization of the strings in $S_{2}(P)$ would violate the condition $p\{x:=r\}\preceq q$ for all $r\in S_{2}(P)$
%where $p$ is a regular pattern in $P$ and $q$ is a regular pattern.
%Further, we investigate the set $\NAVRP$ consisting of at most $k~(k\ge 1)$ non-adjacent regular patterns, that is, regular patterns in which no two variable symbols appear consecutively.
%Further we show that for any $P\in \NAVRPkei$, the set $S_{2}(P)$ is a characteristic set of $L(P)$.
%Additionally, we proved that $\NAVRPkei$ possesses compactness with respect to {\color{red} language} containment if the number of constant symbols is greater than or equal to $k+2$.
%These results imply that it is possible to design an efficient learning algorithm for  finite unions of pattern languages generated by non-adjacent regular patterns, requiring fewer constant symbols than in the general case of regular patterns.
%}
%\begin{abstract}


%A \textit{regular pattern} is defined as a string composed of constant symbols and distinct variable symbols. 
%The language $L(p)$ of a regular pattern $p$ is the set of all constant strings obtained by replacing each variable symbol in $p$ with a constant string. 
%Let $\mathcal{RP}^k$ denote the class of all sets containing at most $k$ $(k \ge 2)$ regular patterns. 
%Sato et al.\ (Proc.\ ALT'98, 1998) demonstrated that the finite set $S_2(P)$, derived from a set $P \in \mathcal{RP}^k$ by replacing variables with constant strings of length at most two, serves as a characteristic set for the language $L(P) = \bigcup_{p \in P} L(p)$. 
%They also claimed that $\mathcal{RP}^k$ exhibits compactness with respect to language containment when the number of constant symbols is greater than or equal to $2k - 1$.
%In this paper, \textbf{we revisit their results and point out an error in the original proof} of their theorem. 
%We then \textbf{provide a new and correct proof} by introducing additional conditions that ensure the validity of their claim. 
%Furthermore, we investigate the class $\mathcal{RP}^k_{\mathrm{NAV}}$, consisting of at most $k$ $(k \ge 1)$ \textit{non-adjacent regular patterns}—that is, patterns in which no two variable symbols appear consecutively. 
%We show that for any $P \in \mathcal{RP}^k_{\mathrm{NAV}}$, the set $S_2(P)$ serves as a characteristic set of $L(P)$. 
%Additionally, we prove that $\mathcal{RP}^k_{\mathrm{NAV}}$ possesses compactness with respect to language containment if the number of constant symbols is greater than or equal to $k + 2$.
%These results imply that it is possible to design an efficient learning algorithm for finite unions of pattern languages generated by non-adjacent regular patterns, requiring fewer constant symbols than in the general case of regular patterns.
%\end{abstract}


%A \textit{regular pattern} is a string composed of constant and distinct variable symbols. 
%The language $L(p)$ of a pattern $p$ is the set of constant strings obtained by replacing each variable with a constant string. 
%Let $\mathcal{RP}^k$ denote the class of all sets containing at most $k$ $(k \ge 2)$ regular patterns. 
%Sato et al.\ (Proc.\ ALT'98, 1998) showed that the finite set $S_2(P)$, obtained from $P \in \mathcal{RP}^k$ by substituting variables with constant strings of length at most two, serves as a characteristic set for $L(P) = \bigcup_{p \in P} L(p)$. 
%They also claimed that $\mathcal{RP}^k$ is compact with respect to language containment when the number of constant symbols is at least $2k - 1$.
%In this paper, \textbf{we revisit their results and point out an error in the original proof}. 
%We then \textbf{provide a new and correct proof} by introducing additional conditions that guarantee the validity of their claim. 
%We further study the class $\mathcal{RP}^k_{\mathrm{NAV}}$ of at most $k$ $(k \ge 1)$ \textit{non-adjacent regular patterns}, where no two variables appear consecutively. 
%For any $P \in \mathcal{RP}^k_{\mathrm{NAV}}$, we show that $S_2(P)$ is a characteristic set of $L(P)$, and that $\mathcal{RP}^k_{\mathrm{NAV}}$ is compact with respect to containment if the number of constant symbols is at least $k + 2$. 
%These results enable efficient learning of finite unions of non-adjacent regular pattern languages with fewer constant symbols than in the general case.

{\color{red}
A \textit{regular pattern} is a string consisting of constant and distinct variable symbols. 
The language $L(p)$ of a pattern $p$ is defined as the set of all constant strings obtained by replacing each variable with a constant string. 
Let $\mathcal{RP}^k$ denote the class of all sets containing at most $k$ $(k \ge 2)$ regular patterns. 
Sato et al.\ (Proc.\ ALT'98, 1998) showed that the finite set $S_2(P)$, obtained from $P \in \mathcal{RP}^k$ by replacing variables with constant strings of length at most two, serves as a characteristic set for the language $L(P) = \bigcup_{p \in P} L(p)$. 
They also claimed that $\mathcal{RP}^k$ is compact with respect to language containment when the number of constant symbols is at least $2k - 1$.
In this paper, we revisit their results and identify an error in the original proof of their theorem. 
We then present a new and correct proof by introducing additional conditions that guarantee the validity of their claim. 
Furthermore, we study the subclass $\mathcal{RP}^k_{\mathrm{NAV}}$, consisting of at most $k$ $(k \ge 1)$ \textit{non-adjacent regular patterns}, in which no two variable symbols occur consecutively. 
For any $P \in \mathcal{RP}^k_{\mathrm{NAV}}$, we prove that the set $S_2(P)$ serves as a characteristic set of $L(P)$ and that $\mathcal{RP}^k_{\mathrm{NAV}}$ is compact with respect to {\color{red} language} containment if the number of constant symbols is at least $k + 2$.
These results demonstrate that finite unions of non-adjacent regular pattern languages can be learned efficiently under weaker constraints on constant symbols than those required in the general case. 
Our analysis thus refines and extends the compactness properties of regular pattern languages originally discussed by Sato et al., providing a corrected theoretical foundation for subsequent studies on the learnability of pattern languages.
}

%\textbf{Keywords:} Regular Pattern language; compactness with respect to {\color{red} language} containment; non-adjacent regular patterns
\end{summary}
