\begin{summary}
  {\color{red}
A regular pattern is defined as a string composed of constant symbols and distinct variable symbols.
The language $L(p)$ of a regular pattern $p$ is the set of all constant strings obtained by replacing each variable symbol in $p$ with a constant string.
Let $\RPatkei$ denote the class of all sets containing at most $k~(k\geq 2)$ regular patterns.
Sato et al. (Proc. ALT'98, 1998) demonstrated that the finite set $S_2(P)$,
derived from a set $P\in \RPatkei$ by replacing variables with constant strings of length at most two, serves as a characteristic set for the language $L(P)=\bigcup_{p\in P}L(p)$.
They also established that $\RPatkei$  exhibits compactness with respect to {\color{red} language} containment when the number of constant symbols is greater than or equal to $2k-1$.
In this paper, we revisit their results and correct an error in the proof of their theorem by introducing additional conditions.
Specifically, we show that any generalization of the strings in $S_{2}(P)$ would violate the condition $p\{x:=r\}\preceq q$ for all $r\in S_{2}(P)$
where $p$ is a regular pattern in $P$ and $q$ is a regular pattern.
Further, we investigate the set $\NAVRP$ consisting of at most $k~(k\ge 1)$ non-adjacent regular patterns, that is, regular patterns in which no two variable symbols appear consecutively.
Further we show that for any $P\in \NAVRPkei$, the set $S_{2}(P)$ is a characteristic set of $L(P)$.
Additionally, we proved that $\NAVRPkei$ possesses compactness with respect to {\color{red} language} containment if the number of constant symbols is greater than or equal to $k+2$.
These results imply that it is possible to design an efficient learning algorithm for  finite unions of pattern languages generated by non-adjacent regular patterns, requiring fewer constant symbols than in the general case of regular patterns.
  }
\end{summary}
