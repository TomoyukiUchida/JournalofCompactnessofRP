\newif\ifkokyuroku\kokyurokufalse

%%%%%%%%%%%%%%%
% 講究録原稿の場合は次の一行を有効にする
%\kokyurokutrue

\ifkokyuroku
\documentclass[11pt,a4paper]{jarticle}
\else
\documentclass[10pt,a4paper,twocolumn]{jarticle}
\fi

%%%%%%%%%
% タイトル
\title{変数が隣接しない正規パターンにより \\ 定義される言語の有限和に対するコンパクト性}
%%%%%%%%%

\ifkokyuroku
\usepackage{authblk}

\renewcommand{\qedsymbol}{$\blacksquare$}
\renewcommand\Authsep{\qquad}
\renewcommand\Authand{\qquad}
\renewcommand\Authands{\qquad}

%%%%%%%%%
% 講究録用(著者情報)
\author[1]{著者 あ    author A}
\author[2]{著者 い    author asdf B}
\author[1]{著者 う  author asdf C}
\author[1]{\\著者 え    author D}
\author[2]{著者 お    author asdf E}

\affil[1]{
 ABC大学大学院DEF学研究科 \authorcr
 Graduate School of DEF, ABC University
}
\affil[2]{
 GHI大学JKL学部 \authorcr
 Department of JKL, GHI University
}
% end of 講究録用
%%%%%%%%%

\setlength{\topmargin}{-0.3cm}
\setlength{\textheight}{23cm}
\setlength{\oddsidemargin}{0.5cm}
\setlength{\textwidth}{15.0cm}

\else

%%%%%%%%%
% 予稿用(講演番号と著者情報)
\def\kouenbangou{3S} %数字を自分の講演番号に書き換えて下さい.
\author{
	\begin{tabular}[t]{lll}
	武田 直人\thanks{広島市立大学大学院情報科学研究科知能工学専攻 \\ \ \ \ \ \ (mh67011@e.hiroshima-cu.ac.jp)}  & 		内田 智之 \footnotemark[1]  & 	正代 隆義\thanks{福岡工業大学情報工学部情報工学科}\\
		松本 哲志\thanks{東海大学理学部情報数理学科} &		鈴木 祐介\footnotemark[1] & 	宮原 哲浩\footnotemark[1]
	\end{tabular}
	%「\footnotemark[N]」で第N著者と同じマークが付きます.
}
% end of 予稿用
%%%%%%%%%

\usepackage{amsmath}
\usepackage{amssymb}
\usepackage{amsfonts}
\usepackage{comment}
\usepackage{bm}
\usepackage{multicol}
\usepackage{amsthm}
\usepackage[dvipdfmx]{graphicx}
\usepackage{here} 
\allowdisplaybreaks

\newtheorem{dfn}{Definition} 
\newtheorem{thm}{Theorem}
\newtheorem{lem}{Lemma}
\newtheorem{col}{Corollary}
\newtheorem{ex}{Example}
\newtheorem{cl}{Claim}
\newenvironment{proof}{\noindent{\bf Proof.}}{\par\medskip}
\renewcommand{\labelenumi}{(\arabic{enumi})}
\newcommand{\proofname}{\textbf{Proof.}}
\newcommand{\qedsymbol}{(Q.E.D)}
\newcommand{\Pat}{\mathcal P}
\newcommand{\RPat}{\mathcal{RP}}
\newcommand{\PatL}{\mathcal{PL}}
\newcommand{\RPatL}{\mathcal{RPL}}
\newcommand{\Patplus}{\mathcal{P^{+}}}
\newcommand{\RPatplus}{\mathcal{RP^{+}}}
%\newcommand{\Patk}{\mathcal{\Pat^{k}}}
\newcommand{\RPatkei}{\mathcal{RP^{\mbox{$k$}}}}
\newcommand{\PatLkei}{\mathcal{PL^{\mbox{$k$}}}}
\newcommand{\NAVRP}{\mathcal{RP_{NAV}}}
\newcommand{\NAVRPplus}{\mathcal{RP^{+}_{NAV}}}
\newcommand{\NAVRPkei}{\mathcal{RP^{\mbox{$k$}}_{NAV}}}


\makeatletter
\def\ps@LAheadings{
	\def\@oddhead{}
	\def\@evenhead{}
	\def\@evenfoot{\hfil \kouenbangou\,--\,\thepage\hfill}
	\def\@oddfoot{\hfil \kouenbangou\,--\,\thepage\hfil}
}
\def\ps@LAtitleheadings{
	\def\@oddhead{2023年度冬のLAシンポジウム\,[\kouenbangou]\hfil}
	\def\@evenhead{}
	\def\@evenfoot{\hfil \kouenbangou\,--\,\thepage\hfill}
	\def\@oddfoot{\hfil \kouenbangou\,--\,\thepage\hfil}
}
\makeatother

\pagestyle{LAheadings}
\fi

\date{}



\begin{document}
\maketitle

\ifkokyuroku
\else
\thispagestyle{LAtitleheadings}
\fi

\input{abst}

\section{はじめに}
パターンとは,定数記号と変数記号から成る記号列である.
例えば,$a,b,c$を定数記号,$x,y$を変数記号とするとき,$axbxcy$はパターンである.
パターンから成る集合の全体を$\Pat$で表す.
パターン$p\in \Pat$に対し,すべての変数記号を空記号列$\varepsilon$でない定数記号列で置き換えて得られる記号列の集合を,\textbf{$\bm{p}$により生成されるパターン言語}あるいは単にパターン言語といい,$L(p)$と書く.
なお,同じ変数記号には同じ定数記号列で置き換える.
例えば,上記のパターン$axbxcy$により生成されるパターン言語$L(axbxcy)$は$\{ aubucw \mid w,u\mbox{は$\varepsilon$でない定数記号列} \}$を表す.
各変数記号が高々1回しか現れないパターンを\textbf{正規パターン}という.
例えば,パターン$axbxcy$は正規パターンではないが,変数記号$x,y,z$を持つパターン$axbzcy$は正規パターンである.
正規パターンから成る全体の集合を$\RPat$で表す.
パターン$p\in \Pat$がパターン$q\in\Pat$の変数記号をパターンで置き換えることで得られるとき,$q$は$p$の汎化といい,
$p\preceq q$と書く.
例えば,パターン$q=axz$はパターン$p=axbxcy$の汎化である.
$q$の変数記号$z$をパターン$bxcy$で置き換えると$p$が得られるからである.
よって,$p\preceq q$である.
パターン$p,q\in \Pat$に対して,$p\preceq q$ならば$L(p)\subseteq L(q)$であることは明らかである.
しかし,その逆,つまり$L(p)\subseteq L(q)$ならば$p\preceq q$は成り立つとは限らない.
これに対し,Mukouchi\cite{Mukouchi1991}は,定数記号の数が3以上の場合,任意の正規パターン$p,q\in \RPat$に対して,
$L(p)\subseteq L(q)$ならば$p\preceq q$も成り立つことを示した.

$\RPatplus$を$\RPat$の空でない有限集合の集合とする.
$\RPatkei$を高々$k~(k\geq 2)$個の正規パターンから成る集合の全体のクラスとする.
正規パターンの集合$P\in\RPatkei$に対し,$L(P)=\bigcup_{p\in P}L(p)$とし,
$\RPatkei$に対する正規パターン言語のクラス$\{L(P) \mid P\in \RPatkei\}$を$\RPatL^{k}$とする.
$P,Q\in \RPatkei$とし,$Q= \{ q_{1},\ldots ,q_{k} \}$とする.
任意の正規パターン$p\in P$に対し,ある正規パターン$q_{i}$が存在し,$p\preceq q_{i}$が成り立つとき
$P\sqsubseteq Q$と書く.
定義より,$P \sqsubseteq Q$ならば$L(P)\subseteq L(Q)$であることは明らかである.
そこで,Satoら\cite{Sato1}は,$k \ge 3$であり定数記号の数が$2k-1$であるとき,
各変数記号に対し長さが高々2の定数記号列を代入することで$P\in \RPatkei$から得られる定数記号列の有限集合$S_2(P)$が
$L(P)$の特徴集合であること,つまり任意の正規パターン言語$L'\in \RPatL^{k}$に対して,$S_2(P) \subseteq L'$ならば$L(P)\subseteq L'$となることを
示し,$(i) S_2(P) \subseteq L(Q)$,(ii) $P\sqsubseteq Q$ および (iii) $L(P)\subseteq L(Q)$が同値であることを示した.
しかし,この結果の根拠となる補題14\cite{Sato1}に誤りがあるため,
本稿では,まずその修正を行い,Satoらが示した3つの命題の同値性の正しい証明を与えた.
Satoら\cite{Sato1}は,定数記号の数が$2k-1$以上のとき,$\RPatkei$が包含に関してコンパクト性を持つことも示した.
これに対し,本稿では,隣接した変数記号(隣接変数)を持たない正規パターンである非隣接変数正規パターン全体の集合$\NAVRP$を与え,
高々$k~(k\ge 1)$個の非隣接変数正規パターンの集合全体のクラス$\NAVRPkei$に属する集合$P$から得られる$S_2(P)$が$L(P)$の特徴集合であることを示した.
さらに,定数記号の数が$k+2$以上のとき,$\NAVRPkei$が包含に関してコンパクト性を持つことを示した.
表\ref{表1}に本稿の結果をまとめて示す.
\begin{table}
	\begin{center}
	%\vspace{-0.4cm}
	\caption{包含に関してコンパクト性を持つための定数記号の数に関する条件}
	\label{表1}
	\begin{tabular}{llll}
	\hline
	\multicolumn{1}{|c|}{$k$}   & \multicolumn{1}{c|}{2}                            & \multicolumn{1}{c|}{3以上} \\ 
	\hline
	\hline
	\multicolumn{1}{|c|}{$\RPatkei$} & \multicolumn{1}{c|}{4以上}                          & \multicolumn{1}{c|}{$2k-1$以上} \\ \hline
	\multicolumn{1}{|c|}{$\NAVRPkei$} & \multicolumn{2}{c|}{$k+2$以上} \\ \hline
	%\vspace{-1cm}
	\end{tabular}
	\end{center}
	\end{table}

%これは,正規パターン言語のときに下界を示すために用いられた関数$2k-1$より正確な関数表現$k+2$を与えることができ,
本稿の結果は,言語の有限和の表現である正規パターンの集合あるいは非隣接変数正規パターンの集合を対象とした効率的な学習アルゴリズムをそれぞれ与えられることを示唆している.
%その定数記号の数の条件で,非隣接変数正規パターン言語の有限和に関する効率的な学習アルゴリズムが設計できることを示した.

本稿の構成は以下の通りである.
第2節では,準備としてパターン言語,正規パターン言語,コンパクト性などの定義を与え,さらに$\RPatplus$の特徴集合に関するSatoらの結果を紹介する.
第3節では,$S_{2}(P)$は$\RPatL^{k}$における$L(P)$の特徴集合であること,
および$\RPat^{k}$が包含に関するコンパクト性を持つことを示す.
第4節では,非隣接変数正規パターンを与え,$\NAVRPkei$に属する集合$P$から得られる$S_2(P)$が
$L(P)$の特徴集合であること,および$\NAVRPkei$が包含に関してコンパクト性をもつことを示す.



%Angluin \cite{Angluin1980}は,パターン言語族$\mathcal{PL}$はAngluinによって提案された.



\input{sec2-1}

%\section{特徴集合としての$S_{1}(P)$}
%この章では,$\sharp\Sigma \ge 2k+1$のとき,$S_{1}(P)$は$\mathcal{RPL}^{k}$における$L(P)$の特徴集合となることを示す..

$\RPat^{k}$について,任意のパターン$p\in \RPatplus$に対し,ある特定の有限部分集合$S \subseteq L(p)$が存在して,$S \subseteq L(Q)$ならば,ある$q \in Q$に対して$L(p) \subseteq L(q)$となることが知られている\cite{Mukouchi1991}.
また,$S \subseteq L(Q)$ならば$L(p) \subseteq L(Q)$である.
これにより,$S$は次の定義される$L(p)$の特徴集合であることがわかる.
%このような集合$S$は$L(p)$の特徴集合と呼ばれ,次のように定義される.
\begin{dfn}
$\mathcal{L}$を言語クラスとする.$L$を$\mathcal{L}$に属する言語とする.
空でない有限部分集合$S \subseteq \Sigma^{+}$は$\mathcal{L}$における$L$の\textbf{特徴集合}であるとは,
任意の$L^{\prime} \in \mathcal{L}$に対して$S \subseteq L^{\prime}$ならば$L \subseteq L^{\prime}$となるときをいう.
\end{dfn}
\noindent
%この2つの性質は,次のように定義される.
%\begin{dfn}[Wright\cite{Wright} and Motoki et al.\cite{Motoki1}]\label{fe}
%言語族$\mathcal{L}$が有限弾力性を持つとは,全ての$i \ge 1$に対して,次のような文字列の無限列($w_{i})_{i \ge 0}$と$\mathcal{L}$に含まれる言語の無限列($L_{i})_{i \ge 1}$が存在しないことをいう.
%\begin{equation*}
%{ w_{0}, \cdots , w_{i-1} } \subseteq L_{i}, ただし,w_{i} \not \in L_{i}
%\end{equation*}
%\end{dfn}
%\begin{dfn}[Sato\cite{Sato2}]\label{fc}
%$\mathcal{L}$を言語族とする.言語$L$が$\mathcal{L}$における有限交差性を持つとは,次のような文字列の有限集合の無限列$(T_{n})_{n \ge 1}$と$\mathcal{L}$に含まれる言語の無限列$(L_{i})_{i \ge 1}$が存在しないことをいう.
%\begin{align*}
%(\mathrm{i}) T_{1} \varsubsetneq T_{2} \varsubsetneq \cdots, (\mathrm{ii}) \cup_{i=1}^{\infty}T_{i}=L,
%(\mathrm{iii}) T_{i} \subseteq L_{i}, \\ただし,T_{i+1} \not \subseteq L_{i} (i \ge 1)
%\end{align*}
%\end{dfn}
%また,次の2つの補題が与えられる.
%\begin{lem}[Sato\cite{Sato2}]\label{fcfe}
%$\mathcal{L}$を言語族,$L$を言語とする.
%\begin{comment}
%\begin{align*}
%Lは\mathcal{L}における有限交差性を持つ \ \ \ \\
%\Leftrightarrow \ \ \ \ \ \ \ \ \ \ \ \ \ \ \ \ \ \ \ \ \ \ \ \\
%全てのLは\mathcal{L}における有限弾力性を持つ
%\end{align*}
%\end{comment}
%\begin{align*}
%Lは\mathcal{L}における有限交差性を持つ \Leftrightarrow \\
%全てのLは\mathcal{L}における有限弾力性を持つ
%\end{align*}
%\end{lem}
%\begin{lem}[Sato et al.\cite{Sato1}]\label{fcc}
%$\mathcal{L}$を言語族,$L$を言語とする.
%\begin{align*}
%Lは\mathcal{L}における有限交差性を持つ \Leftrightarrow \\ 
%\mathcal{L}におけるLの特徴集合が存在する
%\end{align*}
%\end{lem}
%Wright\cite{Wright}は$\mathcal{PL}^{k}$,その部分集合である$\mathcal{RPL}^{k}$が有限弾力性を持つことを示した.
%したがって,補題\ref{fcfe}, 補題\ref{fcc}より, $\mathcal{RPL}^{k}$における$L$の特徴集合が存在する.

$m~(m\geq 0)$個の変数記号$x_{1},\ldots, x_{m}$を含む正規パターン$p$と$n~(n\geq 1)$に対して,%次のような$L(p)$の部分集合$S_{n}(p)$を定義する.
%\begin{dfn}\label{部分集合}
$p$中の各変数記号に長さが高々$n$の$\Sigma^{+}$の定数記号列を代入して得られるすべての定数記号列の集合を$S_{n}(p)$で表す.
さらに,正規パターンの空でない有限集合$P$に対して,
%\begin{equation*}
$S_{n}(P)= \bigcup_{p \in P} S_{n}(p)$
%\end{equation*}
%\end{dfn}
とする.
このとき,
任意の自然数$n~(n \ge 1)$に対して, $S_{n}(P) \subseteq S_{n+1}(P) \subseteq L(P)$である.
%$L(P)$の特徴集合は有限集合であるため,
よって,次の定理が成り立つ.
\begin{thm}[Sato et al.\cite{Sato1}]
任意の$P \in \RPat^{k}$に対して,$S_{n}(P)$がクラス$\RPatL^{k}$内の正規パターン言語$L(P)$の特徴集合であるような自然数$n~(n \ge 1)$が存在する.
\end{thm}

$p_{1},p_{2},r,q$を正規パターンとし,
$p_{1}rp_{2} \preceq q$が成り立つとする.
また,$x_{1}, \ldots, x_{n}$を$q$に含まれる変数記号とする.
このとき,$q=q_{1}x_{i}q_{2}$に対して,$p_{1}=(q_{1} \theta )r^{\prime}$かつ$p_{2}=r^{\prime\prime}(q_{2} \theta )$を満たす変数記号$x_{i}$と代入$\theta =
\{ x_{1} := r_{1}, \ldots , x_{i} := r^{\prime}rr^{\prime\prime}, \ldots , x_{n} := r_{n} \}$が存在すれば,$p_{1}rp_{2}$に含まれる正規パターン$r$は$q$の変数記号への代入により生成できる.
よって,$p_{1}rp_{2}$に含まれる正規パターン$r$が$q$の変数記号への代入により生成できるとき,$p_{1}xp_{2} \preceq q$が成り立つ.

\begin{lem}[Sato et al.\cite{Sato1}]\label{補題9}
    $p=p_{1}xp_{2}, \ q=q_{1}q_{2}q_{3}$を正規パターンとする.
    以下の{\rm (i), (ii), (iii)}がすべて成り立つとき,$p \preceq q$である.
    \[
    \begin{tabular}{ll}
    $(\mathrm{i})$ $p_{1} \preceq q_{1}q_{2},$
    $(\mathrm{ii})$ $p_{2} \preceq q_{2}q_{3},$\\
    $(\mathrm{iii})$ $q_{2}$は変数記号を含む.
    \end{tabular}
    \]
\end{lem}
\begin{proof}
$y$を$q_{2}$に含まれる変数記号とし,$q_{2}=q_{2}^{\prime}yq_{2}^{\prime \prime}$とする.
$p_{1} \preceq q_{1}q_{2}=q_{1}(q_{2}^{\prime}yq_{2}^{\prime \prime})$より,$p_{1}^{\prime} \preceq q_{1}q_{2}^{\prime}$かつ$p_{1}^{\prime\prime} \preceq yq_{2}^{\prime\prime}$となるような$p_{1}^{\prime},p_{1}^{\prime\prime}$を定義すると,$p_{1}=p_{1}^{\prime}p_{1}^{\prime\prime}$となる.
同様に,$p_{2} \preceq q_{2}q_{3}=(q_{2}^{\prime}yq_{2}^{\prime\prime})q_{3}$より,$p_{2}^{\prime} \preceq q_{2}^{\prime}y$かつ$p_{2}^{\prime\prime} \preceq q_{2}^{\prime\prime}q_{3}$となるような$p_{2}^{\prime},p_{2}^{\prime\prime}$を定義すると,$p_{2}=p_{2}^{\prime}p_{2}^{\prime\prime}$となる.
このとき,$p=p_{1}xp_{2}=p_{1}^{\prime}(p_{1}^{\prime\prime}xp_{2}^{\prime})p_{2}^{\prime\prime} \preceq q_{1}q_{2}^{\prime}(p_{1}^{\prime\prime}xp_{2}^{\prime})q_{2}^{\prime\prime}q_{3}=q\theta \preceq q$となる.
\end{proof}

ある$a \in  \Sigma$に対して,$p \{ x:=a \} \preceq q$のとき,$p_{1}xp_{2} \not \preceq q$ならば,$p_{1}ap_{2}$の定数記号$a$は,$q$の変数記号への代入によって生成することはできない.
すなわち,$p_{1} \preceq q_{1}$かつ$p_{2} \preceq q_{2}$を満たす$q=q_{1}aq_{2}$が存在する.
これにより,次の補題が得られる.
\begin{lem}[Sato et al.\cite{Sato1}]\label{補題10}
$\sharp \Sigma \ge 3$,$p=p_{1}xp_{2},q$を正規パターン,$a,~b,~c$を$\Sigma$に属する相異なる定数記号とする.
このとき,$p_{1}ap_{2} \preceq q$,$\ p_{1}bp_{2} \preceq q$かつ$p_{1}cp_{2} \preceq q$が成り立つならば,
$p\preceq q$が成り立つ.
\end{lem}
\begin{proof}
$p \not \preceq q$と仮定する.
このとき,$p_{1}ap_{2}$の$a$,~$p_{1}bp_{2}$の$b$,~$p_{1}cp_{2}$の$c$は$q$の変数記号を置き換えることによって生成できない.
よって,
\medskip

\begin{tabular}{llll}
(1) & $p_{1} \preceq q_{1}$ & (1') & $p_{2} \preceq q_{2}bq_{3}cq_{4}$ \\
(2) & $p_{1} \preceq q_{1}aq_{2}$ & (2') & $p_{2} \preceq q_{3}cq_{4}$ \\
(3) & $p_{1} \preceq q_{1}aq_{2}bq_{3}$ & (3') & $p_{2} \preceq q_{4}$
\end{tabular}	
\indent ($q_{1}, q_{2}, q_{3}, q_{4}$は正規パターン) 	
\medskip

\noindent を満たす$q=q_{1}aq_{2}bq_{3}cq_{4}$が存在する.
(2)と(1')より,$q_{2}$に変数記号が含まれる場合,補題\ref{補題9}より,$p \preceq q$となる.
これは仮定に矛盾する.
よって,$q_{2}$は定数記号列である.同様に,(3)と(2')より,$q_{3}$は定数記号列である.
したがって,$w=q_{2}, w^{\prime}=q_{3}$ ($w, w^{\prime}$は定数記号列)とおく.

$|w|=|w^{\prime}|$のとき,(2)と(3)より,$p_{1}$の接尾辞は$awbw^{\prime}$かつ$aw$である.
$|w|=|w^{\prime}|$より,$bw^{\prime}=aw$である.
これは,$b=a$となり,$a, b$が互いに異なる定数記号であることに矛盾する.

$|w| < |w^{\prime}|$のとき,(2)と(3)より,$p_{1}$の接尾辞は$awbw^{\prime}$かつ$aw$である.
$w^{\prime}=w_{1}w$とおくと,$awbw^{\prime}=awbw_{1}w$となる.
このとき,$w_{1}$の最後の記号は$a$となる.
(1')と(2')より,$p_{2}$の接頭辞は$wbw^{\prime}c$かつ$w^{\prime}c$である.
$w^{\prime}=w_{1}w$とおくと,$wbw^{\prime}c=wbw_{1}wc$となり,$w^{\prime}=ww_{2}$とおくと,$w^{\prime}c=ww_{2}c$となる.
$|wbw_{1}|=|ww_{2}c|$より,$w_{1}$の最後の記号は$c$となる.
よって,$w_{1}$の接尾辞は$a=c$となる.
これは,$a, c$が互いに異なる定数記号であることに矛盾する.

$|w| > |w^{\prime}|$のとき,(1')と(2')より,$p_{2}$の接頭辞は$wbw^{\prime}c$かつ$w^{\prime}c$である.
$w=w^{\prime}w_{1}$とおくと,$wbw^{\prime}c=w^{\prime}w_{1}bw^{\prime}c$となる.
このとき,$w_{1}$の最初の記号は$c$となる.
(2)と(3)より,$p_{1}$の接尾辞は$awbw^{\prime}$と$aw$である.
$w=w^{\prime}w_{1}$とおくと,$awbw^{\prime}=aw^{\prime}w_{1}bw^{\prime}$となり, $w=w_{2}w^{\prime}$とおくと,$aw=aw_{2}w^{\prime}$となる.
$|w_{1}bw^{\prime}|=|aw_{2}w^{\prime}|$より,$w_{1}$の最初の記号は$a$となる.
よって,$a=c$となる.
これは,$a, c$が互いに異なる定数記号であることに矛盾する.
\end{proof}
次の補題\ref{2個}は,相異なる定数記号$a, b$に対して,$p \{ x:=a \} \preceq q$かつ$p \{ x:=b \} \preceq q$ならば$p \not \preceq q$となる正規パターン$p, q$が存在することを示している.
\begin{lem}[Sato et al.\cite{Sato1}]\label{2個}
$\sharp\Sigma \ge 3$とする.%$p, q$を正規パターン, 
$a, b$を相異なる定数記号とする.
次の条件{\rm (i), (ii), (iii)}を満たす正規パターン$p=p_{1}AwxwBp_{2}$と$q=q_{1}AwBq_{2}$に対して,
$p \{ x:= a \} \preceq q$かつ$p \{ x:=b \} \preceq q$ならば$p \not \preceq q$である.
ここで,$p_{1}, p_{2}, q_{1}, q_{2}$は正規パターン,$w$は定数記号列である.
\[
    \begin{tabular}{ll}
        $\mathrm{(i)}~p_{1} \preceq q_{1}$,$\mathrm{(ii)}~p_{2} \preceq q_{2}$,\\
        $\mathrm{(iii)}~A=a, B=b \mathrm{ または } A=b, B=a$,
    \end{tabular}
\]

\end{lem}


\begin{comment}
\begin{proof}
$p=p_{1}^{\prime}xp_{2}^{\prime} \ (p_{1}^{\prime}, p_{2}^{\prime}$は正規パターン)とする.
補題\ref{補題10}と同様に考えると, $p \{ x:= a \} \preceq q, \ p \{ x:=b \} \preceq q, \ p \not \preceq q$より,
\medskip

\indent$(1) \ p_{1}^{\prime} \preceq q_{1}, \ (1^{\prime}) \ p_{2}^{\prime} \preceq wBq_{2}$ \\
\indent $(2) \ p_{1}^{\prime} \preceq q_{1}Aw, \ (2^{\prime}) \ p_{2}^{\prime}
 \preceq q_{2}$ 
\medskip

\noindent を満たす$q=q_{1}AwBq_{2}$が存在する.

(1), (2), (1$^{\prime}), (2^{\prime})$より,$p_{1}^{\prime}=p_{1}Aw, \ p_{2}^{\prime} = wBp_{2}$ ($p_{1} \preceq q_{1}, \ p_{2} \preceq q_{2}$)とおける.
よって,$p=p_{1}^{\prime}xp_{2}^{\prime}=p_{1}AwxwBp_{2}$となる.
\end{proof}
\end{comment}

補題\ref{補題10}より,次の定理が成り立つ.
\begin{thm}[Sato et al.\cite{Sato1}]\label{定理10}
$\sharp \Sigma \ge 2k+1$とし,$P \in \RPatplus,~Q \in \RPat^{k}$とする.
このとき,次の{\rm (i), (ii), (iii)}は同値である.
\[
\begin{tabular}{ll}
$(\mathrm{i})$ $S_{1}(P) \subseteq L(Q),$
$(\mathrm{ii})$ $P \sqsubseteq Q,$
$(\mathrm{iii})$ $L(P) \subseteq L(Q),$
\end{tabular}
\]
\end{thm}

次の例は,$\sharp \Sigma = 2k$における定理\ref{定理10}の反例である.
\begin{ex}\label{例題1}
$\Sigma = \{ a_{1}, \ldots , a_{k}, b_{1}, \ldots , b_{k} \}$を$2k$個の定数記号から成る集合,$p$を正規パターン,$Q = \{ q_{1}, \ldots , q_{k} \}$とする.
$w_{1}, \ldots , w_{k}$を
$w_{i} = w_{i+1}b_{i+1}a_{i+1}w_{i+1}$ $(i = 1,2, \ldots , k-1), w_{k} = \varepsilon$
のように定義する.				
\begin{eqnarray*}
p = x_{1}a_{1}w_{1}xw_{1}b_{1}x_{2},
q_{i} = x_{1}a_{i}w_{i}b_{i}x_{2},
\end{eqnarray*}
$p \{ x:=a_{i} \} \preceq q_{i}$かつ$p \{ x:=b_{i} \} \preceq q_{i}$ $(i = 1,2, \ldots , k)$である場合を考える.
$i=1$のとき,$p \{ x:=a_{1} \} = (x_{1}a_{1}w_{1})a_{1}(w_{1}b_{1}x_{2}) = q_{1} \{ x_{1} := x_{1}a_{1}w_{1} \} \preceq q_{1}$かつ$p \{ x:=b_{1} \} = q_{1} \{ x_{2} := w_{1}b_{1}x_{2} \} \preceq q_{1}$となる.
$i \ge 2$のとき,$w_{i}$の定義より,ある記号列$w^{(i)},w^{\prime (i)}$に対して,$w_{1} = (w_{i}b_{i})w^{(i)} = w^{\prime (i)}(a_{i}w_{i})$となる.
したがって,任意の$i~(i \ge 2)$に対して, 
\begin{eqnarray*}
p \{ x:=a_{i} \} & = & (x_{1}a_{1}w_{1})a_{i}(w_{1}b_{1}x_{2})\\
& = & (x_{1}a_{1}w_{1})a_{i}(w_{i}b_{i}w^{(i)})b_{1}x_{2}\\
& = & (x_{1}a_{1}w_{1})(a_{i}w_{i}b_{i})(w^{(i)}b_{1}x_{2})\\
& = & q_{i} \{ x_{1} := x_{1}a_{1}w_{1}, x_{2} := w^{(i)}b_{1}x_{2} \}\\
& \preceq & q_{i},\\
p \{ x:=b_{i} \} & = & (x_{1}a_{1}w_{1})b_{i}(w_{1}b_{1}x_{2})\\
& = & x_{1}a_{1}(w^{\prime (i)}a_{i}w_{i})b_{i}(w_{1}b_{1}x_{2}) \\
& = & (x_{1}a_{1}w^{\prime (i)})a_{i}w_{i}b_{i}(w_{1}b_{1}x_{2}) \\
& = & q_{i} \{ x_{1} := x_{1}a_{1}w^{\prime (i)}, x_{2} := w_{1}b_{1}x_{2} \}\\
& \preceq & q_{i}.
\end{eqnarray*}

したがって,$S_{1}(p) \subseteq L(Q)$である. 
一方で,$p \not \preceq q_{i}$であるため,
$L(p) \not \subseteq L(q_{i})$ $(i=1, \ldots , k)$である.
\end{ex}

定理\ref{定理10}より,次の系が得られる.
\begin{col}[Sato et al.\cite{Sato1}]
$\sharp \Sigma \ge 3$とし,$p,q$を正規パターンとする.
このとき,次の{\rm (i), (ii), (iii)}は同値である.
\[
\begin{tabular}{ll}
$(\mathrm{i})$ $S_{1}(p) \subseteq L(q),$
$(\mathrm{ii})$ $p \preceq q,$
$(\mathrm{iii})$ $L(p) \subseteq L(q).$
\end{tabular}
\]
\end{col}

\input{new-sec3}

\section{非隣接変数正規パターン}

隣接した変数記号を持たない正規パターンを\textbf{非隣接変数正規パターン}という.
例えば,パターン$axybc$は正規パターンであるが,非隣接変数正規パターンではない.パターン$axbcy$は非隣接変数正規パターンである.
$\NAVRP$を非隣接変数正規パターン全体の集合とする.
$\NAVRP$の空でない有限部分集合の集合を$\NAVRPplus$で,
高々$k~(k\geq 1)$個のパターンから成る$\NAVRP$の部分集合$\{P\in \NAVRPplus \mid \sharp P \leq k\}$を$\NAVRPkei$で表す.
このとき,次の定理が成り立つ.

\begin{thm}\label{非隣接kが4以上}
$\sharp \Sigma \ge k+2,P\in \NAVRPplus,Q \in \NAVRPkei$とする.
このとき,以下の{\rm (i), (ii), (iii)}は同値である.
\[
\begin{tabular}{ll}
$(\mathrm{i})$ $S_{2}(P) \subseteq L(Q),$
$(\mathrm{ii})$ $P \sqsubseteq Q,$
$(\mathrm{iii})$ $L(P) \subseteq L(Q).$
\end{tabular}
\]
\end{thm}

\begin{proof}
定義より,
(ii) $\Rightarrow$ (iii)と(iii) $\Rightarrow$ (i)は自明に成り立つ.
よって,(i) $\Rightarrow$ (ii)が成り立つことを,$p$に現れる変数記号の数$n$に関する数学的帰納法で証明する.

$n=0$のとき,$S_{2}(p)= \{ p \}$であり,$p \in L(Q)$となる.よって,ある$q \in Q$に対して,$p \preceq q$となる.

$n \ge 0$個の変数記号を含む任意の正規パターンに対して,題意が成り立つと仮定する.
$p$を$S_{2}(p) \subseteq L(Q)$を満たす$n+1$個の変数記号を含む非隣接変数正規パターンとする.
$p \not \preceq q_{i}$ ($i=1, 2$)と仮定する.
非隣接変数正規パターン$p$を$p=p_{1}xp_{2}$, $Q=\{ q_{1}, \ldots , q_{k} \}$とおく.
ここで,$p_{1}$は末尾が定数記号である非隣接変数正規パターンであり,$p_{2}$は先頭が定数記号である非隣接変数正規パターン,$x$は変数記号,任意の$i$ ($i=1, \ldots, k$)に対して,$q_{i}$は非隣接変数正規パターンである.
$a, b \in \Sigma$に対して,$p_{a}=p \{ x := a \}$,$p_{ab}=p \{ x := ab \}$とおく.
このとき,$p_{a}, p_{ab}$は$n$個の変数記号が含まれ,$S_{2}(p_{a}) \subseteq L(Q)$かつ$S_{2}(p_{ab}) \subseteq L(Q)$が成り立つことに注意する.
帰納法の仮定より,任意の$a, b \in \Sigma$に対して,$p_{a} \preceq q_{i}$かつ$p_{ab} \preceq q_{i^{\prime}}$を満たすような$i, i^{\prime} \le k$が存在する.

補題\ref{追加補題1}より,ある$i$に対して$p \{ x:=xy \} \preceq q_{i}$が成り立つ.
このとき,$p \{ x:=xy \} =p_{1}xyp_{2}$の部分パターン$xy$は$q_{i}$の変数記号を置き換えることで生成できない.
このことは,$q_{i}$に$xy$が含まれることを示している.
これは,$q_{i}$が非隣接変数正規パターンであることに矛盾する.

以上より,(i) $\Rightarrow$ (ii)が成り立つ.
\end{proof}

\begin{col}
$\sharp\Sigma \ge k+2$,$P \in \NAVRPplus$とする.このとき,$S_{2}(P)$は$\mathcal{RPL^{\mbox{$k$}}_{NAV}}$における$L(P)$の特徴集合である.
\end{col}

\begin{lem}\label{k+2のとき}
$\sharp\Sigma \le k+1$とする.このとき,$\NAVRPkei$は包含に関してコンパクト性を持たない.
\end{lem}
\begin{proof}
$\Sigma = \{ a_{1}, \ldots , a_{k+1} \}$を$k+1$個の定数記号から成る集合,$p, q_{i}$を正規パターンとする.
$p \{ x := a_{i}y \} \preceq q_{i}$かつ$p \{ x := ya_{i+1} \} \preceq q_{i}~(i=1,2, \ldots ,k)$とする.
$p \{ x:= a_{k+1}a_{1} \} \preceq q_{1}$であるとき,$S_{2}(p) \backslash S_{1}(p) \subseteq \bigcup^{k}_{i=1} L(q_{i})$となる. 
すなわち,$L(p) \subseteq L(Q)$である.
しかし,$p \not \preceq q_{i}$であるため,$L(p) \not \subseteq L(q_{i})~(i=1,2, \ldots k)$である.
したがって,$\NAVRPkei$は包含に関するコンパクト性を持たない.
\end{proof}

コンパクト性をもたない例を例\ref{反例k+1}に示す.
\begin{figure*}[tb]
\begin{ex}\label{反例k+1}
$\Sigma= \{a_{1}, a_{2}, a_{3},a_{4} \}$を$4$つの定数記号から成る集合,$p,q_{1},q_{2},q_{3}$を正規パターン,$x,x^{\prime},x^{\prime\prime}$を変数記号とする.
$p,q_{1},q_{2},q_{3}$を以下のように定義する.
\begin{align*}
p & = x^{\prime}a_{3}a_{1}a_{4}a_{1}a_{4}a_{1}a_{1}a_{4}a_{1}a_{3}a_{2}a_{1}a_{4}a_{1}a_{4}a_{1}a_{1}a_{4}a_{1}xa_{1}a_{4}a_{1}a_{4}a_{1}a_{1}a_{4}a_{1}a_{3}a_{2}a_{1}a_{4}a_{1}a_{4}a_{1}a_{1}a_{4}a_{1}a_{2}x^{\prime\prime},\\
q_{1} & = x^{\prime}a_{3}a_{1}a_{4}a_{1}a_{4}a_{1}a_{1}a_{4}a_{1}a_{3}a_{2}a_{1}a_{4}a_{1}a_{4}a_{1}a_{1}a_{4}a_{1}a_{2}x^{\prime\prime},\\
q_{2} & = x^{\prime}a_{2}a_{1}a_{4}a_{1}a_{4}a_{1}a_{1}a_{4}a_{1}a_{3}x^{\prime\prime},\\
q_{3} & = x^{\prime}a_{1}a_{1}a_{4}a_{1}a_{4}x^{\prime\prime}.
\end{align*}

これは,$L(p) \subseteq L(q_{1}) \cup L(q_{2}) \cup L(q_{3})$となる.
しかし,$p \not \preceq q_{1},p \not \preceq q_{2}$かつ$p \not \preceq q_{3}$である.
\end{ex}
\end{figure*}

定理\ref{非隣接kが4以上}と補題\ref{k+2のとき}より,次の定理が成り立つ.

\begin{thm}
$\sharp\Sigma \ge k+2$とする.
このとき,$\RPat^{k}$は包含に関してコンパクト性を持つ.
\end{thm}

\section{おわりに}
本稿では,高々$k$ $(k\ge 2)$個の正規パターン集合全体のクラス$\RPatkei$について,(1) 正規パターン集合$P\in\RPatkei$から得られる記号列の集合$S_2(P)$が$P$により生成される言語$L(P)$の特徴集合となること,
および(2) $\RPatkei$が包含に関してコンパクト性を持つこと
を示したSatoら\cite{Sato1}の結果の証明の誤りを修正した.
次に,隣接する変数がない正規パターンである非隣接変数正規パターンについて,
高々$k(k\ge 3)$個の非隣接変数正規パターン集合全体のクラス$\NAVRPkei$から得られる記号列の集合$S_2(P)$が,
正規パターン言語の有限和に対する特徴集合と,定数記号の数が$k+2$以上のとき,
$\NAVRPkei$が包含に関してコンパクト性をもつことを示した.
これらにより,Arimuraら\cite{Arimura1994}が示した$\RPatkei$に対する学習アルゴリズムを非隣接変数正規パターン言語の有限和に関する効率的な学習アルゴリズムが設計できることを示した.

今後の課題として,$\NAVRPkei$に対する特徴集合を活用し,非隣接変数正規パターン言語の有限和を正例から極限同定する多項式時間帰納推論アルゴリズム
および一つの正例と多項式回の所属性質問を用いて同定する質問学習アルゴリズムの高速化が考えられる.
また,正規パターン言語の有限和に対する特徴集合の概念を
線形項木パターン言語\cite{Suzuki2006}の有限和や正則FGS言語\cite{Uchida1994}に拡張することが考えられる.



\section*{謝辞}
本研究はJSPS科研費 19K12103, 20K04973, 21K12021, 22K12172の助成を受けたものである.

\input{LA-summer.bib}

\end{document}
