%補題\ref{追加補題1},補題\ref{補題15}より,次の定理が成り立つ.
From the Lemma\ref{追加補題1} and Lemma\ref{補題15},
we have the following theorem.

% \begin{thm}\label{定理17}
%     $k \ge 3$,$\sharp \Sigma \ge 2k-1$,$P \in \RPatplus,~Q \in \RPat^{k}$とする.
%     このとき,以下の{\rm (i),(ii),(iii)}は同値である.
%     \[
%         \begin{tabular}{ll}
%             $(\mathrm{i})$ $S_{2}(P) \subseteq L(Q)$,
%             $(\mathrm{ii})$ $P \sqsubseteq Q$,
%             $(\mathrm{iii})$ $L(P) \subseteq L(Q)$.
%         \end{tabular}
%     \]
% \end{thm}

\begin{thm}\label{定理17}
    Let $k \geq$, $\sharp\Sigma \geq 2k-1$, $P \in \RPatplus$ and
    $Q \in \RPatkei$.
    Then, the following (i),(ii) and (iii) are equivalent:
    %このとき,以下の{\rm (i),(ii),(iii)}は同値である.
    \[
        \begin{tabular}{ll}
            $(\mathrm{i})$ $S_{2}(P) \subseteq L(Q)$,
            $(\mathrm{ii})$ $P \sqsubseteq Q$,
            $(\mathrm{iii})$ $L(P) \subseteq L(Q)$.
        \end{tabular}
    \]
\end{thm}


\begin{proof}
    it is clear that (ii) implies (iii) and (iii) implies (i).
    %(ii) $\Rightarrow$ (iii)と(iii) $\Rightarrow$ (i)は自明である.
    From Theorem\ref{定理10}, if $\sharp\Sigma \geq 2k+1$, then
    (i) implies (ii).
    % 定理\ref{定理10}より,$\sharp\Sigma \ge 2k+1$のとき,
    % (i) $\Rightarrow$ holds.
    Let $\sharp Q = k$, $p \in P$, $\sharp\Sigma = 2k-1$ or $2k$.
    Then, we show that (i) implies (ii).
    It suffices to show that $S_{2}(p) \subseteq L(Q)$ implies $P \sqsubseteq Q$
    for any regular pattern $p \in \RPat$.
    The proof is done by mathematical induction on $n$, where $n$ is the number of variable symbols appears in $p$.

    % よって,$\sharp Q=k$のとき,$\sharp\Sigma = 2k-1$または$\sharp\Sigma = 2k$の場合,
    % % (i) $\Rightarrow$ (ii)が成り立つことを,$p$に含まれる変数記号の数$n$に関する
    % 数学的帰納法により証明する.

    % $n=0$のとき,$S_{2}(p) = \{ p \}$であり,$p \in L(Q)$となる.
    % よって,ある$q \in Q$に対して,$p \preceq q$となる.
    In case $n=0$, $S_{2}(p) = \{p\}$.
    By (i), we have $\{p\} = L(Q)$. Thus, $p \preceq q$ for some $q \in Q$.

    For $n \geq 0$,
    we assume that it is valid for any regular pattern $p$
    with $n$ variable symbols.
    %$n \ge 0$個の変数記号を含む任意の正規パターンに対して題意が成り立つと仮定する.
    Let $p$ be a regular pattern such that $n+1$ variable symbols appear in $p$
    and $S_{2}(p) \subseteq L(Q)$.

    %$p$を$S_{2}(p) \subseteq L(Q)$を満たす$n+1$個の変数記号を含む正規パターンとする.
    We assume that $p \not\sqsubseteq Q$, that is, $p \not\preceq q_{i}$
    for any $i \in \{1,\ldots,k\}$.
    %$p \not \preceq q_{i}$ ($i=1, \ldots, k$)と仮定する.
    Let $Q = \{q_{1},\ldots,q_{k}\}$ and
    $p_{1}$, $p_{2}$ regular patterns, $x$ a variable symbol with
    $p = p_{1}xp_{2}$.
    % $p=p_{1}xp_{2}$ ($p_{1},p_{2}$は正規パターン,$x$は変数記号),
    % $Q=\{ q_{1}, \ldots , q_{k} \}$を考える.
    For $a, b \in \Sigma$,
    let $p_{a}=p\{x:=a\}$ and $p_{ab}=p\{x:=ab\}$.
    %$a, b \in \Sigma$に対して,$p_{a}=p \{ x := a \}$と$p_{ab}=p \{ x := ab \}$とおく.
    Both $p_{a}$ and $p_{ab}$ have $n$ variable symbols respectively.
    Thus, $S_{2}(p_{a}) \subseteq L(Q)$ and $S_{2}(p_{ab}) \subseteq L(P)$ hold.
    % このとき,$p_{a},p_{ab}$は$n$個の変数記号が含まれ,
    % $S_{2}(p_{a}) \subseteq L(Q)$かつ$S_{2}(p_{ab}) \subseteq L(Q)$が成り立つことに注意する.
    By the induction hypothesis,
    there exist $i, i' \in \{1,\ldots,k\}$ such that
    $p_{a} \preceq q_{i}$ and $p_{ab} \preceq q_{i'}$.
    % 帰納法の仮定より,任意の$a,b \in \Sigma$に対して,
    % $p_{a} \preceq q_{i}$かつ$p_{ab} \preceq q_{i^{\prime}}$を満たすような$i, i^{\prime} \le k$が存在する.
    Let $D_{i} = \{a \in \Sigma \mid p\{x:=a\} \preceq q_{i}\}$ \ ($i=1,\ldots,k$).
    % $D_{i}=\{ a \in \Sigma \mid p \{ x:=a \} \preceq q_{i} \}$ 
    % \ ($i=1, \ldots, k$)とする.
    We assume that $\sharp D_{i} \geq 3$ for some $i \in \{1,\ldots, k\}$.
    By Lemma~\ref{補題10}, we have $p \preceq q_{i}$.
    This contradicts the assumption.
    Thus, we have $\sharp D_{i} \leq 2$ for any $i \in \{1,\ldots,k\}$.
    % ある$i$に対して,$\sharp D_{i} \ge 3$であるとき,
    % 補題\ref{補題10}より,$p \preceq q_{i}$となる.これは仮定に矛盾する.
    % よって,$\sharp D_{i} \le 2$ ($i=1, \ldots, k$)となる場合を考える.
    If $\sharp\Sigma = 2k-1$, then
    $\sharp D_{i}=2$ or $\sharp D_{i}=1$ for any $i \in \{1,\ldots,k\}$.
    Moreover,
    If $\sharp\Sigma = 2k$, then
    $\sharp D_{i}=2$ for any $i \in \{1,\ldots,k\}$.
    % $\sharp\Sigma = 2k-1$のとき,任意の$i$に対して,
    % $\sharp D_{i}=2$または$\sharp D_{i}=1$,
    % $\sharp\Sigma = 2k$のとき,任意の$i$に対して,$\sharp D_{i}=2$となる.
    Since $k \geq 3$, $2k+1 \geq k+2$ holds.
    %$k \ge 3$であるとき,$2k+1 \ge k+2$となる.
    By Lemma~\ref{追加補題1},
    %よって,補題\ref{追加補題1}より,
    there exists $i \in \{1,\ldots,k\}$ such that $p\{x:=xy\} \preceq q_{i}$.
    %$p \{ x:=xy \} \preceq q_{i}$となる$i$が存在する.
    Therefore, by Lemma~\ref{補題15}, we have $p \preceq q_{i}$.
    %したがって,補題\ref{補題15}より,$p \preceq q_{i}$となる.
    This contradicts the assumption.
    %これは仮定に矛盾する.
    Thus, (i) implies (ii).
    %以上より,(i) $\Rightarrow$ (ii)が成り立つ.
\end{proof}

%この定理\ref{定理17}より,次の系が得られる.
From Theorem~\ref{定理17}, the following corollary holds.

% \begin{col}\label{命題18}
%     $k \ge 3$,$\sharp\Sigma \ge 2k-1$,$P \in \mathcal{RP}^{+}$とする.このとき,$S_{2}(P)$は$\mathcal{RPL}^{k}$における$L(P)$の特徴集合である.
% \end{col}

\begin{col}\label{命題18}
    Let $k \geq 3$, $\sharp\Sigma \geq 2k-1$ and $P \in \RPatplus$.
    %このとき,$S_{2}(P)$は$\mathcal{RPL}^{k}$における$L(P)$の特徴集合である.
    Then, $S_{2}(P)$ is a characteristic set for $L(P)$ within $\RPatLkei$.
\end{col}

% \begin{lem}[Sato et al.\cite{Sato1}]\label{補題19}
%     $\sharp\Sigma \le 2k-2$とする.このとき,$\mathcal{RP}^{k}$は包含に関するコンパクト性を持たない.
% \end{lem}

% \begin{lem}[Sato et al.\cite{Sato1}]\label{補題19}
%     Let $k
%     $\sharp\Sigma \le 2k-2$とする.
%         このとき,$\mathcal{RP}^{k}$は包含に関するコンパクト性を持たない.
% \end{lem}

\begin{lem}[Sato et al.\cite{Sato1}]\label{補題19}
    Let $k \geq 3$ and $\sharp\Sigma \leq 2k-2$.
    Then, $\RPatkei$ does not have compactness with respect to containment.
\end{lem}

% \begin{proof}
%     $\Sigma = \{ a_{1}, \ldots , a_{k-1}, b_{1}, \ldots , b_{k-1} \}$を$(2k-2)$個の定数記号から成る集合,$p, q_{i}$を正規パターン,$w_{i}~(i = 1, \ldots , k-1)$を例\ref{例題1}と同様に定義された記号列とする.
%     $q_{k} = x_{1}a_{1}w_{1}xyw_{1}b_{1}x_{2}$とする.
%     例\ref{例題1}で示した通り,$p \{ x := a_{i} \} \preceq q_{i}$かつ$p \{ x := b_{i} \} \preceq q_{i}~(i=1,2, \ldots ,k-1)$であるとき,$S_{1}(p) \subseteq \bigcup^{k-1}_{i=1} L(q_{i})$となる.
%     一方で,任意の$w$ $(|w| \ge 2)$に対して,$p \{ x:= w \} \preceq q_{k}$となる.
%     すなわち,$L(p) \subseteq L(Q)$である.
%     しかし,$p \not \preceq q_{i}$であるため,$L(p) \not \subseteq L(q_{i}) (i=1,2, \ldots k)$である.
%     したがって,$\RPatkei$は包含に関するコンパクト性を持たない.
% \end{proof}

\begin{lem}[Mukouchi\cite{Mukouchi1991}]\label{regularPatternEquivalence}
    Let $p$ and $q$ be regular patterns.
    Then $p \preceq q$ if and only if $L(p) \subseteq L(q)$.
\end{lem}

\begin{proof}
    Let $\Sigma =\{a_{1},\ldots,a_{k-1},b_{1},\ldots,b_{k-1}\}$ and
    $p, q_{i}$ regular patterns,$w_{i} \in \Sigma^{\ast}$\ $(i=1,\ldots,k-1)$
    defined in a similar way to Example~\ref{例題1}.
    Let $q_{k}=x_{1}a_{1}w_{1}xyw_{1}b_{1}x_{2}$.
    Since
    $p\{x:=a_{i}\} = x_{1}a_{1}w_{1}a_{i}w_{1}b_{1}x_{2} \preceq q_{i}$ and
    $p\{x:=b_{i}\} = x_{1}a_{1}w_{1}b_{i}w_{1}b_{1}x_{2} \preceq q_{i}$
    for any $i \in \{1,\ldots,k-1\}$,
    we have $S_{1}(p) \subseteq \bigcup_{i=1}^{k-1}L(q_{i})$.
    % 例\ref{例題1}で示した通り,$p \{ x := a_{i} \} \preceq q_{i}$かつ
    % $p \{ x := b_{i} \} \preceq q_{i}~(i=1,2, \ldots ,k-1)$であるとき,
    % $S_{1}(p) \subseteq \bigcup^{k-1}_{i=1} L(q_{i})$となる.
    For any $w \in \{s \in \Sigma^{+} \mid |s| \geq 2\}$,
    $p\{x:=w\}=x_{1}a_{1}w_{1}ww_{1}b_{1}x_{2} \preceq q_{k}$.
    % 一方で,任意の$w$ $(|w| \ge 2)$に対して,$p \{ x:= w \} \preceq q_{k}$となる.
    Thus, we have $L(p) \subseteq L(Q)$.
    %すなわち,$L(p) \subseteq L(Q)$である.
    By Theorem~\ref{regularPatternEquivalence},
    since $p \not \preceq q_{i}$, $L(p) \not \subseteq L(q_{i})$
    for any $i \in \{1,\ldots, k\}$.
    Therefore, $\RPatkei$ does not have compactness with respect to containment.
\end{proof}

%定理\ref{定理17}と補題\ref{補題19}より,次の定理が成り立つ.
From Theorem~\ref{定理17} and Lemma~\ref{補題19},
we have the following thorem.

% \begin{thm}
%     $k \ge 3$とし,$\sharp\Sigma \ge 2k-1$とする.
%     このとき,$\RPat^{k}$は包含に関してコンパクト性を持つ.
% \end{thm}

\begin{thm}
    Let $k \geq 3$ and $\sharp\Sigma \geq 2k-1$.
    Then, $\RPatkei$ has compactness with respect to containment.
\end{thm}

In case $k=2$, we have the following theorem.

% \begin{thm}\label{補題21}
%     $\sharp \Sigma \ge 4$とし,$P \in \RPatplus$,$Q \in \RPat^{2}$とする.
%     このとき,以下の{\rm (i),(ii),(iii)}は同値である.
%     \[
%         \begin{tabular}{ll}
%             $(\mathrm{i})$ $S_{2}(P) \subseteq L(Q)$,
%             $(\mathrm{ii})$ $P \sqsubseteq Q$,
%             $(\mathrm{iii})$ $L(P) \subseteq L(Q)$.
%         \end{tabular}
%     \]
% \end{thm}

\begin{thm}\label{補題21}
    Let $\sharp\Sigma \geq 4$, $P \in \RPatplus$ and $Q \in \RPat^{2}$.
    %$\sharp \Sigma \ge 4$とし,$P \in \RPatplus$,$Q \in \RPat^{2}$とする.
    %このとき,以下の{\rm (i),(ii),(iii)}は同値である. 
    The following (i), (ii) and (iii) are equivalent:
    \[
        \mathrm{(i)}\ S_{2}(P) \subseteq L(Q),\ \
        \mathrm{(ii)}\ P \sqsubseteq Q, \ \
        \mathrm{(iii)}\ L(P) \subseteq L(Q).
    \]
\end{thm}

% \begin{proof}
%     (ii) $\Rightarrow$ (iii)と(iii) $\Rightarrow$ (i)は自明に成り立つ.
%     よって,(i) $\Rightarrow$ (ii)が成り立つことを示す.
%     $Q= \{ q_{1}, q_{2} \}$とするとき,$p$に含まれる変数記号の数$n$に関する数学的帰納法で示す.\\
%     \noindent (1) $n=0$のとき,$p$は定数記号列となるので$S_{2}(p)= \{ p \}$となり
%     (i)より,$p \in L(Q)$となる.
%     よって,ある$q \in Q$に対して$p \preceq q$となる.\\
%     \noindent (2) $n=k$個の変数記号を含むすべての正規パターンに対して有効であると仮定する.
%     そして,$p$を$S_{2}(p) \subseteq L(Q)$を満たす$(n+1)$個の変数記号を含む正規パターンとする.

%     $p \not \preceq q_{i}$ ($i=1, 2$)と仮定する.
%     $p=p_{1}xp_{2}$ ($p_{1}, p_{2}$は正規パターン,$x$は変数記号)を考える.
%     $a, b \in \Sigma$に対して,$p_{a}=p \{ x := a \}$,$p_{ab}=p \{ x := ab \}$とおく.
%     このとき,$p_{a},p_{ab}$は$n$個の変数記号が含まれ,$S_{2}(p_{a}) \subseteq L(Q)$,$S_{2}(p_{ab}) \subseteq L(Q)$が成り立つことに注意する.
%     帰納法の仮定より,任意の$a, b \in \Sigma$に対して,$p_{a} \preceq q_{i}, p_{ab} \preceq q_{i^{\prime}}$を満たすような$i, i^{\prime} \le k$が存在する.

%     ある$i$に対して$\sharp D_{i} \ge 3$のとき,補題\ref{補題10}より,$p \preceq q_{i}$となる.
%     よって,任意の$i$に対して,$\sharp D_{i} \le 2$となる.
%     したがって,$\sharp D_{1}=2$かつ$\sharp D_{2}=2$となる場合を考える.

%     $\sharp \Sigma = k+2$であるとき,$k=2$より,$\sharp \Sigma =4$となる.
%     よって,補題\ref{追加補題1}より,ある$i$に対して,$p \{ x:=xy \} \preceq q_{i}$となる.
%     したがって,補題\ref{補題15}より,$p \preceq q_{i}$となる.
%     これは,仮定に矛盾する.

%     以上より,(i) $\Rightarrow$ (ii)が成り立つ.
% \end{proof}

\begin{proof}
    It is clear that (ii) implies (iii), and (iii) implies (i).
    %(ii) $\Rightarrow$ (iii)と(iii) $\Rightarrow$ (i)は自明に成り立つ.
    Thus, we show that (i) implies (ii).
    %よって,(i) $\Rightarrow$ (ii)が成り立つことを示す.
    It suffices to show that $S_{2}(p) \subseteq L(Q)$ implies $P \sqsubseteq Q$
    for any regular pattern $p \in \RPat$.
    Let $Q = \{q_{1}, q_{2}\}$.
    The proof is done by mathematical induction on $n$,
    where $n$ is the number of variable symbols appearing in $p$.
    % $Q= \{ q_{1}, q_{2} \}$とするとき,
    % $p$に含まれる変数記号の数$n$に関する数学的帰納法で示す.\\
    In case $n=0$, $p \in \Sigma^{+}$.
    Since $S_{2}(p) = \{p\} \subseteq L(Q)$, we have $p \preceq q$
    for some $q \in Q$.
    % \noindent (1) $n=0$のとき,$p$は定数記号列となるので$S_{2}(p)= \{ p \}$となり
    % (i)より,$p \in L(Q)$となる.
    % よって,ある$q \in Q$に対して$p \preceq q$となる.\\
    For $n \geq 0$, we assume that it is valid for any regular pattern $p$ with
    $n$ variable symbols.
    % \noindent (2) $n=k$個の変数記号を含むすべての正規パターンに対して有効であると仮定する.
    Let $p$ be a regular pattern such that $n+1$ variable symbols appear in $p$,
    ans $S_{2}(p) \subseteq L(Q)$.
    % そして,$p$を$S_{2}(p) \subseteq L(Q)$を満たす
    % $(n+1)$個の変数記号を含む正規パターンとする.
    We assume that $p \not\preceq q_{i}$ ($i=1,2$).
    Let $p_{1}$, $p_{2}$ be regular patterns and $x$ a variable symbol
    with $p = p_{1}xp_{2}$.
    %$p \not \preceq q_{i}$ ($i=1, 2$)と仮定する.
    %$p=p_{1}xp_{2}$ ($p_{1}, p_{2}$は正規パターン,$x$は変数記号)を考える.
    For $a, b \in \Sigma$,
    let $p_{a} = p\{x:=a\}$ and $p_{ab} = p\{x:=ab\}$.
    % $a, b \in \Sigma$に対して,$p_{a}=p \{ x := a \}$,
    % $p_{ab}=p \{ x := ab \}$とおく.
    Note that $p_{a}$ and $p_{ab}$ have $n$ variable symbols.
    %このとき,$p_{a},p_{ab}$は$n$個の変数記号が含まれ,
    Thus, by the assumption,
    $S_{2}(p_{a}) \subseteq L(Q)$ and $S_{2}(p_{ab}) \subseteq L(Q)$
    implies $p_{a} \preceq q_{i}$ and $p_{ab} \preceq q_{i'}$ for some
    $i, i' \in \{1,2\}$.
    % $S_{2}(p_{a}) \subseteq L(Q)$,$S_{2}(p_{ab}) \subseteq L(Q)$
    % が成り立つことに注意する.
    % 帰納法の仮定より,
    % 任意の$a, b \in \Sigma$に対して,
    % $p_{a} \preceq q_{i}, p_{ab} \preceq q_{i^{\prime}}$を満たすような
    % $i, i^{\prime} \le k$が存在する.
    %
    Let $D_{i} = \{a \in \Sigma \mid p\{x:=a\} \preceq q_{i}\}$ $(i=1,2)$.
    By Lemma~\ref{補題10}, if $\sharp D_{i} \geq 3$ for some $i \in \{1,2\}$,
    then $p \preceq q_{i}$.
    % ある$i$に対して$\sharp D_{i} \ge 3$のとき,補題\ref{補題10}より,
    % $p \preceq q_{i}$となる.
    This contradicts that $p \not\preceq q_{i}$ $(i=1,2)$.
    Thus, we have $\sharp D_{i} \leq 2$ for any $i \in \{1,2\}$.
    %よって,任意の$i$に対して,$\sharp D_{i} \le 2$となる.
    Since $\sharp \Sigma \geq 4$,
    %したがって,$\sharp D_{1}=2$かつ$\sharp D_{2}=2$となる場合を考える.
    We consider that $\sharp D_{1}=2$ and $\sharp D_{2} = 2$.
    %$\sharp \Sigma = k+2$であるとき,$k=2$より,$\sharp \Sigma =4$となる.
    From Lemma~\ref{追加補題1}, $p\{x:=xy\} \preceq q_{i}$ for some $i \in \{1,2\}$.
    % よって,補題\ref{追加補題1}より,ある$i$に対して,
    % $p \{ x:=xy \} \preceq q_{i}$となる.
    From Lemma~\ref{補題15}, we have $p \preceq q_{i}$ for some $i \in \{1,2\}$.
    %したがって,補題\ref{補題15}より,$p \preceq q_{i}$となる.
    これは,仮定に矛盾する.
    This contradicts that $p \not\preceq q_{i}$ ($i=1,2$).
    Therefore, (i) implies (ii).
    %以上より,(i) $\Rightarrow$ (ii)が成り立つ.
\end{proof}

%次の例は,$k = 2$における定理\ref{補題21}の反例である.
The next example is a counter-example of Theorem~\ref{補題21}.

% \begin{ex}\label{反例thm17}
%     $\Sigma= \{a, b, c \}$を$3$つの定数記号から成る集合,$p,q_{1},q_{2}$を正規パターン,$x,x^{\prime},x^{\prime\prime}$を変数記号とする.
%     \begin{eqnarray*}
%         p = x^{\prime}axbx^{\prime\prime},
%         q_{1} = x^{\prime}abx^{\prime\prime},
%         q_{2} = x^{\prime}cx^{\prime\prime}.
%     \end{eqnarray*}
%     $w \in \Sigma^{+}$とする.$w$に$c$が含まれるとき,$p \{ x:=w \} \preceq q_{2}$となり,$c$が含まれないとき,$p \{ x:=w \} \preceq q_{1}$となる.
%     よって,$L(p) \subseteq L(q_{1}) \cup L(q_{2})$である.
%     しかし,$p \not \preceq q_{1}$かつ$p \not \preceq q_{2}$である.
% \end{ex}

\begin{ex}\label{反例thm17}
    % $\Sigma= \{a, b, c \}$を$3$つの定数記号から成る集合,
    % $p,q_{1},q_{2}$を正規パターン,
    % $x,x^{\prime},x^{\prime\prime}$を変数記号とする.
    Let $\Sigma = \{a,b,c\}$, $p$, $q_{1}$, $q_{2}$ regular patterns and
    $x,x',x''$ variable symbols such that
    $p = x'axbx''$, $q_{1} = x'abx''$ and $q_{2} = x'cx''$.
    Let $w \in \Sigma^{+}$.
    If $w$ contains $c$, then $p \{x:=w\} \preceq q_{2}$.
    On the other hand, if $w$ does not contain $c$,
    then $p\{x:=w\} \preceq q_{1}$.
    Thus, $L(p) \subseteq L(q_{1}) \cup L(q_{2})$.
    However, $p \not\preceq q_{1}$ and $p \not\preceq q_{2}$.
\end{ex}

From Theorem~\ref{補題21}, we have that following two corollaries.

% \begin{col}
%     $\sharp\Sigma \ge 4$とし,$P \in \RPatplus$とする.
%     このとき,$S_{2}(P)$は,$\RPatL^{2}$における$L(P)$の特徴集合である.
% \end{col}

\begin{col}
    Let $\sharp\Sigma \geq 4$ and $P \in \RPatplus$.
    Then, $S_{2}(P)$ is a characteristic set for $L(P)$ within $\RPatL^{2}$.
    %このとき,$S_{2}(P)$は,$\RPatL^{2}$における$L(P)$の特徴集合である.
\end{col}

% \begin{col}
%     $\sharp\Sigma \ge 4$とする.このとき,クラス$\mathcal{RP}^{2}$は包含に関してコンパクト性を持つ.
% \end{col}

\begin{col}
    Let $\sharp\Sigma \geq 4$.
    Then, $\RPat^{2}$ has compactness with respect to containment.
\end{col}
