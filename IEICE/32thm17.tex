補題\ref{追加補題1},補題\ref{補題15}より,次の定理が成り立つ.

\begin{thm}\label{定理17}
$k \ge 3$,$\sharp \Sigma \ge 2k-1$,$P \in \RPatplus,~Q \in \RPat^{k}$とする.
このとき,以下の{\rm (i),(ii),(iii)}は同値である.
\[
\begin{tabular}{ll}
$(\mathrm{i})$ $S_{2}(P) \subseteq L(Q)$,
$(\mathrm{ii})$ $P \sqsubseteq Q$,
$(\mathrm{iii})$ $L(P) \subseteq L(Q)$.
\end{tabular}
\]
\end{thm}

\begin{proof}
(ii) $\Rightarrow$ (iii)と(iii) $\Rightarrow$ (i)は自明である.
定理\ref{定理10}より,$\sharp\Sigma \ge 2k+1$のとき,(i) $\Rightarrow$ (ii)は成り立つ.
よって,$\sharp Q=k$のとき,$\sharp\Sigma = 2k-1$または$\sharp\Sigma = 2k$の場合,(i) $\Rightarrow$ (ii)が成り立つことを,$p$に含まれる変数記号の数$n$に関する数学的帰納法により証明する.

$n=0$のとき,$S_{2}(p)= \{ p \}$であり,$p \in L(Q)$となる.よって,ある$q \in Q$に対して,$p \preceq q$となる.

$n \ge 0$個の変数記号を含む任意の正規パターンに対して題意が成り立つと仮定する.
$p$を$S_{2}(p) \subseteq L(Q)$を満たす$n+1$個の変数記号を含む正規パターンとする.
$p \not \preceq q_{i}$ ($i=1, \ldots, k$)と仮定する.
$p=p_{1}xp_{2}$ ($p_{1},p_{2}$は正規パターン,$x$は変数記号),$Q=\{ q_{1}, \ldots , q_{k} \}$を考える.
$a, b \in \Sigma$に対して,$p_{a}=p \{ x := a \}$と$p_{ab}=p \{ x := ab \}$とおく.
このとき,$p_{a},p_{ab}$は$n$個の変数記号が含まれ,$S_{2}(p_{a}) \subseteq L(Q)$かつ$S_{2}(p_{ab}) \subseteq L(Q)$が成り立つことに注意する.
帰納法の仮定より,任意の$a,b \in \Sigma$に対して,$p_{a} \preceq q_{i}$かつ$p_{ab} \preceq q_{i^{\prime}}$を満たすような$i, i^{\prime} \le k$が存在する. 
$D_{i}=\{ a \in \Sigma \mid p \{ x:=a \} \preceq q_{i} \}$ \ ($i=1, \ldots, k$)とする.
ある$i$に対して,$\sharp D_{i} \ge 3$であるとき,補題\ref{補題10}より,$p \preceq q_{i}$となる.これは仮定に矛盾する.
よって,$\sharp D_{i} \le 2$ ($i=1, \ldots, k$)となる場合を考える.
$\sharp\Sigma = 2k-1$のとき,任意の$i$に対して,$\sharp D_{i}=2$または$\sharp D_{i}=1$,$\sharp\Sigma = 2k$のとき,任意の$i$に対して,$\sharp D_{i}=2$となる.
$k \ge 3$であるとき,$2k+1 \ge k+2$となる.
よって,補題\ref{追加補題1}より,$p \{ x:=xy \} \preceq q_{i}$となる$i$が存在する.
したがって,補題\ref{補題15}より,$p \preceq q_{i}$となる.
これは仮定に矛盾する.
  
以上より,(i) $\Rightarrow$ (ii)が成り立つ.
\end{proof}

この定理\ref{定理17}より,次の系が得られる.

\begin{col}\label{命題18}
$k \ge 3$,$\sharp\Sigma \ge 2k-1$,$P \in \mathcal{RP}^{+}$とする.このとき,$S_{2}(P)$は$\mathcal{RPL}^{k}$における$L(P)$の特徴集合である.
\end{col}

\begin{lem}[Sato et al.\cite{Sato1}]\label{補題19}
$\sharp\Sigma \le 2k-2$とする.このとき,$\mathcal{RP}^{k}$は包含に関するコンパクト性を持たない.
\end{lem}

\begin{proof}
$\Sigma = \{ a_{1}, \ldots , a_{k-1}, b_{1}, \ldots , b_{k-1} \}$を$(2k-2)$個の定数記号から成る集合,$p, q_{i}$を正規パターン,$w_{i}~(i = 1, \ldots , k-1)$を例\ref{例題1}と同様に定義された記号列とする.
$q_{k} = x_{1}a_{1}w_{1}xyw_{1}b_{1}x_{2}$とする.
例\ref{例題1}で示した通り,$p \{ x := a_{i} \} \preceq q_{i}$かつ$p \{ x := b_{i} \} \preceq q_{i}~(i=1,2, \ldots ,k-1)$であるとき,$S_{1}(p) \subseteq \bigcup^{k-1}_{i=1} L(q_{i})$となる. 
一方で,任意の$w$ $(|w| \ge 2)$に対して,$p \{ x:= w \} \preceq q_{k}$となる. 
すなわち,$L(p) \subseteq L(Q)$である.
しかし,$p \not \preceq q_{i}$であるため,$L(p) \not \subseteq L(q_{i}) (i=1,2, \ldots k)$である.
したがって,$\RPatkei$は包含に関するコンパクト性を持たない.
\end{proof}

定理\ref{定理17}と補題\ref{補題19}より,次の定理が成り立つ.

\begin{thm}
$k \ge 3$とし,$\sharp\Sigma \ge 2k-1$とする.
このとき,$\RPat^{k}$は包含に関してコンパクト性を持つ.
\end{thm}

$k=2$のとき,次の定理が成り立つ.

\begin{thm}\label{補題21}
$\sharp \Sigma \ge 4$とし,$P \in \RPatplus$,$Q \in \RPat^{2}$とする.
このとき,以下の{\rm (i),(ii),(iii)}は同値である.
\[
\begin{tabular}{ll}
$(\mathrm{i})$ $S_{2}(P) \subseteq L(Q)$,
$(\mathrm{ii})$ $P \sqsubseteq Q$,
$(\mathrm{iii})$ $L(P) \subseteq L(Q)$.
\end{tabular}
\]
\end{thm}

\begin{proof}
(ii) $\Rightarrow$ (iii)と(iii) $\Rightarrow$ (i)は自明に成り立つ.
よって,(i) $\Rightarrow$ (ii)が成り立つことを示す.
$Q= \{ q_{1}, q_{2} \}$とするとき,$p$に含まれる変数記号の数$n$に関する数学的帰納法で示す.\\
\noindent (1) $n=0$のとき,$p$は定数記号列となるので$S_{2}(p)= \{ p \}$となり
(i)より,$p \in L(Q)$となる.
よって,ある$q \in Q$に対して$p \preceq q$となる.\\
\noindent (2) $n=k$個の変数記号を含むすべての正規パターンに対して有効であると仮定する.
そして,$p$を$S_{2}(p) \subseteq L(Q)$を満たす$(n+1)$個の変数記号を含む正規パターンとする.

$p \not \preceq q_{i}$ ($i=1, 2$)と仮定する.
$p=p_{1}xp_{2}$ ($p_{1}, p_{2}$は正規パターン,$x$は変数記号)を考える.
$a, b \in \Sigma$に対して,$p_{a}=p \{ x := a \}$,$p_{ab}=p \{ x := ab \}$とおく.
このとき,$p_{a},p_{ab}$は$n$個の変数記号が含まれ,$S_{2}(p_{a}) \subseteq L(Q)$,$S_{2}(p_{ab}) \subseteq L(Q)$が成り立つことに注意する.
帰納法の仮定より,任意の$a, b \in \Sigma$に対して,$p_{a} \preceq q_{i}, p_{ab} \preceq q_{i^{\prime}}$を満たすような$i, i^{\prime} \le k$が存在する.

ある$i$に対して$\sharp D_{i} \ge 3$のとき,補題\ref{補題10}より,$p \preceq q_{i}$となる.
よって,任意の$i$に対して,$\sharp D_{i} \le 2$となる.
したがって,$\sharp D_{1}=2$かつ$\sharp D_{2}=2$となる場合を考える.

$\sharp \Sigma = k+2$であるとき,$k=2$より,$\sharp \Sigma =4$となる.
よって,補題\ref{追加補題1}より,ある$i$に対して,$p \{ x:=xy \} \preceq q_{i}$となる.
したがって,補題\ref{補題15}より,$p \preceq q_{i}$となる.
これは,仮定に矛盾する.

以上より,(i) $\Rightarrow$ (ii)が成り立つ.
\end{proof}

次の例は,$k = 2$における定理\ref{補題21}の反例である.
\begin{ex}\label{反例thm17}
$\Sigma= \{a, b, c \}$を$3$つの定数記号から成る集合,$p,q_{1},q_{2}$を正規パターン,$x,x^{\prime},x^{\prime\prime}$を変数記号とする.
\begin{eqnarray*}
p = x^{\prime}axbx^{\prime\prime},
q_{1} = x^{\prime}abx^{\prime\prime},
q_{2} = x^{\prime}cx^{\prime\prime}.
\end{eqnarray*}
$w \in \Sigma^{+}$とする.$w$に$c$が含まれるとき,$p \{ x:=w \} \preceq q_{2}$となり,$c$が含まれないとき,$p \{ x:=w \} \preceq q_{1}$となる.
よって,$L(p) \subseteq L(q_{1}) \cup L(q_{2})$である.
しかし,$p \not \preceq q_{1}$かつ$p \not \preceq q_{2}$である.
\end{ex}

定理\ref{補題21}より,次の2つの系が成り立つ.
\begin{col}
$\sharp\Sigma \ge 4$とし,$P \in \RPatplus$とする.
このとき,$S_{2}(P)$は,$\RPatL^{2}$における$L(P)$の特徴集合である.
\end{col}

\begin{col}
$\sharp\Sigma \ge 4$とする.このとき,クラス$\mathcal{RP}^{2}$は包含に関してコンパクト性を持つ.
\end{col}
