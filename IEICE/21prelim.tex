\section{Preliminaries}

% Preliminaries
Let $\Sigma$ be a nonempty finite set of constant symbols and
$X$ an infinite set of variable symbols such that $\Sigma \cap X = \emptyset$.
%$\Sigma$を有限アルファベットとし,$X$を$\Sigma \cap X=\emptyset$を満たす可算無限集合とする.
%$\Sigma$と$X$の要素をそれぞれ定数記号と変数記号という.
%%$\Sigma$を少なくとも2つの記号を含む定数記号の有限集合, $X$を変数記号からなる可算集合とする.
%%ただし, $\Sigma \cap X = \phi$とする.
A pattern is a finite string which consists of symbols in $\Sigma \cup X$.
Note that the empty string $\varepsilon$ is a regular pattern in this paper.
A pattern $p$ is said to be \textit{regular} if each variable symbol appears
at most one in $p$.
It is clear that $\varepsilon$ is a regular pattern.
The length of $p$, denote by $|p|$, is the number of symbols in $p$.
%$\Sigma$と$X$の記号から成る記号列を\textbf{パターン}という.
%また,各変数記号が高々1回しか現れないパターンを\textbf{正規パターン}という.
%パターン$p$の長さ,つまりその記号列の長さを$|p|$で表す.
The set of all patterns and regular patterns are denoted by $\Pat$ and $\RPat$ respectively.
%Note that the empty string $\varepsilon$ is a regular pattern in this paper.
%Thus, we have $\Pat = (\Sigma \cup X)^{\ast}$ and $\RPat = (\Sigma \cup X)^{+}$.
%すべてのパターンの集合とすべての正規パターンの集合をそれぞれ$\Pat$と$\RPat$で表す.便宜上,空記号列$\varepsilon$もパターンとしていることに注意する.
%つまり,$\Pat=(\Sigma \cup X)^{*}$であり,$\Pat\setminus\{\varepsilon\}=(\Sigma \cup X)^{+}$である.
%集合$A$の要素数を$\sharp A$で表す.
For a set $S$, we denote by $\sharp S$ the number of elements in $S$.
In this paper, we assume $\sharp \Sigma \geq 2$.
%本稿では,$\sharp \Sigma \geq 2$と仮定する.
%$\Pat$の要素を$p,q,\ldots,p_1,p_2,\ldots,$で表す.
%
%%\begin{comment}
%%空文字列を$\varepsilon$で表す.
%%$\Sigma$の要素数を$\sharp\Sigma$で表す.
%%$\Sigma \cup X$上の全ての文字列を$(\Sigma \cup X)^{\ast}$で表し, $\varepsilon$を除く$\Sigma \cup X$上の全ての文字列を$(\Sigma \cup X)^{+}$で表す.
%%\end{comment}
%
%%\textbf{パターン}とは$(\Sigma \cup X)^{\ast}$に含まれる文字列である.
%%便宜上,空文字列$\varepsilon$をパターンとして考える.
%%全てのパターンの族を$\mathcal{P}$と表し, パターン$p$の文字列の長さを$|p|$で表す.
%
A substitution $\theta$ is a mapping from $(\Sigma \cup X)^{\ast}$ to $(\Sigma \cup X)^{\ast}$ such that
(1) $\theta$ is a homomorphism with respect to string concatenation, that is, for patterns $\pi_{1}$ and $\pi_{2}$,
$\theta(\pi_{1} \cdot \pi_{2}) = \theta(\pi_{1}) \cdot \theta(\pi_{2})$ holds,
(2) for each constant symbol $a \in \Sigma$, $\theta(a) = a$ holds,
and (3) for each variable symbol $x \in X$, $|\theta(x)| \geq 1$ holds.
Let $x_{1},\ldots,x_{n}$ are variable symbols and $p_{1},\ldots,p_{n}$ nonempty patterns.
We denote by $\{x_{1}:=p_{1},\ldots,x_{n}:=p_{n}\}$ a substitution that replaces each variable symbol $x_{i}$
with a nonempty pattern $p_{i}$ for $i \in \{1,\ldots,n\}$ respectively.
We denote by $\theta(\pi)$ a new pattern obtained by $\theta$ for a pattern $\pi$.
%
%与えられたパターンの変数記号に長さ1以上のパターンを代入することで,別のパターンを生成することができる.
%ただし,同じ変数記号には同じパターンを代入し,空記号列$\varepsilon$は代入しないこととする.
%パターン$p \in \Pat$に対し,$p$中の各変数記号$x_i~(i=1,2,\ldots,k)$にそれぞれパターン$q_i$を代入することを$\theta=\{x_1:=q_1,x_2:=q_2,\ldots,x_k:=q_k\}$で表すこととし,
%このような代入の操作を$p$に施した結果のパターンを$p\theta$で表す.
%便宜上,$\theta$を代入と呼ぶ.
%%\textbf{代入}\bm{$\theta$}とは,全ての定数をそれ自身に移すパターンからパターンへの準同型写像をいう.
%%代入$\theta$によるパターン$q$の像を$q \theta$で表す.
For a pattern $p$ and $q$,
$q$ is a \textit{generalization} of $p$, or $p$ is an \textit{instance} of $q$, denoted by $p \preceq p$,
if there exists a substitution $\theta$ sudh that $p = q\theta$.
We say that $p$ is equal to $q$, denoted by $p \equiv q$, if $p \preceq q$ and $p \succeq p$.
%For a pattern $p$, the \textit{pattern language} of $p$, denoted by $L(p)$, is the set of all strings $w$ in $\Sigma^{+}$
%such that $w$ is obtained from $p$ by replacing all variable symbols in $p$ with nonempty constant symbols, that is,
%such that $w$ is obtained from $p$ by replacing all variable symbols in $p$ with nonempty constant symbols, that is,
%
%$q$が$p$の汎化,あるいは$p$が$q$の例化であるとは,
%$p=q \theta$を満たす代入$\theta$が存在するときをいい,
%\bm{$p \preceq q$}で表す.
%また,$p \preceq q$かつ$q \preceq p$であるとき,$p$と$q$は等価であるといい, \bm{$p \equiv q$}で表す.
%
For a pattern $p$, the \textit{pattern language} of $p$, denoted by $L(p)$, is the set $\{w \in \Sigma^{+} \mid w \preceq p\}$.
In particular, if $p$ is a regular pattern, we say that $L(p)$ is a \textit{regular pattern language}.
The set of all pattern languages and regular patterns languages are denoted by $\PatL$ and $\RPatL$ respectively.
For patterns $p$ and $q$, it is clear that $L(p) = L(q)$ if $p \equiv q$, and $L(p) \subseteq L(q)$ if $p \preceq q$.
% In general, it is not valid that 
Note that $L(\varepsilon) = \emptyset$.
%パターン$p$に対し,$p$が表す言語($\Sigma^{*}$の部分集合)を,$p$に代入を施すことにより生成できる定数記号列の集合$L(p)$,つまり,
%$L(p)=\{w\in \Sigma^{+} \mid w \preceq p\}$
%と定義する.
%ここで,$p \equiv q$ならば$L(p)=L(q)$であることに注意する.
%パターンおよび正規パターンによって生成される言語をそれぞれパターン言語および正規パターン言語という.
%また,すべてのパターン言語の集合および正規パターン言語の集合をそれぞれ$\PatL$および$\RPatL$で表す.
%%$\Sigma$上の言語$L$は, $L=L(p)$を満たすパターン$p$が存在するとき, \textbf{パターン言語}といい, 全てのパターン言語の族を$\mathcal{PL}$で表す.
%%正規パターンのパターン言語を\textbf{正規パターン言語}といい, 全ての正規パターンの族を$\mathcal{RP}$, 全ての正規パターン言語の族を$\mathcal{RPL}$で表す.
%%正規パターン$p, q$に対して, $p \preceq q$ならば$L(p) \subseteq L(q)$となる.この逆は一般に成立しない.
%%正規パターンに関して,次のような結果が得られている.
%正規パターンについては,次の補題が成り立つ.

\begin{lem}[Mukouchi\cite{Mukouchi1991}]\label{regularPatternEquivalence}
  Let $p$ and $q$ be regular patterns.
  Then $p \preceq q$ if and only if $L(p) \subseteq L(q)$.
\end{lem}

% \begin{lem}[Mukouchi\cite{Mukouchi1991}]\label{Mukouchi:Lemma1}
%   Let $p$ and $q$ be regular patterns. 
%   Then $p \preceq L(p)$ if and only if $L(p) = L(q)$.
% \end{lem}

%%\begin{comment}
%\begin{lem}[Mukouchi\cite{Mukouchi1991}]\label{補題1}
%$\sharp \Sigma \geq 3$とする.
%%$p, q$を正規パターンとする.%このとき,
%  任意の正規パターン$p,q \in \RPat$に対して,
%$p \preceq q$ならばその時に限り$L(p) \subseteq L(q)$である.
%\end{lem}
%%\end{comment}
Next, we consider unions of pattern languages. % or regular pattern languages.
The class of all nonempty finite subsets of $\Pat$ is denoted by $\Patplus$, that is,
$\Patplus = \{P \subseteq \Pat \mid 0 < |P| < \infty\}$.
For a positive integer $k$, the class of nonempty sets consisting of at most $k$ patterns, that is,
$\Patkei = \{P \subseteq \Pat \mid 0 < |P| \leq k\}$.
We denote by $\PatLkei$ the class of unions of at most $k$ pattern languages,
that is, $\PatLkei = \{L(P) \mid P \in \Patkei\}$,
where $L(P) = \cup_{p \in P}L(p)$.
In a similar way, we also define $\RPatplus$, $\RPatkei$ and $\RPatLkei$ respectively.
%%次にパターン言語の有限和を考える.
%$\Pat$の空でない有限部分集合の集合を$\Patplus$で,
%高々$k~(k\geq 1)$個のパターンから成る$\Pat$の部分集合$\{P\in \Patplus \mid \sharp P \leq k\}$を$\Pat^{k}$で表す.
%また,高々$k~(k\geq 1)$個のパターン集合$P\in \Pat^{k}$に対して,$P$が表すパターン言語$\bigcup_{p\in P}L(p)$を$L(P)$で,
%$\Pat^{k}$に属するパターン集合が表すパターン言語のクラス$\{ L(P) \ | \ P \in \Pat^{k} \}$を$\PatL^{k}$で表す.
%同様に,$\RPat$の空でない有限部分集合の集合を$\RPatplus$で,
%高々$k~(k\geq 1)$個のパターンから成る$\RPat$の部分集合$\{P \in \RPatplus \mid \sharp P \leq k\}$を$\RPat^{k}$,
%$\RPat^{k}$に属するパターン集合が表すパターン言語のクラス$\{L(P) \mid P\in \RPat^{k}\}$を$\RPatL^{k}$で表す. 
%%\begin{dfn}\label{bi}
%
For $P$, $Q$ in $\Patplus$,
we denote by $P \sqsubseteq Q$ if for any $p \in \Pat$, there is a pattern $q \in Q$ such that $p \preceq q$.
It is clear that $P \sqsubseteq Q$ implies $L(P) \subseteq L(Q)$.
However, note that the converse is not valid in general.

%$P, Q$を$\Patplus$に属するパターン集合とする.
%このとき,任意のパターン$p \in P$に対して,あるパターン$q\in Q$が存在し,
%$p\preceq q$が成り立つとき$P \sqsubseteq Q$と書く.
%%\end{dfn}
%%全ての$p \in P$に対して,$p \preceq q$を満たす$q \in Q$が存在するとき,二項関係$P \sqsubseteq Q$が成り立つ.
%%
%このとき,$P \sqsubseteq Q$ならば$L(P) \subseteq L(Q)$である.
%なお,一般にこの逆は成り立たないことに注意する.