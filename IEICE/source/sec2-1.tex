\section{準備}
$\Sigma$を有限アルファベットとし,$X$を$\Sigma \cap X=\emptyset$を満たす可算無限集合とする.
$\Sigma$と$X$の要素をそれぞれ定数記号と変数記号という.
%$\Sigma$を少なくとも2つの記号を含む定数記号の有限集合, $X$を変数記号からなる可算集合とする.
%ただし, $\Sigma \cap X = \phi$とする.
$\Sigma$と$X$の記号から成る記号列を\textbf{パターン}という.
また,各変数記号が高々1回しか現れないパターンを\textbf{正規パターン}という.
パターン$p$の長さ,つまりその記号列の長さを$|p|$で表す.
すべてのパターンの集合とすべての正規パターンの集合をそれぞれ$\Pat$と$\RPat$で表す.便宜上,空記号列$\varepsilon$もパターンとしていることに注意する.
つまり,$\Pat=(\Sigma \cup X)^{*}$であり,$\Pat\setminus\{\varepsilon\}=(\Sigma \cup X)^{+}$である.
集合$A$の要素数を$\sharp A$で表す.
本稿では,$\sharp \Sigma \geq 2$と仮定する.
$\Pat$の要素を$p,q,\ldots,p_1,p_2,\ldots,$で表す.

%\begin{comment}
%空文字列を$\varepsilon$で表す.
%$\Sigma$の要素数を$\sharp\Sigma$で表す.
%$\Sigma \cup X$上の全ての文字列を$(\Sigma \cup X)^{\ast}$で表し, $\varepsilon$を除く$\Sigma \cup X$上の全ての文字列を$(\Sigma \cup X)^{+}$で表す.
%\end{comment}

%\textbf{パターン}とは$(\Sigma \cup X)^{\ast}$に含まれる文字列である.
%便宜上,空文字列$\varepsilon$をパターンとして考える.
%全てのパターンの族を$\mathcal{P}$と表し, パターン$p$の文字列の長さを$|p|$で表す.

与えられたパターンの変数記号に長さ1以上のパターンを代入することで,別のパターンを生成することができる.
ただし,同じ変数記号には同じパターンを代入し,空記号列$\varepsilon$は代入しないこととする.
パターン$p \in \Pat$に対し,$p$中の各変数記号$x_i~(i=1,2,\ldots,k)$にそれぞれパターン$q_i$を代入することを$\theta=\{x_1:=q_1,x_2:=q_2,\ldots,x_k:=q_k\}$で表すこととし,
このような代入の操作を$p$に施した結果のパターンを$p\theta$で表す.
便宜上,$\theta$を代入と呼ぶ.
%\textbf{代入}\bm{$\theta$}とは,全ての定数をそれ自身に移すパターンからパターンへの準同型写像をいう.
%代入$\theta$によるパターン$q$の像を$q \theta$で表す.
$q$が$p$の汎化,あるいは$p$が$q$の例化であるとは,
$p=q \theta$を満たす代入$\theta$が存在するときをいい,
\bm{$p \preceq q$}で表す.
また,$p \preceq q$かつ$q \preceq p$であるとき,$p$と$q$は等価であるといい, \bm{$p \equiv q$}で表す.

パターン$p$に対し,$p$が表す言語($\Sigma^{*}$の部分集合)を,$p$に代入を施すことにより生成できる定数記号列の集合$L(p)$,つまり,
$L(p)=\{w\in \Sigma^{+} \mid w \preceq p\}$
と定義する.
ここで,$p \equiv q$ならば$L(p)=L(q)$であることに注意する.
パターンおよび正規パターンによって生成される言語をそれぞれパターン言語および正規パターン言語という.
また,すべてのパターン言語の集合および正規パターン言語の集合をそれぞれ$\PatL$および$\RPatL$で表す.
%$\Sigma$上の言語$L$は, $L=L(p)$を満たすパターン$p$が存在するとき, \textbf{パターン言語}といい, 全てのパターン言語の族を$\mathcal{PL}$で表す.
%正規パターンのパターン言語を\textbf{正規パターン言語}といい, 全ての正規パターンの族を$\mathcal{RP}$, 全ての正規パターン言語の族を$\mathcal{RPL}$で表す.
%正規パターン$p, q$に対して, $p \preceq q$ならば$L(p) \subseteq L(q)$となる.この逆は一般に成立しない.
%正規パターンに関して,次のような結果が得られている.
正規パターンについては,次の補題が成り立つ.
%\begin{comment}
\begin{lem}[Mukouchi\cite{Mukouchi1991}]\label{補題1}
  $\sharp \Sigma \geq 3$とする.
  %$p, q$を正規パターンとする.%このとき,
  任意の正規パターン$p,q \in \RPat$に対して,
$p \preceq q$ならばその時に限り$L(p) \subseteq L(q)$である.
\end{lem}
%\end{comment}
%次にパターン言語の有限和を考える.
$\Pat$の空でない有限部分集合の集合を$\Patplus$で,
高々$k~(k\geq 1)$個のパターンから成る$\Pat$の部分集合$\{P\in \Patplus \mid \sharp P \leq k\}$を$\Pat^{k}$で表す.
また,高々$k~(k\geq 1)$個のパターン集合$P\in \Pat^{k}$に対して,$P$が表すパターン言語$\bigcup_{p\in P}L(p)$を$L(P)$で,
$\Pat^{k}$に属するパターン集合が表すパターン言語のクラス$\{ L(P) \ | \ P \in \Pat^{k} \}$を$\PatL^{k}$で表す.
同様に,$\RPat$の空でない有限部分集合の集合を$\RPatplus$で,
高々$k~(k\geq 1)$個のパターンから成る$\RPat$の部分集合$\{P \in \RPatplus \mid \sharp P \leq k\}$を$\RPat^{k}$,
$\RPat^{k}$に属するパターン集合が表すパターン言語のクラス$\{L(P) \mid P\in \RPat^{k}\}$を$\RPatL^{k}$で表す. 
%\begin{dfn}\label{bi}
$P, Q$を$\Patplus$に属するパターン集合とする.
このとき,任意のパターン$p \in P$に対して,あるパターン$q\in Q$が存在し,
$p\preceq q$が成り立つとき$P \sqsubseteq Q$と書く.
%\end{dfn}
%全ての$p \in P$に対して,$p \preceq q$を満たす$q \in Q$が存在するとき,二項関係$P \sqsubseteq Q$が成り立つ.
%
このとき,$P \sqsubseteq Q$ならば$L(P) \subseteq L(Q)$である.
なお,一般にこの逆は成り立たないことに注意する.