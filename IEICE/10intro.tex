\section{Introduction}

%intro 
A pattern is a string consisting of constant symbols and variable symbols.
For example, we consider constant symbols $a,b,c$ and variable symbols $axbxcy$,  then $axbxcy$ is a pattern.
$\Pat$ denotes the set of all patterns.
For a pattern $p\in \Pat$, {\it the pattern language generated by $p$} or a pattern language is the set of all strings obtained by replacing all variables symbols with constant strings, where the same variables symbol is replaced by the same constant string.
For example the pattern language $L(axbxcy)$ generated by the above pattern $axbxcy$ denotes $\{ aubucw \mid \mbox{  u and w are constant strings that are not $\varepsilon$} \}$
A pattern where each variable symbol appears at most one is called a {\it regular pattern}. 
For example, a pattern $axbxcy$ is not a regular pattern, but a pattern $axbzcy$ with variable symbols $x,y,z$ is a regular pattern.
$\RPat$ denotes the set of regular patterns.
If a pattern $p\in \Pat$ is obtained from a pattern $q\in\Pat$ by replacing variable symbols in $q$ with patterns, we say that $q$ is a {\it generalization} of $p$ and denoted by $p\preceq q$.
For example, a pattern $q=axz$ is a generalization of a pattern $p=axbxcy$, because $p$ is obtained from $q$ by replacing the variable $z$ in $q$ by a pattern $bxcy$.
So we write $p\preceq q$.
For patterns $p,q\in \Pat$, it is obvious that $p\preceq q$ implies $L(p)\subseteq L(q)$.
But, the converse, that is, the statement $L(p)\subseteq L(q)$ implies $p\preceq q$ does note always hold.
Contrary to this statement, Mukouchi\cite{Mukouchi1991} showed that it holds that if the number of constant symbols is greater than 3, for any regular pattern $p,q\in \RPat$ $L(p)\subseteq L(q)$ implies $p\preceq q$.

% memo 240618-1630

$\RPatplus$を$\RPat$の空でない有限集合の集合とする.
$\RPatkei$を高々$k~(k\geq 2)$個の正規パターンから成る集合の全体のクラスとする.
正規パターンの集合$P\in\RPatkei$に対し,$L(P)=\bigcup_{p\in P}L(p)$とし,
$\RPatkei$に対する正規パターン言語のクラス$\{L(P) \mid P\in \RPatkei\}$を$\RPatL^{k}$とする.
$P,Q\in \RPatkei$とし,$Q= \{ q_{1}, \ldots, q_{k} \}$とする.
任意の正規パターン$p\in P$に対し,ある正規パターン$q_{i}$が存在し,$p\preceq q_{i}$が成り立つとき
$P\sqsubseteq Q$と書く.
定義より,$P \sqsubseteq Q$ならば$L(P)\subseteq L(Q)$であることは明らかである.
そこで,Satoら\cite{Sato1}は,$k \ge 3$であり定数記号の数が$2k-1$であるとき,
各変数記号に対し長さが高々2の定数記号列を代入することで$P\in \RPatkei$から得られる定数記号列の有限集合$S_2(P)$が
$L(P)$の特徴集合であること,つまり任意の正規パターン言語$L'\in \RPatL^{k}$に対して,$S_2(P) \subseteq L'$ならば$L(P)\subseteq L'$となることを
示し,$(i) S_2(P) \subseteq L(Q)$,(ii) $P\sqsubseteq Q$ および (iii) $L(P)\subseteq L(Q)$が同値であることを示した.
しかし,この結果の根拠となる補題14\cite{Sato1}に誤りがあるため,
本稿では,まずその修正を行い,Satoらが示した3つの命題の同値性の正しい証明を与えた.
Satoら\cite{Sato1}は,定数記号の数が$2k-1$以上のとき,$\RPatkei$が包含に関してコンパクト性を持つことも示した.
これに対し,本稿では,隣接した変数記号(隣接変数)を持たない正規パターンである非隣接変数正規パターン全体の集合$\NAVRP$を与え,
高々$k~(k\ge 1)$個の非隣接変数正規パターンの集合全体のクラス$\NAVRPkei$に属する集合$P$から得られる$S_2(P)$が$L(P)$の特徴集合であることを示した.
さらに,定数記号の数が$k+2$以上のとき,$\NAVRPkei$が包含に関してコンパクト性を持つことを示した.
表\ref{表1}に本稿の結果をまとめて示す.
\begin{table}
\begin{center}
\vspace{-0.4cm}
\caption{包含に関してコンパクト性を持つための定数記号の数に関する条件}
\label{表1}
\begin{tabular}{llll}
\hline
\multicolumn{1}{|c|}{$k$}   & \multicolumn{1}{c|}{2}                            & \multicolumn{1}{c|}{3以上} \\ 
\hline
\hline
\multicolumn{1}{|c|}{$\RPatkei$} & \multicolumn{1}{c|}{4以上}                          & \multicolumn{1}{c|}{$2k-1$以上} \\ \hline
\multicolumn{1}{|c|}{$\NAVRPkei$} & \multicolumn{2}{c|}{$k+2$以上} \\ \hline
%\vspace{-1cm}
\end{tabular}
\end{center}
\end{table}

%これは,正規パターン言語のときに下界を示すために用いられた関数$2k-1$より正確な関数表現$k+2$を与えることができ,
本稿の結果は,言語の有限和の表現である正規パターンの集合あるいは非隣接変数正規パターンの集合を対象とした効率的な学習アルゴリズムをそれぞれ与えられることを示唆している.
%その定数記号の数の条件で,非隣接変数正規パターン言語の有限和に関する効率的な学習アルゴリズムが設計できることを示した.

本稿の構成は以下の通りである.
第2節では,準備としてパターン言語,正規パターン言語,コンパクト性などの定義を与え,さらに$\RPatplus$の特徴集合に関するSatoらの結果を紹介する.
第3節では,$S_{2}(P)$は$\RPatL^{k}$における$L(P)$の特徴集合であること,
および$\RPat^{k}$が包含に関するコンパクト性を持つことを示す.
第4節では,非隣接変数正規パターンを与え,$\NAVRPkei$に属する集合$P$から得られる$S_2(P)$が
$L(P)$の特徴集合であること,および$\NAVRPkei$が包含に関してコンパクト性をもつことを示す.
