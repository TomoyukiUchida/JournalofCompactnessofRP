\section{Regular Pattern without Adjacent Variable Symbols}

A regular pattern $p$ is said to be a {\it non-adjacent variable regular pattern} ($\NAV$ regular pattern)  
if $p$ does not contain consecutive variable symbols.
For example, the regular pattern $p=axybc$ is not a $\NAV$ regular pattern because $xy$ is appeared in $p$.
Let $\NAVRP$ be the set of all $\NAV$ regular patterns.
Let $\NAVRPplus$ be the set of all finite subsets $S$ of $\NAVRP$ such that $S$ is not the empty set, i.e., $\NAVRPplus=\{S \subseteq \NAVRP \mid \sharp S \leq 1\}$,
and $\NAVRPkei$ the set of all subsets $P$ of $\NAVRPplus$ such that $P$ consists of at most $k~(k\geq 1)$ $\NAV$ regular patterns, i.e., $\NAVRPkei=\{P\in \NAVRPplus \mid \sharp P \leq k\}$.
We can define the compactness with respect to containment for $\NAVRPkei$ in a similar way as Def.\ref{def:compactness}.
 For any $\NAV$ regular pattern $p \in \NAVRP$ and any set $Q \in \NAVRPkei$ with $k~(k\geq 1)$,
  the set $\NAVRPkei$ said to have {\it compactness with respect to containment}
  if there exists a $\NAV$ regular pattern $q \in Q$ such that $L(p) \subseteq L(q)$ holds if $L(p) \subseteq L(Q)$ holds.
Then, we have the following Theorem \ref{KeyTheoforNAVRP}.

\begin{thm}\label{KeyTheoforNAVRP}%\label{非隣接kが4以上}
%For $k~(k\ge 2)$, \sharp \Sigma \ge k+2,~P\in \NAVRPplus,~Q \in \NAVRPkei$とする.
For an integer $k~(k\ge 2)$, let $\sharp \Sigma \ge k+2,~P\in \NAVRPplus,~Q \in \NAVRPkei$.  
Then, the following (i), (ii) and (iii) are equivalent:
\[
\begin{tabular}{ll}
(i) $S_{2}(P) \subseteq L(Q)$,
(ii) $P \sqsubseteq Q$,
(iii) $L(P) \subseteq L(Q)$.
\end{tabular}
\]
\end{thm}

\begin{proof}
From the definitions of $\NAVRPplus$ and $\NAVRPkei$, it is clear that (ii) implies (iii)  and  (iii) implies (i).
Hence, we will show that (i) implies (ii) 
by mathematical induction on the number $n$ of variable symbols that appear in a $\NAV$ regular pattern $p\in P$ as follows:
If $n=0$, then we have $S_{2}(\{p\})= \{ p \}$.
Hence, $p \in L(Q)$.
Therefore, there exists $q \in Q$ such that $p \preceq q$ holds.

If $n \ge 0$, we assume that the proposition holds for any regular $\NAV$ regular pattern containing $n \ge 0$ variable symbols.
Let $p$ be a $\NAV$ regular pattern containing $n+1$ variable symbols such that $S_{2}(\{p\}) \subseteq L(Q)$ holds and $p$ contains a variable symbol $x$.
There exist two $\NAV$ regular patterns $p_{1},p_{2}$ such that $p=p_{1}xp_{2}$ holds.
By the induction hypothesis, for any constant string $w\in \Sigma^{\ast}$ with $|w|=2$, $\{p\{x:=w\}\}\preceq Q$ holds because $p\{x:=w\}$ contains $n$ variable symbols.
Hence, there exists a $\NAV$ regular pattern $q_{w} \in Q$ such that $p \{ x:=w \} \preceq q_{w}$ holds.
From Lemma \ref{追加補題1}, there exists a regular pattern $q \in Q$ such that $p \{ x:=xy \} \preceq q$ holds, where $y$ is a variable symbol that does not appear in $q$.
This contradicts the condition $Q \in \NAVRPkei$.
Thus, we have that (i) implies (ii).
\end{proof}

\begin{col}
Let $k\ge 2$, $\sharp\Sigma \ge k+2$ and $P \in \NAVRPplus$.
Then, $S_{2}(P)$ is a characteristic set for $\NAVRPLkei$.%$\mathcal{RPL^{\mbox{$k$}}_{NAV}}$.
\end{col}

\begin{lem}\label{Case_k+2}\label{k+2のとき}
Let $k\ge 2$ and $\sharp\Sigma \le k+1$.
Then, $\NAVRPkei$ does not have compactness with respect to containment.
\end{lem}
\begin{proof}
Let $\Sigma$ be the set of $k+1$ constant symbols $a_{1}, \ldots , a_{k+1}$, i.e., $\Sigma = \{ a_{1}, \ldots , a_{k+1} \}$.
We assume that for $i=1,2,\ldots,k$, $p \{ x := a_{i}y \} \preceq q_{i}$ and $p \{ x := ya_{i+1} \} \preceq q_{i}~(i=1,2, \ldots ,k)$ hold.
If $p \{ x:= a_{k+1}a_{1} \} \preceq q_{1}$ holds, $S_{2}(p) \backslash S_{1}(p) \subseteq \bigcup^{k}_{i=1} L(q_{i})$ holds.
This show that $L(p) \subseteq L(Q)$ holds.
However, for $i=1,2,\ldots,k$, since $p \not \preceq q_{i}$ holds, we have that $L(p) \not \subseteq L(q_{i})$ holds.
Hence, $\NAVRPkei$ does not have compactness with respect to containment.
\end{proof}

Next, we give an example for Lemma \label{Case_k+2} in Example \ref{Case_k+1}.
\begin{ex}\label{Case_k+1}\label{反例k+1}
Let $\Sigma$ be the set of four constant symbols $a,~b,~c,~d$, i.e., $\Sigma= \{a, b, c, d \}$ and $x,x^{\prime},x^{\prime\prime}$ three distinct variable symbols.
Let $p,q_{1},q_{2},q_{3}$ be the $\NAV$ regular patterns given in Fig. \ref{Fig:CounterExampleforNAVR}. 
\begin{figure}[tb]
%\begin{align*}
  \begin{tabular}{l}
$p  = x^{\prime}cadadaadacbadadaadaxadadaadacbadadaadabx^{\prime\prime}$,\\
$q_{1} = x^{\prime}cadadaadacbadadaadacx^{\prime\prime}$,\\
$q_{2} = x^{\prime}badadaadacx^{\prime\prime}$,\\
$q_{3} = x^{\prime}aadadx^{\prime\prime}$.
  \end{tabular}
%\end{align*}
\caption{$\NAV$ regular patterns $p$, $q_{1}$, $q_{2}$, and $q_{3}$}\label{Fig:CounterExampleforNAVR}
\end{figure}
\noindent
Then, we have  $L(p) \subseteq L(q_{1}) \cup L(q_{2}) \cup L(q_{3})$.
This show that for $P=\{p\},~Q=\{q_{1},q_{2},q_{3}\}$, (iii) of Theorem \ref{KeyTheoforNAVRP} holds.
However, since $p \not \preceq q_{1},~p \not \preceq q_{2}$ and $p \not \preceq q_{3}$ hold,
we have $P \not \sqsubseteq Q$, that is, (ii) of Theorem \ref{KeyTheoforNAVRP} does not hold.
\end{ex}

From Theorem \ref{KeyTheoforNAVRP} and Lemma \ref{Case_k+2}, we have the following theorem.

\begin{thm}\label{MainTheforNAVRP}
Let $k\ge 2$ and $\sharp\Sigma \ge k+2$.
Then, the set $\NAVRPLkei$ has compactness with respect to containment.
\end{thm}