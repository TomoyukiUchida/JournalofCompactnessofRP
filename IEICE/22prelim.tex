%\section{特徴集合としての$S_{1}(P)$}
%この章では,$\sharp\Sigma \ge 2k+1$のとき,
%$S_{1}(P)$は$\mathcal{RPL}^{k}$における$L(P)$の特徴集合となることを示す..

% $\RPat^{k}$について,任意のパターン$p\in \RPatplus$に対し,
% ある特定の有限部分集合$S \subseteq L(p)$が存在して,
% $S \subseteq L(Q)$ならば,
% ある$q \in Q$に対して$L(p) \subseteq L(q)$となることが知られている\cite{Mukouchi1991}.
% また,$S \subseteq L(Q)$ならば$L(p) \subseteq L(Q)$である.
% これにより,$S$は次の定義される$L(p)$の特徴集合であることがわかる.
% このような集合$S$は$L(p)$の特徴集合と呼ばれ,次のように定義される.

% \begin{dfn}
%     $\mathcal{L}$を言語クラスとする.$L$を$\mathcal{L}$に属する言語とする.
%     空でない有限部分集合$S \subseteq \Sigma^{+}$は$\mathcal{L}$における$L$の\textbf{特徴集合}であるとは,
%     任意の$L^{\prime} \in \mathcal{L}$に対して$S \subseteq L^{\prime}$ならば$L \subseteq L^{\prime}$となるときをいう.
% \end{dfn}
% \noindent
\begin{dfn}
    Let $\L$ be a class of languages, $L$ a language in $\L$ and
    $S$ a nonempty finite set of $L$.
    We say that $S$ is a \textit{characteristic} set for $L$ within $\L$
    if for any $L' \in \L$,
    $S \subseteq L'$ implies $L \subseteq L'$.
\end{dfn}

%
%Mukouchi showed that 
%この2つの性質は,次のように定義される.
%\begin{dfn}[Wright\cite{Wright} and Motoki et al.\cite{Motoki1}]\label{fe}
%言語族$\mathcal{L}$が有限弾力性を持つとは,全ての$i \ge 1$に対して,次のような文字列の無限列($w_{i})_{i \ge 0}$と$\mathcal{L}$に含まれる言語の無限列($L_{i})_{i \ge 1}$が存在しないことをいう.
%\begin{equation*}
%{ w_{0}, \cdots , w_{i-1} } \subseteq L_{i}, ただし,w_{i} \not \in L_{i}
%\end{equation*}
%\end{dfn}
%
%\begin{dfn}[Sato\cite{Sato2}]\label{fc}
%$\mathcal{L}$を言語族とする.言語$L$が$\mathcal{L}$における有限交差性を持つとは,次のような文字列の有限集合の無限列$(T_{n})_{n \ge 1}$と$\mathcal{L}$に含まれる言語の無限列$(L_{i})_{i \ge 1}$が存在しないことをいう.
%\begin{align*}
%(\mathrm{i}) T_{1} \varsubsetneq T_{2} \varsubsetneq \cdots, (\mathrm{ii}) \cup_{i=1}^{\infty}T_{i}=L,
%(\mathrm{iii}) T_{i} \subseteq L_{i}, \\ただし,T_{i+1} \not \subseteq L_{i} (i \ge 1)
%\end{align*}
%\end{dfn}
%また,次の2つの補題が与えられる.
%\begin{lem}[Sato\cite{Sato2}]\label{fcfe}
%$\mathcal{L}$を言語族,$L$を言語とする.
%\begin{comment}
%\begin{align*}
%Lは\mathcal{L}における有限交差性を持つ \ \ \ \\
%\Leftrightarrow \ \ \ \ \ \ \ \ \ \ \ \ \ \ \ \ \ \ \ \ \ \ \ \\
%全てのLは\mathcal{L}における有限弾力性を持つ
%\end{align*}
%\end{comment}
%\begin{align*}
%Lは\mathcal{L}における有限交差性を持つ \Leftrightarrow \\
%全てのLは\mathcal{L}における有限弾力性を持つ
%\end{align*}
%\end{lem}
%\begin{lem}[Sato et al.\cite{Sato1}]\label{fcc}
%$\mathcal{L}$を言語族,$L$を言語とする.
%\begin{align*}
%Lは\mathcal{L}における有限交差性を持つ \Leftrightarrow \\ 
%\mathcal{L}におけるLの特徴集合が存在する
%\end{align*}
%\end{lem}
%Wright\cite{Wright}は$\mathcal{PL}^{k}$,その部分集合である$\mathcal{RPL}^{k}$が有限弾力性を持つことを示した.
%したがって,補題\ref{fcfe}, 補題\ref{fcc}より, $\mathcal{RPL}^{k}$における$L$の特徴集合が存在する.

% $m~(m\geq 0)$個の変数記号$x_{1},\ldots, x_{m}$を含む正規パターン$p$と$n~(n\geq 1)$に対して,
%次のような$L(p)$の部分集合$S_{n}(p)$を定義する.
Let $n$ be a positive integer and $p$ a regular pattern. % and $x_{1},\ldots,x_{m}$ ($m \geq 0$) variable symbols appearing in $p$.
We denote by $S_{n}(p)$ the set of all strings in $\Sigma^{\ast}$ obtained by replacing all variable symbols in $p$
with strings in $\Sigma^{+}$ of which the length is at most $n$.
Moreover, for a set $P \in \RPatplus$, we define $S_{n}(P)$ as follows:
\[
    S_{n}(P)=\bigcup_{p \in P}S_{n}(p).
\]
It is clear that $S_{n}(P) \subseteq S_{n+1}(P) \subseteq L(P)$ for any positive integer $n$.
% %\begin{dfn}\label{部分集合}
% $p$中の各変数記号に長さが高々$n$の$\Sigma^{+}$の定数記号列を代入して得られるすべての定数記号列の集合を$S_{n}(p)$で表す.
% さらに,正規パターンの空でない有限集合$P$に対して,
% %\begin{equation*}
% $S_{n}(P)= \bigcup_{p \in P} S_{n}(p)$
% %\end{equation*}
% %\end{dfn}
%
% %\begin{dfn}\label{部分集合}
% $p$中の各変数記号に長さが高々$n$の$\Sigma^{+}$の定数記号列を代入して得られるすべての定数記号列の集合を$S_{n}(p)$で表す.
% さらに,正規パターンの空でない有限集合$P$に対して,
% %\begin{equation*}
% $S_{n}(P)= \bigcup_{p \in P} S_{n}(p)$
% %\end{equation*}
% %\end{dfn}
%
% とする.
% このとき,
% 任意の自然数$n~(n \ge 1)$に対して, $S_{n}(P) \subseteq S_{n+1}(P) \subseteq L(P)$である.
% %$L(P)$の特徴集合は有限集合であるため,
% よって,次の定理が成り立つ.
% \begin{thm}[Sato et al.\cite{Sato1}]
% 任意の$P \in \RPat^{k}$に対して,$S_{n}(P)$がクラス$\RPatL^{k}$内の正規パターン言語$L(P)$の特徴集合であるような自然数$n~(n \ge 1)$が存在する.
% \end{thm}

\begin{thm}[Sato et al.\cite{Sato1}]
    Let $k$ be a positive integer and $P \in \RPatkei$.
    Then, there exists a positive integer $n$ such that $S_{n}(P)$ is a characteristic set for $L(P)$ within $\RPatLkei$.
    %任意の$P \in \RPat^{k}$に対して,$S_{n}(P)$がクラス$\RPatL^{k}$内の正規パターン言語$L(P)$の特徴集合であるような自然数$n~(n \ge 1)$が存在する.
\end{thm}

Let $p_{1}$, $p_{2}$, $r$, $q$ be regular patterns with $p_{1}rp_{2} \preceq q$ and
$x_{1},\ldots,x_{n}$ variable symbols appearing in $q$.
In~\cite{Mukouchi1991}, the regular pattern $r$ in $p_{1}rp_{2}$ is said to be generated from $q$ by variable substitution if
there exist a variable symbol $x_{i}$ and a substitution
$\theta = \{x_{1}:=r_{1},\ldots,x_{i}:=r'rr'',\ldots,x_{n}:=r_{n}\}$ such that
$p_{1} = (q_{1}\theta)r'$, $p_{2} = r''(q_{2}\theta)$ for $q = q_{1}x_{i}q_{2}$.
It is clear that $p_{1}xp_{2} \preceq q$ if the regular pattern $r$ in $p_{1}rp_{2}$ is generated from $q$ by variable substitution.
%
% $p_{1},p_{2},r,q$を正規パターンとし,
% $p_{1}rp_{2} \preceq q$が成り立つとする.
% また,$x_{1}, \ldots, x_{n}$を$q$に含まれる変数記号とする.
% このとき,$q=q_{1}x_{i}q_{2}$に対して,$p_{1}=(q_{1} \theta )r^{\prime}$かつ$p_{2}=r^{\prime\prime}(q_{2} \theta )$を満たす変数記号$x_{i}$と代入$\theta =
% \{ x_{1} := r_{1}, \ldots , x_{i} := r^{\prime}rr^{\prime\prime}, \ldots , x_{n} := r_{n} \}$が存在すれば,$p_{1}rp_{2}$に含まれる正規パターン$r$は$q$の変数記号への代入により生成できる.
% よって,$p_{1}rp_{2}$に含まれる正規パターン$r$が$q$の変数記号への代入により生成できるとき,$p_{1}xp_{2} \preceq q$が成り立つ.

% \begin{lem}[Sato et al.\cite{Sato1}]\label{補題9}
%     $p=p_{1}xp_{2}, \ q=q_{1}q_{2}q_{3}$を正規パターンとする.
%     以下の{\rm (i), (ii), (iii)}がすべて成り立つとき,$p \preceq q$である.
%     \[
%     \begin{tabular}{ll}
%     $(\mathrm{i})$ $p_{1} \preceq q_{1}q_{2},$
%     $(\mathrm{ii})$ $p_{2} \preceq q_{2}q_{3},$\\
%     $(\mathrm{iii})$ $q_{2}$は変数記号を含む.
%     \end{tabular}
%     \]
% \end{lem}

\begin{thm}[Sato et al.\cite{Sato1}]\label{Sato1:Lemma9}
    Let $p$, $q$, $p_{1}$, $p_{2}$, $q_{1}$, $q_{2}$, $q_{3}$ be regular patterns and $x$ a variable symbol with
    $p = p_{1}xp_{2}$ and $q = q_{1}q_{2}q_{3}$.
    Then $p \preceq q$ if the following three conditions are holds:
    \[
        \begin{tabular}{ll}
            $(\mathrm{i})$ $p_{1} \preceq q_{1}q_{2},$
            $(\mathrm{ii})$ $p_{2} \preceq q_{2}q_{3},$ \\
            $(\mathrm{iii})$ $q_{2}$ contains at least one variable symbol.
        \end{tabular}
    \]
    %     $p=p_{1}xp_{2}, \ q=q_{1}q_{2}q_{3}$を正規パターンとする.
    %     以下の{\rm (i), (ii), (iii)}がすべて成り立つとき,$p \preceq q$である.
    %     \[
    %     \begin{tabular}{ll}
    %     $(\mathrm{i})$ $p_{1} \preceq q_{1}q_{2},$
    %     $(\mathrm{ii})$ $p_{2} \preceq q_{2}q_{3},$\\
    %     $(\mathrm{iii})$ $q_{2}$は変数記号を含む.
    %     \end{tabular}
    %     \]    
\end{thm}

% \begin{proof}
% $y$を$q_{2}$に含まれる変数記号とし,$q_{2}=q_{2}^{\prime}yq_{2}^{\prime \prime}$とする.
% $p_{1} \preceq q_{1}q_{2}=q_{1}(q_{2}^{\prime}yq_{2}^{\prime \prime})$より,$p_{1}^{\prime} \preceq q_{1}q_{2}^{\prime}$かつ$p_{1}^{\prime\prime} \preceq yq_{2}^{\prime\prime}$となるような$p_{1}^{\prime},~p_{1}^{\prime\prime}$を定義すると,$p_{1}=p_{1}^{\prime}p_{1}^{\prime\prime}$となる.
% 同様に,$p_{2} \preceq q_{2}q_{3}=(q_{2}^{\prime}yq_{2}^{\prime\prime})q_{3}$より,$p_{2}^{\prime} \preceq q_{2}^{\prime}y$かつ$p_{2}^{\prime\prime} \preceq q_{2}^{\prime\prime}q_{3}$となるような$p_{2}^{\prime},~p_{2}^{\prime\prime}$を定義すると,$p_{2}=p_{2}^{\prime}p_{2}^{\prime\prime}$となる.
% このとき,$p=p_{1}xp_{2}=p_{1}^{\prime}(p_{1}^{\prime\prime}xp_{2}^{\prime})p_{2}^{\prime\prime} \preceq q_{1}q_{2}^{\prime}(p_{1}^{\prime\prime}xp_{2}^{\prime})q_{2}^{\prime\prime}q_{3}=q\theta \preceq q$となる.
% \end{proof}

Let $p_{1}$, $p_{2}$, $q$ be regular patterns and $a$ a constant symbol with $p_{1}ap_{2} \preceq q$.
If $p_{1}xp_{2} \not\preceq q$, then the constant symbol $a$ in $p_{1}ap_{2}$ is not generated from $q$ by variable substitution.
Thus $q  = q_{1}aq_{2}$ for some regular patterns such that $p_{1} \preceq q_{1}$ and $p_{2} \preceq q_{2}$.
From the above, the following lemma holds.
%
% ある$a \in  \Sigma$に対して,$p \{ x:=a \} \preceq q$のとき,$p_{1}xp_{2} \not \preceq q$ならば,$p_{1}ap_{2}$の定数記号$a$は,$q$の変数記号への代入によって生成することはできない.
% すなわち,$p_{1} \preceq q_{1}$かつ$p_{2} \preceq q_{2}$を満たす$q=q_{1}aq_{2}$が存在する.
% これにより,次の補題が得られる.
% \begin{lem}[Sato et al.\cite{Sato1}]\label{補題10}
% $\sharp \Sigma \ge 3$,~$p=p_{1}xp_{2},~q$を正規パターン,$a,~b,~c$を$\Sigma$に属する相異なる定数記号とする.
% このとき,$p_{1}ap_{2} \preceq q$,~$\ p_{1}bp_{2} \preceq q$かつ$p_{1}cp_{2} \preceq q$が成り立つならば,
% $p\preceq q$が成り立つ.
% \end{lem}
\begin{lem}[Sato et al.\cite{Sato1}]\label{Sato1:Lemma10}
    Suppose $\sharp \Sigma \geq 3$.
    Let $p$, $p_{1}$, $p_{2}$, $q$ are regular patterns and $x$ a variable symbol with $p = p_{1}xp_{2} \preceq p$
    Let $a$, $b$ and $c$ be mutually distinct constant symbols.
    If $p_{1}ap_{2} \preceq q$, $\ p_{1}bp_{2} \preceq q$ and $p_{1}cp_{2} \preceq q$, then $p \preceq q$ holds.
\end{lem}
% \begin{proof}
% $p \not \preceq q$と仮定する.
% このとき,$p_{1}ap_{2}$の$a$,~$p_{1}bp_{2}$の$b$,~$p_{1}cp_{2}$の$c$は$q$の変数記号を置き換えることによって生成できない.
% よって,
% \medskip

% \begin{tabular}{llll}
% (1) & $p_{1} \preceq q_{1}$ & (1') & $p_{2} \preceq q_{2}bq_{3}cq_{4}$ \\
% (2) & $p_{1} \preceq q_{1}aq_{2}$ & (2') & $p_{2} \preceq q_{3}cq_{4}$ \\
% (3) & $p_{1} \preceq q_{1}aq_{2}bq_{3}$ & (3') & $p_{2} \preceq q_{4}$
% \end{tabular}	
% \indent ($q_{1}, q_{2}, q_{3}, q_{4}$は正規パターン) 	
% \medskip

% \noindent を満たす$q=q_{1}aq_{2}bq_{3}cq_{4}$が存在する.
% (2)と(1')より,$q_{2}$に変数記号が含まれる場合,補題\ref{補題9}より,$p \preceq q$となる.
% これは仮定に矛盾する.
% よって,$q_{2}$は定数記号列である.同様に,(3)と(2')より,$q_{3}$は定数記号列である.
% したがって,$w=q_{2}, w^{\prime}=q_{3}$ ($w, w^{\prime}$は定数記号列)とおく.

% $|w|=|w^{\prime}|$のとき,(2)と(3)より,$p_{1}$の接尾辞は$awbw^{\prime}$かつ$aw$である.
% $|w|=|w^{\prime}|$より,$bw^{\prime}=aw$である.
% これは,$b=a$となり,$a, b$が互いに異なる定数記号であることに矛盾する.

% $|w| < |w^{\prime}|$のとき,(2)と(3)より,$p_{1}$の接尾辞は$awbw^{\prime}$かつ$aw$である.
% $w^{\prime}=w_{1}w$とおくと,$awbw^{\prime}=awbw_{1}w$となる.
% このとき,$w_{1}$の最後の記号は$a$となる.
% (1')と(2')より,$p_{2}$の接頭辞は$wbw^{\prime}c$かつ$w^{\prime}c$である.
% $w^{\prime}=w_{1}w$とおくと,$wbw^{\prime}c=wbw_{1}wc$となり,$w^{\prime}=ww_{2}$とおくと,$w^{\prime}c=ww_{2}c$となる.
% $|wbw_{1}|=|ww_{2}c|$より,$w_{1}$の最後の記号は$c$となる.
% よって,$w_{1}$の接尾辞は$a=c$となる.
% これは,$a, c$が互いに異なる定数記号であることに矛盾する.

% $|w| > |w^{\prime}|$のとき,(1')と(2')より,$p_{2}$の接頭辞は$wbw^{\prime}c$かつ$w^{\prime}c$である.
% $w=w^{\prime}w_{1}$とおくと,$wbw^{\prime}c=w^{\prime}w_{1}bw^{\prime}c$となる.
% このとき,$w_{1}$の最初の記号は$c$となる.
% (2)と(3)より,$p_{1}$の接尾辞は$awbw^{\prime}$と$aw$である.
% $w=w^{\prime}w_{1}$とおくと,$awbw^{\prime}=aw^{\prime}w_{1}bw^{\prime}$となり, $w=w_{2}w^{\prime}$とおくと,$aw=aw_{2}w^{\prime}$となる.
% $|w_{1}bw^{\prime}|=|aw_{2}w^{\prime}|$より,$w_{1}$の最初の記号は$a$となる.
% よって,$a=c$となる.
% これは,$a, c$が互いに異なる定数記号であることに矛盾する.
% \end{proof}
%
% 次の補題\ref{2個}は,相異なる定数記号$a, b$に対して,$p\{x:=a\} \preceq q$ かつ $p \{x:=b\} \preceq q$ ならば $p \not\preceq q$
% となる正規パターン $p, q$ が存在することを示している.
%
% \begin{lem}[Sato et. al.\cite{Sato1}]\label{Sato1:Lemma13}
% $\sharp\Sigma \ge 3$とする.%$p, q$を正規パターン, 
% $a, b$を相異なる定数記号とする.
% 次の条件{\rm (i), (ii), (iii)}を満たす正規パターン$p=p_{1}AwxwBp_{2}$と$q=q_{1}AwBq_{2}$に対して,
% $p \{ x:= a \} \preceq q$かつ$p \{ x:=b \} \preceq q$ならば$p \not \preceq q$である.
% ここで,$p_{1}, p_{2}, q_{1}, q_{2}$は正規パターン,$w$は定数記号列である.
% \[
%     \begin{tabular}{ll}
%         $\mathrm{(i)}~p_{1} \preceq q_{1}$,~$\mathrm{(ii)}~p_{2} \preceq q_{2}$,\\
%         $\mathrm{(iii)}~A=a, B=b$ または $A=b, B=a$.
%     \end{tabular}
% \]
% \end{lem}
\begin{lem}[Sato et al.\cite{Sato1}]\label{Sato1:Lemma13}
    Suppose $\sharp\Sigma \geq 3$.
    Let $p_{1}$, $p_{2}$, $q_{1}$, $q_{2}$ be regular patterns and $x$ a variable symbol.
    Let $a$, $b$ be constant symbols with $a \neq b$ and $w$ a string in $\Sigma^{\ast}$.
    Let $p = p_{1}AwxwBp_{2}$ and $q = q_{1}AwBq_{2}$ be regular patterns satisfies the following three conditions:
    \[
        \begin{tabular}{ll}
            $\mathrm{(i)}~p_{1} \preceq q_{1}$,  \\
            $\mathrm{(ii)}~p_{2} \preceq q_{2}$, \\
            $\mathrm{(iii)}~A=a, B=b$ or $A=b, B=a$.
        \end{tabular}
    \]
    If $p\{x:=a\} \preceq q$ and $p\{x:=b\} \preceq q$,
    then we have $p \not\preceq q$.
\end{lem}

% \begin{comment}
% \begin{proof}
% $p=p_{1}^{\prime}xp_{2}^{\prime} \ (p_{1}^{\prime}, p_{2}^{\prime}$は正規パターン)とする.
% 補題\ref{補題10}と同様に考えると, $p \{ x:= a \} \preceq q, \ p \{ x:=b \} \preceq q, \ p \not \preceq q$より,
% \medskip

% \indent$(1) \ p_{1}^{\prime} \preceq q_{1}, \ (1^{\prime}) \ p_{2}^{\prime} \preceq wBq_{2}$ \\
% \indent $(2) \ p_{1}^{\prime} \preceq q_{1}Aw, \ (2^{\prime}) \ p_{2}^{\prime}
%  \preceq q_{2}$ 
% \medskip

% \noindent を満たす$q=q_{1}AwBq_{2}$が存在する.

% (1), (2), (1$^{\prime}), (2^{\prime})$より,$p_{1}^{\prime}=p_{1}Aw, \ p_{2}^{\prime} = wBp_{2}$ ($p_{1} \preceq q_{1}, \ p_{2} \preceq q_{2}$)とおける.
% よって,$p=p_{1}^{\prime}xp_{2}^{\prime}=p_{1}AwxwBp_{2}$となる.
% \end{proof}
% \end{comment}

% 補題\ref{補題10}より,次の定理が成り立つ.
% \begin{thm}[Sato et al.\cite{Sato1}]\label{定理10}
% $\sharp \Sigma \ge 2k+1$とし,$P \in \RPatplus,~Q \in \RPat^{k}$とする.
% このとき,次の{\rm (i), (ii), (iii)}は同値である.
% \[
% \begin{tabular}{ll}
% $(\mathrm{i})$ $S_{1}(P) \subseteq L(Q),$
% $(\mathrm{ii})$ $P \sqsubseteq Q,$
% $(\mathrm{iii})$ $L(P) \subseteq L(Q).$
% \end{tabular}
% \]
% \end{thm}

From Lemma~\ref{補題10}, the following lemma holds.
\begin{thm}[Sato et al.\cite{Sato1}]\label{定理10}
    Let $\sharp\Sigma \geq 2k+1$, $P \in \RPatplus$ and $Q \in \RPatkei$.
    Then, the following (i), (ii) and (iii) are equivalent:
    %$\sharp \Sigma \ge 2k+1$とし,$P \in \RPatplus,~Q \in \RPat^{k}$とする.
    %このとき,次の{\rm (i), (ii), (iii)}は同値である.
    \[
        (\mathrm{i})\ S_{1}(P) \subseteq L(Q),\ \
        (\mathrm{ii})\ P \sqsubseteq Q,\ \
        (\mathrm{iii})\ L(P) \subseteq L(Q).
    \]
\end{thm}
%
%In case $\sharp\Sigma = 2k$,

Example~1 in~\cite{Sato1} is given as a counter-example of Theorem~\ref{定理10}.
%次の例は,$\sharp \Sigma = 2k$における定理\ref{定理10}の反例である.
% \begin{ex}\label{例題1}
%     Let $k$ be a positive integer and
%     $\Sigma = \{a_{1},\ldots,a_{k},b_{1},\ldots,b_{k}\}$.
%     %$p$ a regular pattern and $Q = \{q_{1},\ldots,q_{k}\} \in \RPatkei$.
%     % $\Sigma = \{ a_{1}, \ldots , a_{k}, b_{1}, \ldots , b_{k} \}$を
%     % $2k$個の定数記号から成る集合,$p$を正規パターン,
%     % $Q = \{ q_{1}, \ldots , q_{k} \}$とする.
%     We define $w_{1},\ldots,w_{k}$ recursively as follows:
%     %$w_{1}, \ldots , w_{k}$を
%     $w_{i} = w_{i+1}b_{i+1}a_{i+1}w_{i+1}$ $(i = 1,2,\ldots,k-1)$, and
%     $w_{k} = \varepsilon$.
%     Let $p$, $q_{i}$ ($i=1,2,\ldots,k$) regular patterns as follows:
%     $p = x_{1}a_{1}w_{1}xw_{1}b_{1}x_{2}$, $q_{i} = x_{1}a_{i}w_{i}b_{i}x_{2}$.

%     $p \{ x:=a_{i} \} \preceq q_{i}$かつ
%     $p \{ x:=b_{i} \} \preceq q_{i}$ $(i = 1,2, \ldots , k)$である場合を考える.
%     $i=1$のとき,$p \{ x:=a_{1} \} = (x_{1}a_{1}w_{1})a_{1}(w_{1}b_{1}x_{2}) = q_{1} \{ x_{1} := x_{1}a_{1}w_{1} \} \preceq q_{1}$かつ$p \{ x:=b_{1} \} = q_{1} \{ x_{2} := w_{1}b_{1}x_{2} \} \preceq q_{1}$となる.
%     $i \ge 2$のとき,$w_{i}$の定義より,ある記号列$w^{(i)},w^{\prime (i)}$に対して,$w_{1} = (w_{i}b_{i})w^{(i)} = w^{\prime (i)}(a_{i}w_{i})$となる.
%     したがって,任意の$i~(i \ge 2)$に対して,
%     \begin{eqnarray*}
%         p \{ x:=a_{i} \} & = & (x_{1}a_{1}w_{1})a_{i}(w_{1}b_{1}x_{2})\\
%         & = & (x_{1}a_{1}w_{1})a_{i}(w_{i}b_{i}w^{(i)})b_{1}x_{2}\\
%         & = & (x_{1}a_{1}w_{1})(a_{i}w_{i}b_{i})(w^{(i)}b_{1}x_{2})\\
%         & = & q_{i} \{ x_{1} := x_{1}a_{1}w_{1}, x_{2} := w^{(i)}b_{1}x_{2} \}\\
%         & \preceq & q_{i},\\
%         p \{ x:=b_{i} \} & = & (x_{1}a_{1}w_{1})b_{i}(w_{1}b_{1}x_{2})\\
%         & = & x_{1}a_{1}(w^{\prime (i)}a_{i}w_{i})b_{i}(w_{1}b_{1}x_{2}) \\
%         & = & (x_{1}a_{1}w^{\prime (i)})a_{i}w_{i}b_{i}(w_{1}b_{1}x_{2}) \\
%         & = & q_{i} \{ x_{1} := x_{1}a_{1}w^{\prime (i)}, x_{2} := w_{1}b_{1}x_{2} \}\\
%         & \preceq & q_{i}.
%     \end{eqnarray*}

%     したがって,$S_{1}(p) \subseteq L(Q)$である.
%     一方で,$p \not \preceq q_{i}$であるため,
%     $L(p) \not \subseteq L(q_{i})$ $(i=1, \ldots , k)$である.
% \end{ex}

% 定理\ref{定理10}より,次の系が得られる.
% \begin{col}[Sato et al.\cite{Sato1}]
% $\sharp \Sigma \ge 3$とし,$p,q$を正規パターンとする.
% このとき,次の{\rm (i), (ii), (iii)}は同値である.
% \[
% \begin{tabular}{ll}
% $(\mathrm{i})$ $S_{1}(p) \subseteq L(q),$
% $(\mathrm{ii})$ $p \preceq q,$
% $(\mathrm{iii})$ $L(p) \subseteq L(q),$
% \end{tabular}
% \]
% \end{col}

From Theorem~\ref{定理10}, we have the following corollary.
\begin{col}[Sato et al.\cite{Sato1}]
    Let $\sharp\Sigma \geq 3$ and $p$, $q$ regular patterns.
    % $\sharp \Sigma \ge 3$とし,$p,q$を正規パターンとする.
    Then, the following (i), (ii) and (iii) are equivalent:
    %このとき,次の{\rm (i), (ii), (iii)}は同値である.
    \[
        (\mathrm{i})\ S_{1}(p) \subseteq L(q),\ \
        (\mathrm{ii})\ p \preceq q,\ \
        (\mathrm{iii})\ L(p) \subseteq L(q).
    \]
\end{col}