\begin{summary}
正規パターンとは,定数記号と変数記号から成る,各変数記号が高々1回しか出現しない記号列をいう.
正規パターン$p$の変数記号を定数記号列で置き換えることで生成できる定数記号列全体の集合を$L(p)$で表す.
高々$k$個の正規パターンの集合全体のクラス$\RPat^{k}$に属する正規パターン集合$P$と$Q$に対し,
任意の正規パターン$p\in P$に対して,
$p$より汎化された正規パターン$q$が$Q$に存在するとき,$P\sqsubseteq Q$と書く.
高々$k~(k\geq 2)$個の正規パターンから成る集合の全体のクラスを$\RPatkei$で表す.
1998年にSatoら\cite{Sato1}は,各変数記号に対し,長さが高々2の記号列を代入することで$P\in \RPatkei$から得られる記号列の有限集合$S_2(P)$が,
$L(P)=\bigcup_{p\in P}L(p)$の特徴集合であることを示した.
次に,定数記号の数が$2k-1$以上のとき,$\RPatkei$が包含関係に関してコンパクト性をもつことを示した.
これらの結果に対し,本稿では,まずSatoら\cite{Sato1}の結果を検証し,Satoらが与えた定理の証明の誤りを修正した.
さらに,隣接した変数(隣接変数)を持たない正規パターンである非隣接変数正規パターン全体の集合$\NAVRP$を与え,
高々$k~(k\ge 1)$個の非隣接変数正規パターンから成る集合の全体のクラス$\NAVRPkei$に属する集合$P$から得られる$S_2(P)$が$L(P)$の特徴集合であることを示した.
さらに,定数記号の数が$k+2$以上のとき,$\NAVRPkei$が包含に関してコンパクト性をもつことを示した.
これにより,正規パターン言語のときよりも少ない数の定数記号で,非隣接変数正規パターン言語の有限和に関する効率的な学習アルゴリズムが設計できることを示した.
\end{summary}
