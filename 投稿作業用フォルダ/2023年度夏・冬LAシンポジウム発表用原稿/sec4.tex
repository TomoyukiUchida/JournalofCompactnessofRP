\section{非隣接変数正規パターン}

隣接した変数記号を持たない正規パターンを\textbf{非隣接変数正規パターン}という.
例えば,パターン$axybc$は正規パターンであるが,非隣接変数正規パターンではない.パターン$axbcy$は非隣接変数正規パターンである.
$\NAVRP$を非隣接変数正規パターン全体の集合とする.
$\NAVRP$の空でない有限部分集合の集合を$\NAVRPplus$で,
高々$k~(k\geq 1)$個のパターンから成る$\NAVRP$の部分集合$\{P\in \NAVRPplus \mid \sharp P \leq k\}$を$\NAVRPkei$で表す.
このとき,次の定理が成り立つ.

\begin{thm}\label{非隣接kが4以上}
$\sharp \Sigma \ge k+2,P\in \NAVRPplus,Q \in \NAVRPkei$とする.
このとき,以下の{\rm (i), (ii), (iii)}は同値である.
\[
\begin{tabular}{ll}
$(\mathrm{i})$ $S_{2}(P) \subseteq L(Q),$
$(\mathrm{ii})$ $P \sqsubseteq Q,$
$(\mathrm{iii})$ $L(P) \subseteq L(Q).$
\end{tabular}
\]
\end{thm}

\begin{proof}
定義より,
(ii) $\Rightarrow$ (iii)と(iii) $\Rightarrow$ (i)は自明に成り立つ.
よって,(i) $\Rightarrow$ (ii)が成り立つことを,$p$に現れる変数記号の数$n$に関する数学的帰納法で証明する.

$n=0$のとき,$S_{2}(p)= \{ p \}$であり,$p \in L(Q)$となる.よって,ある$q \in Q$に対して,$p \preceq q$となる.

$n \ge 0$個の変数記号を含む任意の正規パターンに対して,題意が成り立つと仮定する.
$p$を$S_{2}(p) \subseteq L(Q)$を満たす$n+1$個の変数記号を含む非隣接変数正規パターンとする.
$p \not \preceq q_{i}$ ($i=1, 2$)と仮定する.
非隣接変数正規パターン$p$を$p=p_{1}xp_{2}$, $Q=\{ q_{1}, \ldots , q_{k} \}$とおく.
ここで,$p_{1}$は末尾が定数記号である非隣接変数正規パターンであり,$p_{2}$は先頭が定数記号である非隣接変数正規パターン,$x$は変数記号,任意の$i$ ($i=1, \ldots, k$)に対して,$q_{i}$は非隣接変数正規パターンである.
$a, b \in \Sigma$に対して,$p_{a}=p \{ x := a \}$,$p_{ab}=p \{ x := ab \}$とおく.
このとき,$p_{a}, p_{ab}$は$n$個の変数記号が含まれ,$S_{2}(p_{a}) \subseteq L(Q)$かつ$S_{2}(p_{ab}) \subseteq L(Q)$が成り立つことに注意する.
帰納法の仮定より,任意の$a, b \in \Sigma$に対して,$p_{a} \preceq q_{i}$かつ$p_{ab} \preceq q_{i^{\prime}}$を満たすような$i, i^{\prime} \le k$が存在する.

補題\ref{追加補題1}より,ある$i$に対して$p \{ x:=xy \} \preceq q_{i}$が成り立つ.
このとき,$p \{ x:=xy \} =p_{1}xyp_{2}$の部分パターン$xy$は$q_{i}$の変数記号を置き換えることで生成できない.
このことは,$q_{i}$に$xy$が含まれることを示している.
これは,$q_{i}$が非隣接変数正規パターンであることに矛盾する.

以上より,(i) $\Rightarrow$ (ii)が成り立つ.
\end{proof}

\begin{col}
$\sharp\Sigma \ge k+2$,$P \in \NAVRPplus$とする.このとき,$S_{2}(P)$は$\mathcal{RPL^{\mbox{$k$}}_{NAV}}$における$L(P)$の特徴集合である.
\end{col}

\begin{lem}\label{k+2のとき}
$\sharp\Sigma \le k+1$とする.このとき,$\NAVRPkei$は包含に関してコンパクト性を持たない.
\end{lem}
\begin{proof}
$\Sigma = \{ a_{1}, \ldots , a_{k+1} \}$を$k+1$個の定数記号から成る集合,$p, q_{i}$を正規パターンとする.
$p \{ x := a_{i}y \} \preceq q_{i}$かつ$p \{ x := ya_{i+1} \} \preceq q_{i}~(i=1,2, \ldots ,k)$とする.
$p \{ x:= a_{k+1}a_{1} \} \preceq q_{1}$であるとき,$S_{2}(p) \backslash S_{1}(p) \subseteq \bigcup^{k}_{i=1} L(q_{i})$となる. 
すなわち,$L(p) \subseteq L(Q)$である.
しかし,$p \not \preceq q_{i}$であるため,$L(p) \not \subseteq L(q_{i})~(i=1,2, \ldots k)$である.
したがって,$\NAVRPkei$は包含に関するコンパクト性を持たない.
\end{proof}

コンパクト性をもたない例を例\ref{反例k+1}に示す.
\begin{figure*}[tb]
\begin{ex}\label{反例k+1}
$\Sigma= \{a_{1}, a_{2}, a_{3},a_{4} \}$を$4$つの定数記号から成る集合,$p,q_{1},q_{2},q_{3}$を正規パターン,$x,x^{\prime},x^{\prime\prime}$を変数記号とする.
$p,q_{1},q_{2},q_{3}$を以下のように定義する.
\begin{align*}
p & = x^{\prime}a_{3}a_{1}a_{4}a_{1}a_{4}a_{1}a_{1}a_{4}a_{1}a_{3}a_{2}a_{1}a_{4}a_{1}a_{4}a_{1}a_{1}a_{4}a_{1}xa_{1}a_{4}a_{1}a_{4}a_{1}a_{1}a_{4}a_{1}a_{3}a_{2}a_{1}a_{4}a_{1}a_{4}a_{1}a_{1}a_{4}a_{1}a_{2}x^{\prime\prime},\\
q_{1} & = x^{\prime}a_{3}a_{1}a_{4}a_{1}a_{4}a_{1}a_{1}a_{4}a_{1}a_{3}a_{2}a_{1}a_{4}a_{1}a_{4}a_{1}a_{1}a_{4}a_{1}a_{2}x^{\prime\prime},\\
q_{2} & = x^{\prime}a_{2}a_{1}a_{4}a_{1}a_{4}a_{1}a_{1}a_{4}a_{1}a_{3}x^{\prime\prime},\\
q_{3} & = x^{\prime}a_{1}a_{1}a_{4}a_{1}a_{4}x^{\prime\prime}.
\end{align*}

これは,$L(p) \subseteq L(q_{1}) \cup L(q_{2}) \cup L(q_{3})$となる.
しかし,$p \not \preceq q_{1},p \not \preceq q_{2}$かつ$p \not \preceq q_{3}$である.
\end{ex}
\end{figure*}

定理\ref{非隣接kが4以上}と補題\ref{k+2のとき}より,次の定理が成り立つ.

\begin{thm}
$\sharp\Sigma \ge k+2$とする.
このとき,$\RPat^{k}$は包含に関してコンパクト性を持つ.
\end{thm}